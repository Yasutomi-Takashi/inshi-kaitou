\documentclass[
		book,
		head_space=20mm,
		foot_space=20mm,
		gutter=10mm,
		line_length=190mm
]{jlreq}

%----------
%LuaLaTeXで実行する!!
%----------
%各章節には以下を書く. 1-03.texのような名前にする
%----------
% \documentclass[
% 		book,
% 		head_space=20mm,
% 		foot_space=20mm,
% 		gutter=10mm,
% 		line_length=190mm
% ]{jlreq}
% 
%----------
%LuaLaTeXで実行する!!
%----------
%各章節には以下を書く. 1-03.texのような名前にする
%----------
% \documentclass[
% 		book,
% 		head_space=20mm,
% 		foot_space=20mm,
% 		gutter=10mm,
% 		line_length=190mm
% ]{jlreq}
% 
%----------
%LuaLaTeXで実行する!!
%----------
%各章節には以下を書く. 1-03.texのような名前にする
%----------
% \documentclass[
% 		book,
% 		head_space=20mm,
% 		foot_space=20mm,
% 		gutter=10mm,
% 		line_length=190mm
% ]{jlreq}
% \input {preamble.tex}
% \usepackage{docmute} %ファイル分割
% \begin{document}

% %\chapter{章のタイトル}
% \section{節のタイトル}
% no text

% \end{document}
%----------

%main.texには以下を書く
%----------
% \documentclass[
% 		book,
% 		head_space=20mm,
% 		foot_space=20mm,
% 		gutter=10mm,
% 		line_length=190mm,
%         openany
% ]{jlreq}
% \input {preamble.tex}
% \usepackage{docmute} %ファイル分割
% \begin{document}

% %---------- 1章1節
% \input 1-01.tex
% %---------- 1章2節
% \input 1-02.tex
% % ---------- 1章3節
% \input 1-03.tex
% % ---------- 1章4節
% \input 1-04.tex
% % ---------- 1章5節
% \input 1-05.tex
% % ---------- 1章6節
% \input 1-06.tex
% %---------- 1章7節
% \input 1-07.tex
% % ---------- 1章8節
% \input 1-08.tex
% % ---------- 1章9節
% \input 1-09.tex
% % ---------- 1章10節
% \input 1-10.tex
% % ---------- 1章11節
% \input 1-11.tex
% % ---------- 1章12節
% \input 1-12.tex
% % ---------- 参考文献
% \input reference.tex
% \end{document}
% ----------



\usepackage{bxtexlogo}
\usepackage{amsthm}
\usepackage{amsmath}
\usepackage{bbm} %小文字の黒板文字
\usepackage{physics}
\usepackage{amsfonts}
\usepackage{graphicx}
\usepackage{mathtools}
\usepackage{enumitem}
\usepackage[margin=20truemm]{geometry}
\usepackage{textcomp}
\usepackage{bm}
\usepackage{mathrsfs}
\usepackage{latexsym}
\usepackage{amssymb}
\usepackage{algorithmic}
\usepackage{algorithm}
\usepackage{tikz}
\usetikzlibrary{arrows.meta}
\usetikzlibrary{math,matrix,backgrounds}
\usetikzlibrary{angles}
\usetikzlibrary{calc}


%----------
%日本語フォント
% \usepackage[deluxe]{otf} platex用 lualatexでは動かない

%----------
%欧文フォント
\usepackage[T1]{fontenc}

%----------
%文字色
\usepackage{color}

%----------
\setlength{\parindent}{2\zw} %インデントの設定

%----------
% %参照した数式にだけ番号を振る cleverrefと併用するとうまくいかない
% \mathtoolsset{showonlyrefs=true}
%----------

%----------
%集合の中線
\newcommand{\relmiddle}[1]{\mathrel{}\middle#1\mathrel{}}
% \middle| の代わりに \relmiddle| を付ける
\newcommand{\sgn}{\mathop{\mathrm{sgn}}} %置換sgn
\newcommand{\Int}{\mathop{\mathrm{Int}}} %位相空間の内部Int
\newcommand{\Ext}{\mathop{\mathrm{Ext}}} %位相空間の外部Ext
\newcommand{\Cl}{\mathop{\mathrm{Cl}}} %位相空間の閉包Cl
\newcommand{\supp}{\mathop{\mathrm{supp}}} %関数の台supp
\newcommand{\restrict}[2]{\left. #1 \right \vert_{#2}}%関数の制限 \restrict{f}{A} = f|_A
\newcommand{\Ker}{\mathop{\mathrm{Ker}}}
\newcommand{\Coker}{\mathop{\mathrm{Coker}}}
\newcommand{\coker}{\mathop{\mathrm{coker}}}
\newcommand{\Coim}{\mathop{\mathrm{Coim}}}
\newcommand{\coim}{\mathop{\mathrm{coim}}}
\newcommand{\id}{\mathop{\mathrm{id}}}
\newcommand{\Gal}{\mathop{\mathrm{Gal}}}

\newtheorem{definition}{定義}[section]

\usepackage{aliascnt}

% \newaliastheorem{(環境とカウンターの名前)}{(元となるカウンターの名前)}{(表示される文字列)}
\newcommand*{\newaliastheorem}[3]{%
  \newaliascnt{#1}{#2}%
  \newtheorem{#1}[#1]{#3}%
  \aliascntresetthe{#1}%
  \expandafter\newcommand\csname #1autorefname\endcsname{#3}%
}
\newaliastheorem{proposition}{definition}{命題} 
\newaliastheorem{theorem}{definition}{定理}
\newaliastheorem{lemma}{definition}{補題}
\newaliastheorem{corollary}{definition}{系}
\newaliastheorem{example}{definition}{例}
\newaliastheorem{practice}{definition}{演習問題}

\newtheorem*{longproof}{証明}
\newtheorem*{answer}{解答}
\newtheorem*{supplement}{補足}
\newtheorem*{remark}{注意}
%----------

%----------
%古い記法を注意するパッケージ
\RequirePackage[l2tabu, orthodox]{nag}
%----------


% 定理環境(tcolorbox)
\usepackage{tcolorbox} %箱
\tcbuselibrary{breakable,skins,theorems}
\tcolorboxenvironment{definition}{
	blanker,breakable,
	left=3mm,right=3mm,
	top=2mm,bottom=2mm,
	before skip=15pt,after skip=20pt,
	borderline ={0.5pt}{0pt}{black}
}
\newtcolorbox{emptydefinition}{
	blanker,breakable,
	left=3mm,right=3mm,
	top=2mm,bottom=2mm,
	before skip=15pt,after skip=20pt,
	borderline ={0.5pt}{0pt}{black}
}
%----------
\tcolorboxenvironment{proposition}{
	blanker,breakable,
	left=3mm,right=3mm,
	top=3mm,bottom=3mm,
	before skip=15pt,after skip=15pt,
	borderline={0.5pt}{0pt}{black}
}
\newtcolorbox{emptyproposition}{
	blanker,breakable,
	left=3mm,right=3mm,
	top=3mm,bottom=3mm,
	before skip=15pt,after skip=15pt,
	borderline={0.5pt}{0pt}{black}
}
%----------
\tcolorboxenvironment{theorem}{
	blanker,breakable,
	left=3mm,right=3mm,
	top=3mm,bottom=3mm,
    sharp corners,boxrule=0.6pt,
	before skip=15pt,after skip=15pt,
	borderline={0.5pt}{0pt}{black},
    borderline={0.5pt}{1.5pt}{black}
}
\newtcolorbox{emptytheorem}{
	blanker,breakable,
	left=3mm,right=3mm,
	top=3mm,bottom=3mm,
    sharp corners,boxrule=0.6pt,
	before skip=15pt,after skip=15pt,
	borderline={0.5pt}{0pt}{black},
    borderline={0.5pt}{1.5pt}{black}
}
%----------
\tcolorboxenvironment{lemma}{
	blanker,breakable,
	left=3mm,right=3mm,
	top=3mm,bottom=3mm,
	before skip=15pt,after skip=15pt,
	borderline={0.5pt}{0pt}{black}
}
%----------
\tcolorboxenvironment{corollary}{
	blanker,breakable,
	left=3mm,right=3mm,
	top=3mm,bottom=3mm,
	before skip=15pt,after skip=15pt,
	borderline={1.0pt}{0pt}{black,dotted}
}
\newtcolorbox{emptycorollary}{
	blanker,breakable,
	left=3mm,right=3mm,
	top=3mm,bottom=3mm,
	before skip=15pt,after skip=15pt,
	borderline={1.0pt}{0pt}{black,dotted}
}
%----------
\tcolorboxenvironment{example}{
	blanker,breakable,
	left=3mm,right=3mm,
	top=3mm,bottom=3mm,
	before skip=15pt,after skip=15pt,
	borderline={0.5pt}{0pt}{black}
}
%----------
\tcolorboxenvironment{practice}{
	blanker,breakable,
	left=3mm,right=3mm,
	top=3mm,bottom=3mm,
	before skip=15pt,after skip=15pt,
	borderline={0.5pt}{0pt}{black}
}
%----------
\tcolorboxenvironment{proof}{
	blanker,breakable,
	left=3mm,right=3mm,
	top=2mm,bottom=2mm,
	before skip=15pt,after skip=20pt,
	% borderline west={1.5pt}{0pt}{black,dotted}
	borderline vertical={1pt}{0pt}{black,dotted}
	% borderline vertical={0.8pt}{0pt}{black,dotted,arrows={Square[scale=0.5]-Square[scale=0.5]}}
	}
%----------
\tcolorboxenvironment{supplement}{
	blanker,breakable,
	left=3mm,right=3mm,
	top=2mm,bottom=2mm,
	before skip=15pt,after skip=20pt,
	% borderline west={1.5pt}{0pt}{black,dotted}
	% borderline vertical={0.5pt}{0pt}{black,arrows = {Circle[scale=0.7]-Circle[scale=0.7]}}
	borderline vertical={0.5pt}{0pt}{black}
	% borderline vertical={0.5pt}{0pt}{black},
	% borderline north={0.5pt}{0pt}{white,arrows={Circle[black,scale=0.7]-Circle[black,scale=0.7]}}
	}
%----------
\tcolorboxenvironment{remark}{
	blanker,breakable,
	left=3mm,right=3mm,
	top=1mm,bottom=1mm,
	before skip=15pt,after skip=20pt,
	% borderline west={1.5pt}{0pt}{black,dotted}
	% borderline vertical={0.5pt}{0pt}{black,arrows = {Circle[scale=0.7]-Circle[scale=0.7]}}
	borderline vertical={0.5pt}{0pt}{black}
	% borderline vertical={0.5pt}{0pt}{black},
	% borderline north={0.5pt}{0pt}{white,arrows={Circle[black,scale=0.7]-Circle[black,scale=0.7]}}
	}
    
%---------------------
 

%----------
%ハイパーリンク
% 「%」は以降の内容を「改行コードも含めて」無視するコマンド
\usepackage[%
%  dvipdfmx,% 欧文ではコメントアウトする
luatex,%
pdfencoding=auto,%
 setpagesize=false,%
 bookmarks=true,%
 bookmarksdepth=tocdepth,%
 bookmarksnumbered=true,%
 colorlinks=false,%
 pdftitle={},%
 pdfsubject={},%
 pdfauthor={},%
 pdfkeywords={}%
]{hyperref}
%------------


%参照 参照するときに自動で環境名を含んで参照する
\usepackage[nameinlink]{cleveref}
\let\normalref\ref
\renewcommand{\ref}{\cref}
\crefname{definition}{定義}{定義}
\crefname{proposition}{命題}{命題}
\crefname{theorem}{定理}{定理}
\crefname{lemma}{補題}{補題}
\crefname{corollary}{系}{系}
\crefname{example}{例}{例}
\crefname{practice}{演習問題}{演習問題}
\crefname{equation}{式}{式} 
\crefname{chapter}{第}{第}
\creflabelformat{chapter}{#2#1章#3}
\crefname{section}{第}{第}
\creflabelformat{section}{#2#1節#3}
\crefname{subsection}{第}{第}
\creflabelformat{subsection}{#2#1小節#3}
%----------

%---------------------
%章跨ぎの参照が不具合を起こすための代わり
% \mylabl でラベル付け
\newcommand{\mylabel}[1]{
\label{#1}
\hypertarget{#1}{}
}
% \myref で環境名付きリンクをつける
\newcommand{\myref}[1]{
\hyperlink{#1}{\cref*{#1}}
}
%-----------------

\usepackage{autonum} %参照した数式にだけ番号を振る
% \usepackage{docmute} %ファイル分割
% \begin{document}

% %\chapter{章のタイトル}
% \section{節のタイトル}
% no text

% \end{document}
%----------

%main.texには以下を書く
%----------
% \documentclass[
% 		book,
% 		head_space=20mm,
% 		foot_space=20mm,
% 		gutter=10mm,
% 		line_length=190mm,
%         openany
% ]{jlreq}
% 
%----------
%LuaLaTeXで実行する!!
%----------
%各章節には以下を書く. 1-03.texのような名前にする
%----------
% \documentclass[
% 		book,
% 		head_space=20mm,
% 		foot_space=20mm,
% 		gutter=10mm,
% 		line_length=190mm
% ]{jlreq}
% \input {preamble.tex}
% \usepackage{docmute} %ファイル分割
% \begin{document}

% %\chapter{章のタイトル}
% \section{節のタイトル}
% no text

% \end{document}
%----------

%main.texには以下を書く
%----------
% \documentclass[
% 		book,
% 		head_space=20mm,
% 		foot_space=20mm,
% 		gutter=10mm,
% 		line_length=190mm,
%         openany
% ]{jlreq}
% \input {preamble.tex}
% \usepackage{docmute} %ファイル分割
% \begin{document}

% %---------- 1章1節
% \input 1-01.tex
% %---------- 1章2節
% \input 1-02.tex
% % ---------- 1章3節
% \input 1-03.tex
% % ---------- 1章4節
% \input 1-04.tex
% % ---------- 1章5節
% \input 1-05.tex
% % ---------- 1章6節
% \input 1-06.tex
% %---------- 1章7節
% \input 1-07.tex
% % ---------- 1章8節
% \input 1-08.tex
% % ---------- 1章9節
% \input 1-09.tex
% % ---------- 1章10節
% \input 1-10.tex
% % ---------- 1章11節
% \input 1-11.tex
% % ---------- 1章12節
% \input 1-12.tex
% % ---------- 参考文献
% \input reference.tex
% \end{document}
% ----------



\usepackage{bxtexlogo}
\usepackage{amsthm}
\usepackage{amsmath}
\usepackage{bbm} %小文字の黒板文字
\usepackage{physics}
\usepackage{amsfonts}
\usepackage{graphicx}
\usepackage{mathtools}
\usepackage{enumitem}
\usepackage[margin=20truemm]{geometry}
\usepackage{textcomp}
\usepackage{bm}
\usepackage{mathrsfs}
\usepackage{latexsym}
\usepackage{amssymb}
\usepackage{algorithmic}
\usepackage{algorithm}
\usepackage{tikz}
\usetikzlibrary{arrows.meta}
\usetikzlibrary{math,matrix,backgrounds}
\usetikzlibrary{angles}
\usetikzlibrary{calc}


%----------
%日本語フォント
% \usepackage[deluxe]{otf} platex用 lualatexでは動かない

%----------
%欧文フォント
\usepackage[T1]{fontenc}

%----------
%文字色
\usepackage{color}

%----------
\setlength{\parindent}{2\zw} %インデントの設定

%----------
% %参照した数式にだけ番号を振る cleverrefと併用するとうまくいかない
% \mathtoolsset{showonlyrefs=true}
%----------

%----------
%集合の中線
\newcommand{\relmiddle}[1]{\mathrel{}\middle#1\mathrel{}}
% \middle| の代わりに \relmiddle| を付ける
\newcommand{\sgn}{\mathop{\mathrm{sgn}}} %置換sgn
\newcommand{\Int}{\mathop{\mathrm{Int}}} %位相空間の内部Int
\newcommand{\Ext}{\mathop{\mathrm{Ext}}} %位相空間の外部Ext
\newcommand{\Cl}{\mathop{\mathrm{Cl}}} %位相空間の閉包Cl
\newcommand{\supp}{\mathop{\mathrm{supp}}} %関数の台supp
\newcommand{\restrict}[2]{\left. #1 \right \vert_{#2}}%関数の制限 \restrict{f}{A} = f|_A
\newcommand{\Ker}{\mathop{\mathrm{Ker}}}
\newcommand{\Coker}{\mathop{\mathrm{Coker}}}
\newcommand{\coker}{\mathop{\mathrm{coker}}}
\newcommand{\Coim}{\mathop{\mathrm{Coim}}}
\newcommand{\coim}{\mathop{\mathrm{coim}}}
\newcommand{\id}{\mathop{\mathrm{id}}}
\newcommand{\Gal}{\mathop{\mathrm{Gal}}}

\newtheorem{definition}{定義}[section]

\usepackage{aliascnt}

% \newaliastheorem{(環境とカウンターの名前)}{(元となるカウンターの名前)}{(表示される文字列)}
\newcommand*{\newaliastheorem}[3]{%
  \newaliascnt{#1}{#2}%
  \newtheorem{#1}[#1]{#3}%
  \aliascntresetthe{#1}%
  \expandafter\newcommand\csname #1autorefname\endcsname{#3}%
}
\newaliastheorem{proposition}{definition}{命題} 
\newaliastheorem{theorem}{definition}{定理}
\newaliastheorem{lemma}{definition}{補題}
\newaliastheorem{corollary}{definition}{系}
\newaliastheorem{example}{definition}{例}
\newaliastheorem{practice}{definition}{演習問題}

\newtheorem*{longproof}{証明}
\newtheorem*{answer}{解答}
\newtheorem*{supplement}{補足}
\newtheorem*{remark}{注意}
%----------

%----------
%古い記法を注意するパッケージ
\RequirePackage[l2tabu, orthodox]{nag}
%----------


% 定理環境(tcolorbox)
\usepackage{tcolorbox} %箱
\tcbuselibrary{breakable,skins,theorems}
\tcolorboxenvironment{definition}{
	blanker,breakable,
	left=3mm,right=3mm,
	top=2mm,bottom=2mm,
	before skip=15pt,after skip=20pt,
	borderline ={0.5pt}{0pt}{black}
}
\newtcolorbox{emptydefinition}{
	blanker,breakable,
	left=3mm,right=3mm,
	top=2mm,bottom=2mm,
	before skip=15pt,after skip=20pt,
	borderline ={0.5pt}{0pt}{black}
}
%----------
\tcolorboxenvironment{proposition}{
	blanker,breakable,
	left=3mm,right=3mm,
	top=3mm,bottom=3mm,
	before skip=15pt,after skip=15pt,
	borderline={0.5pt}{0pt}{black}
}
\newtcolorbox{emptyproposition}{
	blanker,breakable,
	left=3mm,right=3mm,
	top=3mm,bottom=3mm,
	before skip=15pt,after skip=15pt,
	borderline={0.5pt}{0pt}{black}
}
%----------
\tcolorboxenvironment{theorem}{
	blanker,breakable,
	left=3mm,right=3mm,
	top=3mm,bottom=3mm,
    sharp corners,boxrule=0.6pt,
	before skip=15pt,after skip=15pt,
	borderline={0.5pt}{0pt}{black},
    borderline={0.5pt}{1.5pt}{black}
}
\newtcolorbox{emptytheorem}{
	blanker,breakable,
	left=3mm,right=3mm,
	top=3mm,bottom=3mm,
    sharp corners,boxrule=0.6pt,
	before skip=15pt,after skip=15pt,
	borderline={0.5pt}{0pt}{black},
    borderline={0.5pt}{1.5pt}{black}
}
%----------
\tcolorboxenvironment{lemma}{
	blanker,breakable,
	left=3mm,right=3mm,
	top=3mm,bottom=3mm,
	before skip=15pt,after skip=15pt,
	borderline={0.5pt}{0pt}{black}
}
%----------
\tcolorboxenvironment{corollary}{
	blanker,breakable,
	left=3mm,right=3mm,
	top=3mm,bottom=3mm,
	before skip=15pt,after skip=15pt,
	borderline={1.0pt}{0pt}{black,dotted}
}
\newtcolorbox{emptycorollary}{
	blanker,breakable,
	left=3mm,right=3mm,
	top=3mm,bottom=3mm,
	before skip=15pt,after skip=15pt,
	borderline={1.0pt}{0pt}{black,dotted}
}
%----------
\tcolorboxenvironment{example}{
	blanker,breakable,
	left=3mm,right=3mm,
	top=3mm,bottom=3mm,
	before skip=15pt,after skip=15pt,
	borderline={0.5pt}{0pt}{black}
}
%----------
\tcolorboxenvironment{practice}{
	blanker,breakable,
	left=3mm,right=3mm,
	top=3mm,bottom=3mm,
	before skip=15pt,after skip=15pt,
	borderline={0.5pt}{0pt}{black}
}
%----------
\tcolorboxenvironment{proof}{
	blanker,breakable,
	left=3mm,right=3mm,
	top=2mm,bottom=2mm,
	before skip=15pt,after skip=20pt,
	% borderline west={1.5pt}{0pt}{black,dotted}
	borderline vertical={1pt}{0pt}{black,dotted}
	% borderline vertical={0.8pt}{0pt}{black,dotted,arrows={Square[scale=0.5]-Square[scale=0.5]}}
	}
%----------
\tcolorboxenvironment{supplement}{
	blanker,breakable,
	left=3mm,right=3mm,
	top=2mm,bottom=2mm,
	before skip=15pt,after skip=20pt,
	% borderline west={1.5pt}{0pt}{black,dotted}
	% borderline vertical={0.5pt}{0pt}{black,arrows = {Circle[scale=0.7]-Circle[scale=0.7]}}
	borderline vertical={0.5pt}{0pt}{black}
	% borderline vertical={0.5pt}{0pt}{black},
	% borderline north={0.5pt}{0pt}{white,arrows={Circle[black,scale=0.7]-Circle[black,scale=0.7]}}
	}
%----------
\tcolorboxenvironment{remark}{
	blanker,breakable,
	left=3mm,right=3mm,
	top=1mm,bottom=1mm,
	before skip=15pt,after skip=20pt,
	% borderline west={1.5pt}{0pt}{black,dotted}
	% borderline vertical={0.5pt}{0pt}{black,arrows = {Circle[scale=0.7]-Circle[scale=0.7]}}
	borderline vertical={0.5pt}{0pt}{black}
	% borderline vertical={0.5pt}{0pt}{black},
	% borderline north={0.5pt}{0pt}{white,arrows={Circle[black,scale=0.7]-Circle[black,scale=0.7]}}
	}
    
%---------------------
 

%----------
%ハイパーリンク
% 「%」は以降の内容を「改行コードも含めて」無視するコマンド
\usepackage[%
%  dvipdfmx,% 欧文ではコメントアウトする
luatex,%
pdfencoding=auto,%
 setpagesize=false,%
 bookmarks=true,%
 bookmarksdepth=tocdepth,%
 bookmarksnumbered=true,%
 colorlinks=false,%
 pdftitle={},%
 pdfsubject={},%
 pdfauthor={},%
 pdfkeywords={}%
]{hyperref}
%------------


%参照 参照するときに自動で環境名を含んで参照する
\usepackage[nameinlink]{cleveref}
\let\normalref\ref
\renewcommand{\ref}{\cref}
\crefname{definition}{定義}{定義}
\crefname{proposition}{命題}{命題}
\crefname{theorem}{定理}{定理}
\crefname{lemma}{補題}{補題}
\crefname{corollary}{系}{系}
\crefname{example}{例}{例}
\crefname{practice}{演習問題}{演習問題}
\crefname{equation}{式}{式} 
\crefname{chapter}{第}{第}
\creflabelformat{chapter}{#2#1章#3}
\crefname{section}{第}{第}
\creflabelformat{section}{#2#1節#3}
\crefname{subsection}{第}{第}
\creflabelformat{subsection}{#2#1小節#3}
%----------

%---------------------
%章跨ぎの参照が不具合を起こすための代わり
% \mylabl でラベル付け
\newcommand{\mylabel}[1]{
\label{#1}
\hypertarget{#1}{}
}
% \myref で環境名付きリンクをつける
\newcommand{\myref}[1]{
\hyperlink{#1}{\cref*{#1}}
}
%-----------------

\usepackage{autonum} %参照した数式にだけ番号を振る
% \usepackage{docmute} %ファイル分割
% \begin{document}

% %---------- 1章1節
% \input 1-01.tex
% %---------- 1章2節
% \input 1-02.tex
% % ---------- 1章3節
% \input 1-03.tex
% % ---------- 1章4節
% \input 1-04.tex
% % ---------- 1章5節
% \input 1-05.tex
% % ---------- 1章6節
% \input 1-06.tex
% %---------- 1章7節
% \input 1-07.tex
% % ---------- 1章8節
% \input 1-08.tex
% % ---------- 1章9節
% \input 1-09.tex
% % ---------- 1章10節
% \input 1-10.tex
% % ---------- 1章11節
% \input 1-11.tex
% % ---------- 1章12節
% \input 1-12.tex
% % ---------- 参考文献
% \input reference.tex
% \end{document}
% ----------



\usepackage{bxtexlogo}
\usepackage{amsthm}
\usepackage{amsmath}
\usepackage{bbm} %小文字の黒板文字
\usepackage{physics}
\usepackage{amsfonts}
\usepackage{graphicx}
\usepackage{mathtools}
\usepackage{enumitem}
\usepackage[margin=20truemm]{geometry}
\usepackage{textcomp}
\usepackage{bm}
\usepackage{mathrsfs}
\usepackage{latexsym}
\usepackage{amssymb}
\usepackage{algorithmic}
\usepackage{algorithm}
\usepackage{tikz}
\usetikzlibrary{arrows.meta}
\usetikzlibrary{math,matrix,backgrounds}
\usetikzlibrary{angles}
\usetikzlibrary{calc}


%----------
%日本語フォント
% \usepackage[deluxe]{otf} platex用 lualatexでは動かない

%----------
%欧文フォント
\usepackage[T1]{fontenc}

%----------
%文字色
\usepackage{color}

%----------
\setlength{\parindent}{2\zw} %インデントの設定

%----------
% %参照した数式にだけ番号を振る cleverrefと併用するとうまくいかない
% \mathtoolsset{showonlyrefs=true}
%----------

%----------
%集合の中線
\newcommand{\relmiddle}[1]{\mathrel{}\middle#1\mathrel{}}
% \middle| の代わりに \relmiddle| を付ける
\newcommand{\sgn}{\mathop{\mathrm{sgn}}} %置換sgn
\newcommand{\Int}{\mathop{\mathrm{Int}}} %位相空間の内部Int
\newcommand{\Ext}{\mathop{\mathrm{Ext}}} %位相空間の外部Ext
\newcommand{\Cl}{\mathop{\mathrm{Cl}}} %位相空間の閉包Cl
\newcommand{\supp}{\mathop{\mathrm{supp}}} %関数の台supp
\newcommand{\restrict}[2]{\left. #1 \right \vert_{#2}}%関数の制限 \restrict{f}{A} = f|_A
\newcommand{\Ker}{\mathop{\mathrm{Ker}}}
\newcommand{\Coker}{\mathop{\mathrm{Coker}}}
\newcommand{\coker}{\mathop{\mathrm{coker}}}
\newcommand{\Coim}{\mathop{\mathrm{Coim}}}
\newcommand{\coim}{\mathop{\mathrm{coim}}}
\newcommand{\id}{\mathop{\mathrm{id}}}
\newcommand{\Gal}{\mathop{\mathrm{Gal}}}

\newtheorem{definition}{定義}[section]

\usepackage{aliascnt}

% \newaliastheorem{(環境とカウンターの名前)}{(元となるカウンターの名前)}{(表示される文字列)}
\newcommand*{\newaliastheorem}[3]{%
  \newaliascnt{#1}{#2}%
  \newtheorem{#1}[#1]{#3}%
  \aliascntresetthe{#1}%
  \expandafter\newcommand\csname #1autorefname\endcsname{#3}%
}
\newaliastheorem{proposition}{definition}{命題} 
\newaliastheorem{theorem}{definition}{定理}
\newaliastheorem{lemma}{definition}{補題}
\newaliastheorem{corollary}{definition}{系}
\newaliastheorem{example}{definition}{例}
\newaliastheorem{practice}{definition}{演習問題}

\newtheorem*{longproof}{証明}
\newtheorem*{answer}{解答}
\newtheorem*{supplement}{補足}
\newtheorem*{remark}{注意}
%----------

%----------
%古い記法を注意するパッケージ
\RequirePackage[l2tabu, orthodox]{nag}
%----------


% 定理環境(tcolorbox)
\usepackage{tcolorbox} %箱
\tcbuselibrary{breakable,skins,theorems}
\tcolorboxenvironment{definition}{
	blanker,breakable,
	left=3mm,right=3mm,
	top=2mm,bottom=2mm,
	before skip=15pt,after skip=20pt,
	borderline ={0.5pt}{0pt}{black}
}
\newtcolorbox{emptydefinition}{
	blanker,breakable,
	left=3mm,right=3mm,
	top=2mm,bottom=2mm,
	before skip=15pt,after skip=20pt,
	borderline ={0.5pt}{0pt}{black}
}
%----------
\tcolorboxenvironment{proposition}{
	blanker,breakable,
	left=3mm,right=3mm,
	top=3mm,bottom=3mm,
	before skip=15pt,after skip=15pt,
	borderline={0.5pt}{0pt}{black}
}
\newtcolorbox{emptyproposition}{
	blanker,breakable,
	left=3mm,right=3mm,
	top=3mm,bottom=3mm,
	before skip=15pt,after skip=15pt,
	borderline={0.5pt}{0pt}{black}
}
%----------
\tcolorboxenvironment{theorem}{
	blanker,breakable,
	left=3mm,right=3mm,
	top=3mm,bottom=3mm,
    sharp corners,boxrule=0.6pt,
	before skip=15pt,after skip=15pt,
	borderline={0.5pt}{0pt}{black},
    borderline={0.5pt}{1.5pt}{black}
}
\newtcolorbox{emptytheorem}{
	blanker,breakable,
	left=3mm,right=3mm,
	top=3mm,bottom=3mm,
    sharp corners,boxrule=0.6pt,
	before skip=15pt,after skip=15pt,
	borderline={0.5pt}{0pt}{black},
    borderline={0.5pt}{1.5pt}{black}
}
%----------
\tcolorboxenvironment{lemma}{
	blanker,breakable,
	left=3mm,right=3mm,
	top=3mm,bottom=3mm,
	before skip=15pt,after skip=15pt,
	borderline={0.5pt}{0pt}{black}
}
%----------
\tcolorboxenvironment{corollary}{
	blanker,breakable,
	left=3mm,right=3mm,
	top=3mm,bottom=3mm,
	before skip=15pt,after skip=15pt,
	borderline={1.0pt}{0pt}{black,dotted}
}
\newtcolorbox{emptycorollary}{
	blanker,breakable,
	left=3mm,right=3mm,
	top=3mm,bottom=3mm,
	before skip=15pt,after skip=15pt,
	borderline={1.0pt}{0pt}{black,dotted}
}
%----------
\tcolorboxenvironment{example}{
	blanker,breakable,
	left=3mm,right=3mm,
	top=3mm,bottom=3mm,
	before skip=15pt,after skip=15pt,
	borderline={0.5pt}{0pt}{black}
}
%----------
\tcolorboxenvironment{practice}{
	blanker,breakable,
	left=3mm,right=3mm,
	top=3mm,bottom=3mm,
	before skip=15pt,after skip=15pt,
	borderline={0.5pt}{0pt}{black}
}
%----------
\tcolorboxenvironment{proof}{
	blanker,breakable,
	left=3mm,right=3mm,
	top=2mm,bottom=2mm,
	before skip=15pt,after skip=20pt,
	% borderline west={1.5pt}{0pt}{black,dotted}
	borderline vertical={1pt}{0pt}{black,dotted}
	% borderline vertical={0.8pt}{0pt}{black,dotted,arrows={Square[scale=0.5]-Square[scale=0.5]}}
	}
%----------
\tcolorboxenvironment{supplement}{
	blanker,breakable,
	left=3mm,right=3mm,
	top=2mm,bottom=2mm,
	before skip=15pt,after skip=20pt,
	% borderline west={1.5pt}{0pt}{black,dotted}
	% borderline vertical={0.5pt}{0pt}{black,arrows = {Circle[scale=0.7]-Circle[scale=0.7]}}
	borderline vertical={0.5pt}{0pt}{black}
	% borderline vertical={0.5pt}{0pt}{black},
	% borderline north={0.5pt}{0pt}{white,arrows={Circle[black,scale=0.7]-Circle[black,scale=0.7]}}
	}
%----------
\tcolorboxenvironment{remark}{
	blanker,breakable,
	left=3mm,right=3mm,
	top=1mm,bottom=1mm,
	before skip=15pt,after skip=20pt,
	% borderline west={1.5pt}{0pt}{black,dotted}
	% borderline vertical={0.5pt}{0pt}{black,arrows = {Circle[scale=0.7]-Circle[scale=0.7]}}
	borderline vertical={0.5pt}{0pt}{black}
	% borderline vertical={0.5pt}{0pt}{black},
	% borderline north={0.5pt}{0pt}{white,arrows={Circle[black,scale=0.7]-Circle[black,scale=0.7]}}
	}
    
%---------------------
 

%----------
%ハイパーリンク
% 「%」は以降の内容を「改行コードも含めて」無視するコマンド
\usepackage[%
%  dvipdfmx,% 欧文ではコメントアウトする
luatex,%
pdfencoding=auto,%
 setpagesize=false,%
 bookmarks=true,%
 bookmarksdepth=tocdepth,%
 bookmarksnumbered=true,%
 colorlinks=false,%
 pdftitle={},%
 pdfsubject={},%
 pdfauthor={},%
 pdfkeywords={}%
]{hyperref}
%------------


%参照 参照するときに自動で環境名を含んで参照する
\usepackage[nameinlink]{cleveref}
\let\normalref\ref
\renewcommand{\ref}{\cref}
\crefname{definition}{定義}{定義}
\crefname{proposition}{命題}{命題}
\crefname{theorem}{定理}{定理}
\crefname{lemma}{補題}{補題}
\crefname{corollary}{系}{系}
\crefname{example}{例}{例}
\crefname{practice}{演習問題}{演習問題}
\crefname{equation}{式}{式} 
\crefname{chapter}{第}{第}
\creflabelformat{chapter}{#2#1章#3}
\crefname{section}{第}{第}
\creflabelformat{section}{#2#1節#3}
\crefname{subsection}{第}{第}
\creflabelformat{subsection}{#2#1小節#3}
%----------

%---------------------
%章跨ぎの参照が不具合を起こすための代わり
% \mylabl でラベル付け
\newcommand{\mylabel}[1]{
\label{#1}
\hypertarget{#1}{}
}
% \myref で環境名付きリンクをつける
\newcommand{\myref}[1]{
\hyperlink{#1}{\cref*{#1}}
}
%-----------------

\usepackage{autonum} %参照した数式にだけ番号を振る
% \usepackage{docmute} %ファイル分割
% \begin{document}

% %\chapter{章のタイトル}
% \section{節のタイトル}
% no text

% \end{document}
%----------

%main.texには以下を書く
%----------
% \documentclass[
% 		book,
% 		head_space=20mm,
% 		foot_space=20mm,
% 		gutter=10mm,
% 		line_length=190mm,
%         openany
% ]{jlreq}
% 
%----------
%LuaLaTeXで実行する!!
%----------
%各章節には以下を書く. 1-03.texのような名前にする
%----------
% \documentclass[
% 		book,
% 		head_space=20mm,
% 		foot_space=20mm,
% 		gutter=10mm,
% 		line_length=190mm
% ]{jlreq}
% 
%----------
%LuaLaTeXで実行する!!
%----------
%各章節には以下を書く. 1-03.texのような名前にする
%----------
% \documentclass[
% 		book,
% 		head_space=20mm,
% 		foot_space=20mm,
% 		gutter=10mm,
% 		line_length=190mm
% ]{jlreq}
% \input {preamble.tex}
% \usepackage{docmute} %ファイル分割
% \begin{document}

% %\chapter{章のタイトル}
% \section{節のタイトル}
% no text

% \end{document}
%----------

%main.texには以下を書く
%----------
% \documentclass[
% 		book,
% 		head_space=20mm,
% 		foot_space=20mm,
% 		gutter=10mm,
% 		line_length=190mm,
%         openany
% ]{jlreq}
% \input {preamble.tex}
% \usepackage{docmute} %ファイル分割
% \begin{document}

% %---------- 1章1節
% \input 1-01.tex
% %---------- 1章2節
% \input 1-02.tex
% % ---------- 1章3節
% \input 1-03.tex
% % ---------- 1章4節
% \input 1-04.tex
% % ---------- 1章5節
% \input 1-05.tex
% % ---------- 1章6節
% \input 1-06.tex
% %---------- 1章7節
% \input 1-07.tex
% % ---------- 1章8節
% \input 1-08.tex
% % ---------- 1章9節
% \input 1-09.tex
% % ---------- 1章10節
% \input 1-10.tex
% % ---------- 1章11節
% \input 1-11.tex
% % ---------- 1章12節
% \input 1-12.tex
% % ---------- 参考文献
% \input reference.tex
% \end{document}
% ----------



\usepackage{bxtexlogo}
\usepackage{amsthm}
\usepackage{amsmath}
\usepackage{bbm} %小文字の黒板文字
\usepackage{physics}
\usepackage{amsfonts}
\usepackage{graphicx}
\usepackage{mathtools}
\usepackage{enumitem}
\usepackage[margin=20truemm]{geometry}
\usepackage{textcomp}
\usepackage{bm}
\usepackage{mathrsfs}
\usepackage{latexsym}
\usepackage{amssymb}
\usepackage{algorithmic}
\usepackage{algorithm}
\usepackage{tikz}
\usetikzlibrary{arrows.meta}
\usetikzlibrary{math,matrix,backgrounds}
\usetikzlibrary{angles}
\usetikzlibrary{calc}


%----------
%日本語フォント
% \usepackage[deluxe]{otf} platex用 lualatexでは動かない

%----------
%欧文フォント
\usepackage[T1]{fontenc}

%----------
%文字色
\usepackage{color}

%----------
\setlength{\parindent}{2\zw} %インデントの設定

%----------
% %参照した数式にだけ番号を振る cleverrefと併用するとうまくいかない
% \mathtoolsset{showonlyrefs=true}
%----------

%----------
%集合の中線
\newcommand{\relmiddle}[1]{\mathrel{}\middle#1\mathrel{}}
% \middle| の代わりに \relmiddle| を付ける
\newcommand{\sgn}{\mathop{\mathrm{sgn}}} %置換sgn
\newcommand{\Int}{\mathop{\mathrm{Int}}} %位相空間の内部Int
\newcommand{\Ext}{\mathop{\mathrm{Ext}}} %位相空間の外部Ext
\newcommand{\Cl}{\mathop{\mathrm{Cl}}} %位相空間の閉包Cl
\newcommand{\supp}{\mathop{\mathrm{supp}}} %関数の台supp
\newcommand{\restrict}[2]{\left. #1 \right \vert_{#2}}%関数の制限 \restrict{f}{A} = f|_A
\newcommand{\Ker}{\mathop{\mathrm{Ker}}}
\newcommand{\Coker}{\mathop{\mathrm{Coker}}}
\newcommand{\coker}{\mathop{\mathrm{coker}}}
\newcommand{\Coim}{\mathop{\mathrm{Coim}}}
\newcommand{\coim}{\mathop{\mathrm{coim}}}
\newcommand{\id}{\mathop{\mathrm{id}}}
\newcommand{\Gal}{\mathop{\mathrm{Gal}}}

\newtheorem{definition}{定義}[section]

\usepackage{aliascnt}

% \newaliastheorem{(環境とカウンターの名前)}{(元となるカウンターの名前)}{(表示される文字列)}
\newcommand*{\newaliastheorem}[3]{%
  \newaliascnt{#1}{#2}%
  \newtheorem{#1}[#1]{#3}%
  \aliascntresetthe{#1}%
  \expandafter\newcommand\csname #1autorefname\endcsname{#3}%
}
\newaliastheorem{proposition}{definition}{命題} 
\newaliastheorem{theorem}{definition}{定理}
\newaliastheorem{lemma}{definition}{補題}
\newaliastheorem{corollary}{definition}{系}
\newaliastheorem{example}{definition}{例}
\newaliastheorem{practice}{definition}{演習問題}

\newtheorem*{longproof}{証明}
\newtheorem*{answer}{解答}
\newtheorem*{supplement}{補足}
\newtheorem*{remark}{注意}
%----------

%----------
%古い記法を注意するパッケージ
\RequirePackage[l2tabu, orthodox]{nag}
%----------


% 定理環境(tcolorbox)
\usepackage{tcolorbox} %箱
\tcbuselibrary{breakable,skins,theorems}
\tcolorboxenvironment{definition}{
	blanker,breakable,
	left=3mm,right=3mm,
	top=2mm,bottom=2mm,
	before skip=15pt,after skip=20pt,
	borderline ={0.5pt}{0pt}{black}
}
\newtcolorbox{emptydefinition}{
	blanker,breakable,
	left=3mm,right=3mm,
	top=2mm,bottom=2mm,
	before skip=15pt,after skip=20pt,
	borderline ={0.5pt}{0pt}{black}
}
%----------
\tcolorboxenvironment{proposition}{
	blanker,breakable,
	left=3mm,right=3mm,
	top=3mm,bottom=3mm,
	before skip=15pt,after skip=15pt,
	borderline={0.5pt}{0pt}{black}
}
\newtcolorbox{emptyproposition}{
	blanker,breakable,
	left=3mm,right=3mm,
	top=3mm,bottom=3mm,
	before skip=15pt,after skip=15pt,
	borderline={0.5pt}{0pt}{black}
}
%----------
\tcolorboxenvironment{theorem}{
	blanker,breakable,
	left=3mm,right=3mm,
	top=3mm,bottom=3mm,
    sharp corners,boxrule=0.6pt,
	before skip=15pt,after skip=15pt,
	borderline={0.5pt}{0pt}{black},
    borderline={0.5pt}{1.5pt}{black}
}
\newtcolorbox{emptytheorem}{
	blanker,breakable,
	left=3mm,right=3mm,
	top=3mm,bottom=3mm,
    sharp corners,boxrule=0.6pt,
	before skip=15pt,after skip=15pt,
	borderline={0.5pt}{0pt}{black},
    borderline={0.5pt}{1.5pt}{black}
}
%----------
\tcolorboxenvironment{lemma}{
	blanker,breakable,
	left=3mm,right=3mm,
	top=3mm,bottom=3mm,
	before skip=15pt,after skip=15pt,
	borderline={0.5pt}{0pt}{black}
}
%----------
\tcolorboxenvironment{corollary}{
	blanker,breakable,
	left=3mm,right=3mm,
	top=3mm,bottom=3mm,
	before skip=15pt,after skip=15pt,
	borderline={1.0pt}{0pt}{black,dotted}
}
\newtcolorbox{emptycorollary}{
	blanker,breakable,
	left=3mm,right=3mm,
	top=3mm,bottom=3mm,
	before skip=15pt,after skip=15pt,
	borderline={1.0pt}{0pt}{black,dotted}
}
%----------
\tcolorboxenvironment{example}{
	blanker,breakable,
	left=3mm,right=3mm,
	top=3mm,bottom=3mm,
	before skip=15pt,after skip=15pt,
	borderline={0.5pt}{0pt}{black}
}
%----------
\tcolorboxenvironment{practice}{
	blanker,breakable,
	left=3mm,right=3mm,
	top=3mm,bottom=3mm,
	before skip=15pt,after skip=15pt,
	borderline={0.5pt}{0pt}{black}
}
%----------
\tcolorboxenvironment{proof}{
	blanker,breakable,
	left=3mm,right=3mm,
	top=2mm,bottom=2mm,
	before skip=15pt,after skip=20pt,
	% borderline west={1.5pt}{0pt}{black,dotted}
	borderline vertical={1pt}{0pt}{black,dotted}
	% borderline vertical={0.8pt}{0pt}{black,dotted,arrows={Square[scale=0.5]-Square[scale=0.5]}}
	}
%----------
\tcolorboxenvironment{supplement}{
	blanker,breakable,
	left=3mm,right=3mm,
	top=2mm,bottom=2mm,
	before skip=15pt,after skip=20pt,
	% borderline west={1.5pt}{0pt}{black,dotted}
	% borderline vertical={0.5pt}{0pt}{black,arrows = {Circle[scale=0.7]-Circle[scale=0.7]}}
	borderline vertical={0.5pt}{0pt}{black}
	% borderline vertical={0.5pt}{0pt}{black},
	% borderline north={0.5pt}{0pt}{white,arrows={Circle[black,scale=0.7]-Circle[black,scale=0.7]}}
	}
%----------
\tcolorboxenvironment{remark}{
	blanker,breakable,
	left=3mm,right=3mm,
	top=1mm,bottom=1mm,
	before skip=15pt,after skip=20pt,
	% borderline west={1.5pt}{0pt}{black,dotted}
	% borderline vertical={0.5pt}{0pt}{black,arrows = {Circle[scale=0.7]-Circle[scale=0.7]}}
	borderline vertical={0.5pt}{0pt}{black}
	% borderline vertical={0.5pt}{0pt}{black},
	% borderline north={0.5pt}{0pt}{white,arrows={Circle[black,scale=0.7]-Circle[black,scale=0.7]}}
	}
    
%---------------------
 

%----------
%ハイパーリンク
% 「%」は以降の内容を「改行コードも含めて」無視するコマンド
\usepackage[%
%  dvipdfmx,% 欧文ではコメントアウトする
luatex,%
pdfencoding=auto,%
 setpagesize=false,%
 bookmarks=true,%
 bookmarksdepth=tocdepth,%
 bookmarksnumbered=true,%
 colorlinks=false,%
 pdftitle={},%
 pdfsubject={},%
 pdfauthor={},%
 pdfkeywords={}%
]{hyperref}
%------------


%参照 参照するときに自動で環境名を含んで参照する
\usepackage[nameinlink]{cleveref}
\let\normalref\ref
\renewcommand{\ref}{\cref}
\crefname{definition}{定義}{定義}
\crefname{proposition}{命題}{命題}
\crefname{theorem}{定理}{定理}
\crefname{lemma}{補題}{補題}
\crefname{corollary}{系}{系}
\crefname{example}{例}{例}
\crefname{practice}{演習問題}{演習問題}
\crefname{equation}{式}{式} 
\crefname{chapter}{第}{第}
\creflabelformat{chapter}{#2#1章#3}
\crefname{section}{第}{第}
\creflabelformat{section}{#2#1節#3}
\crefname{subsection}{第}{第}
\creflabelformat{subsection}{#2#1小節#3}
%----------

%---------------------
%章跨ぎの参照が不具合を起こすための代わり
% \mylabl でラベル付け
\newcommand{\mylabel}[1]{
\label{#1}
\hypertarget{#1}{}
}
% \myref で環境名付きリンクをつける
\newcommand{\myref}[1]{
\hyperlink{#1}{\cref*{#1}}
}
%-----------------

\usepackage{autonum} %参照した数式にだけ番号を振る
% \usepackage{docmute} %ファイル分割
% \begin{document}

% %\chapter{章のタイトル}
% \section{節のタイトル}
% no text

% \end{document}
%----------

%main.texには以下を書く
%----------
% \documentclass[
% 		book,
% 		head_space=20mm,
% 		foot_space=20mm,
% 		gutter=10mm,
% 		line_length=190mm,
%         openany
% ]{jlreq}
% 
%----------
%LuaLaTeXで実行する!!
%----------
%各章節には以下を書く. 1-03.texのような名前にする
%----------
% \documentclass[
% 		book,
% 		head_space=20mm,
% 		foot_space=20mm,
% 		gutter=10mm,
% 		line_length=190mm
% ]{jlreq}
% \input {preamble.tex}
% \usepackage{docmute} %ファイル分割
% \begin{document}

% %\chapter{章のタイトル}
% \section{節のタイトル}
% no text

% \end{document}
%----------

%main.texには以下を書く
%----------
% \documentclass[
% 		book,
% 		head_space=20mm,
% 		foot_space=20mm,
% 		gutter=10mm,
% 		line_length=190mm,
%         openany
% ]{jlreq}
% \input {preamble.tex}
% \usepackage{docmute} %ファイル分割
% \begin{document}

% %---------- 1章1節
% \input 1-01.tex
% %---------- 1章2節
% \input 1-02.tex
% % ---------- 1章3節
% \input 1-03.tex
% % ---------- 1章4節
% \input 1-04.tex
% % ---------- 1章5節
% \input 1-05.tex
% % ---------- 1章6節
% \input 1-06.tex
% %---------- 1章7節
% \input 1-07.tex
% % ---------- 1章8節
% \input 1-08.tex
% % ---------- 1章9節
% \input 1-09.tex
% % ---------- 1章10節
% \input 1-10.tex
% % ---------- 1章11節
% \input 1-11.tex
% % ---------- 1章12節
% \input 1-12.tex
% % ---------- 参考文献
% \input reference.tex
% \end{document}
% ----------



\usepackage{bxtexlogo}
\usepackage{amsthm}
\usepackage{amsmath}
\usepackage{bbm} %小文字の黒板文字
\usepackage{physics}
\usepackage{amsfonts}
\usepackage{graphicx}
\usepackage{mathtools}
\usepackage{enumitem}
\usepackage[margin=20truemm]{geometry}
\usepackage{textcomp}
\usepackage{bm}
\usepackage{mathrsfs}
\usepackage{latexsym}
\usepackage{amssymb}
\usepackage{algorithmic}
\usepackage{algorithm}
\usepackage{tikz}
\usetikzlibrary{arrows.meta}
\usetikzlibrary{math,matrix,backgrounds}
\usetikzlibrary{angles}
\usetikzlibrary{calc}


%----------
%日本語フォント
% \usepackage[deluxe]{otf} platex用 lualatexでは動かない

%----------
%欧文フォント
\usepackage[T1]{fontenc}

%----------
%文字色
\usepackage{color}

%----------
\setlength{\parindent}{2\zw} %インデントの設定

%----------
% %参照した数式にだけ番号を振る cleverrefと併用するとうまくいかない
% \mathtoolsset{showonlyrefs=true}
%----------

%----------
%集合の中線
\newcommand{\relmiddle}[1]{\mathrel{}\middle#1\mathrel{}}
% \middle| の代わりに \relmiddle| を付ける
\newcommand{\sgn}{\mathop{\mathrm{sgn}}} %置換sgn
\newcommand{\Int}{\mathop{\mathrm{Int}}} %位相空間の内部Int
\newcommand{\Ext}{\mathop{\mathrm{Ext}}} %位相空間の外部Ext
\newcommand{\Cl}{\mathop{\mathrm{Cl}}} %位相空間の閉包Cl
\newcommand{\supp}{\mathop{\mathrm{supp}}} %関数の台supp
\newcommand{\restrict}[2]{\left. #1 \right \vert_{#2}}%関数の制限 \restrict{f}{A} = f|_A
\newcommand{\Ker}{\mathop{\mathrm{Ker}}}
\newcommand{\Coker}{\mathop{\mathrm{Coker}}}
\newcommand{\coker}{\mathop{\mathrm{coker}}}
\newcommand{\Coim}{\mathop{\mathrm{Coim}}}
\newcommand{\coim}{\mathop{\mathrm{coim}}}
\newcommand{\id}{\mathop{\mathrm{id}}}
\newcommand{\Gal}{\mathop{\mathrm{Gal}}}

\newtheorem{definition}{定義}[section]

\usepackage{aliascnt}

% \newaliastheorem{(環境とカウンターの名前)}{(元となるカウンターの名前)}{(表示される文字列)}
\newcommand*{\newaliastheorem}[3]{%
  \newaliascnt{#1}{#2}%
  \newtheorem{#1}[#1]{#3}%
  \aliascntresetthe{#1}%
  \expandafter\newcommand\csname #1autorefname\endcsname{#3}%
}
\newaliastheorem{proposition}{definition}{命題} 
\newaliastheorem{theorem}{definition}{定理}
\newaliastheorem{lemma}{definition}{補題}
\newaliastheorem{corollary}{definition}{系}
\newaliastheorem{example}{definition}{例}
\newaliastheorem{practice}{definition}{演習問題}

\newtheorem*{longproof}{証明}
\newtheorem*{answer}{解答}
\newtheorem*{supplement}{補足}
\newtheorem*{remark}{注意}
%----------

%----------
%古い記法を注意するパッケージ
\RequirePackage[l2tabu, orthodox]{nag}
%----------


% 定理環境(tcolorbox)
\usepackage{tcolorbox} %箱
\tcbuselibrary{breakable,skins,theorems}
\tcolorboxenvironment{definition}{
	blanker,breakable,
	left=3mm,right=3mm,
	top=2mm,bottom=2mm,
	before skip=15pt,after skip=20pt,
	borderline ={0.5pt}{0pt}{black}
}
\newtcolorbox{emptydefinition}{
	blanker,breakable,
	left=3mm,right=3mm,
	top=2mm,bottom=2mm,
	before skip=15pt,after skip=20pt,
	borderline ={0.5pt}{0pt}{black}
}
%----------
\tcolorboxenvironment{proposition}{
	blanker,breakable,
	left=3mm,right=3mm,
	top=3mm,bottom=3mm,
	before skip=15pt,after skip=15pt,
	borderline={0.5pt}{0pt}{black}
}
\newtcolorbox{emptyproposition}{
	blanker,breakable,
	left=3mm,right=3mm,
	top=3mm,bottom=3mm,
	before skip=15pt,after skip=15pt,
	borderline={0.5pt}{0pt}{black}
}
%----------
\tcolorboxenvironment{theorem}{
	blanker,breakable,
	left=3mm,right=3mm,
	top=3mm,bottom=3mm,
    sharp corners,boxrule=0.6pt,
	before skip=15pt,after skip=15pt,
	borderline={0.5pt}{0pt}{black},
    borderline={0.5pt}{1.5pt}{black}
}
\newtcolorbox{emptytheorem}{
	blanker,breakable,
	left=3mm,right=3mm,
	top=3mm,bottom=3mm,
    sharp corners,boxrule=0.6pt,
	before skip=15pt,after skip=15pt,
	borderline={0.5pt}{0pt}{black},
    borderline={0.5pt}{1.5pt}{black}
}
%----------
\tcolorboxenvironment{lemma}{
	blanker,breakable,
	left=3mm,right=3mm,
	top=3mm,bottom=3mm,
	before skip=15pt,after skip=15pt,
	borderline={0.5pt}{0pt}{black}
}
%----------
\tcolorboxenvironment{corollary}{
	blanker,breakable,
	left=3mm,right=3mm,
	top=3mm,bottom=3mm,
	before skip=15pt,after skip=15pt,
	borderline={1.0pt}{0pt}{black,dotted}
}
\newtcolorbox{emptycorollary}{
	blanker,breakable,
	left=3mm,right=3mm,
	top=3mm,bottom=3mm,
	before skip=15pt,after skip=15pt,
	borderline={1.0pt}{0pt}{black,dotted}
}
%----------
\tcolorboxenvironment{example}{
	blanker,breakable,
	left=3mm,right=3mm,
	top=3mm,bottom=3mm,
	before skip=15pt,after skip=15pt,
	borderline={0.5pt}{0pt}{black}
}
%----------
\tcolorboxenvironment{practice}{
	blanker,breakable,
	left=3mm,right=3mm,
	top=3mm,bottom=3mm,
	before skip=15pt,after skip=15pt,
	borderline={0.5pt}{0pt}{black}
}
%----------
\tcolorboxenvironment{proof}{
	blanker,breakable,
	left=3mm,right=3mm,
	top=2mm,bottom=2mm,
	before skip=15pt,after skip=20pt,
	% borderline west={1.5pt}{0pt}{black,dotted}
	borderline vertical={1pt}{0pt}{black,dotted}
	% borderline vertical={0.8pt}{0pt}{black,dotted,arrows={Square[scale=0.5]-Square[scale=0.5]}}
	}
%----------
\tcolorboxenvironment{supplement}{
	blanker,breakable,
	left=3mm,right=3mm,
	top=2mm,bottom=2mm,
	before skip=15pt,after skip=20pt,
	% borderline west={1.5pt}{0pt}{black,dotted}
	% borderline vertical={0.5pt}{0pt}{black,arrows = {Circle[scale=0.7]-Circle[scale=0.7]}}
	borderline vertical={0.5pt}{0pt}{black}
	% borderline vertical={0.5pt}{0pt}{black},
	% borderline north={0.5pt}{0pt}{white,arrows={Circle[black,scale=0.7]-Circle[black,scale=0.7]}}
	}
%----------
\tcolorboxenvironment{remark}{
	blanker,breakable,
	left=3mm,right=3mm,
	top=1mm,bottom=1mm,
	before skip=15pt,after skip=20pt,
	% borderline west={1.5pt}{0pt}{black,dotted}
	% borderline vertical={0.5pt}{0pt}{black,arrows = {Circle[scale=0.7]-Circle[scale=0.7]}}
	borderline vertical={0.5pt}{0pt}{black}
	% borderline vertical={0.5pt}{0pt}{black},
	% borderline north={0.5pt}{0pt}{white,arrows={Circle[black,scale=0.7]-Circle[black,scale=0.7]}}
	}
    
%---------------------
 

%----------
%ハイパーリンク
% 「%」は以降の内容を「改行コードも含めて」無視するコマンド
\usepackage[%
%  dvipdfmx,% 欧文ではコメントアウトする
luatex,%
pdfencoding=auto,%
 setpagesize=false,%
 bookmarks=true,%
 bookmarksdepth=tocdepth,%
 bookmarksnumbered=true,%
 colorlinks=false,%
 pdftitle={},%
 pdfsubject={},%
 pdfauthor={},%
 pdfkeywords={}%
]{hyperref}
%------------


%参照 参照するときに自動で環境名を含んで参照する
\usepackage[nameinlink]{cleveref}
\let\normalref\ref
\renewcommand{\ref}{\cref}
\crefname{definition}{定義}{定義}
\crefname{proposition}{命題}{命題}
\crefname{theorem}{定理}{定理}
\crefname{lemma}{補題}{補題}
\crefname{corollary}{系}{系}
\crefname{example}{例}{例}
\crefname{practice}{演習問題}{演習問題}
\crefname{equation}{式}{式} 
\crefname{chapter}{第}{第}
\creflabelformat{chapter}{#2#1章#3}
\crefname{section}{第}{第}
\creflabelformat{section}{#2#1節#3}
\crefname{subsection}{第}{第}
\creflabelformat{subsection}{#2#1小節#3}
%----------

%---------------------
%章跨ぎの参照が不具合を起こすための代わり
% \mylabl でラベル付け
\newcommand{\mylabel}[1]{
\label{#1}
\hypertarget{#1}{}
}
% \myref で環境名付きリンクをつける
\newcommand{\myref}[1]{
\hyperlink{#1}{\cref*{#1}}
}
%-----------------

\usepackage{autonum} %参照した数式にだけ番号を振る
% \usepackage{docmute} %ファイル分割
% \begin{document}

% %---------- 1章1節
% \input 1-01.tex
% %---------- 1章2節
% \input 1-02.tex
% % ---------- 1章3節
% \input 1-03.tex
% % ---------- 1章4節
% \input 1-04.tex
% % ---------- 1章5節
% \input 1-05.tex
% % ---------- 1章6節
% \input 1-06.tex
% %---------- 1章7節
% \input 1-07.tex
% % ---------- 1章8節
% \input 1-08.tex
% % ---------- 1章9節
% \input 1-09.tex
% % ---------- 1章10節
% \input 1-10.tex
% % ---------- 1章11節
% \input 1-11.tex
% % ---------- 1章12節
% \input 1-12.tex
% % ---------- 参考文献
% \input reference.tex
% \end{document}
% ----------



\usepackage{bxtexlogo}
\usepackage{amsthm}
\usepackage{amsmath}
\usepackage{bbm} %小文字の黒板文字
\usepackage{physics}
\usepackage{amsfonts}
\usepackage{graphicx}
\usepackage{mathtools}
\usepackage{enumitem}
\usepackage[margin=20truemm]{geometry}
\usepackage{textcomp}
\usepackage{bm}
\usepackage{mathrsfs}
\usepackage{latexsym}
\usepackage{amssymb}
\usepackage{algorithmic}
\usepackage{algorithm}
\usepackage{tikz}
\usetikzlibrary{arrows.meta}
\usetikzlibrary{math,matrix,backgrounds}
\usetikzlibrary{angles}
\usetikzlibrary{calc}


%----------
%日本語フォント
% \usepackage[deluxe]{otf} platex用 lualatexでは動かない

%----------
%欧文フォント
\usepackage[T1]{fontenc}

%----------
%文字色
\usepackage{color}

%----------
\setlength{\parindent}{2\zw} %インデントの設定

%----------
% %参照した数式にだけ番号を振る cleverrefと併用するとうまくいかない
% \mathtoolsset{showonlyrefs=true}
%----------

%----------
%集合の中線
\newcommand{\relmiddle}[1]{\mathrel{}\middle#1\mathrel{}}
% \middle| の代わりに \relmiddle| を付ける
\newcommand{\sgn}{\mathop{\mathrm{sgn}}} %置換sgn
\newcommand{\Int}{\mathop{\mathrm{Int}}} %位相空間の内部Int
\newcommand{\Ext}{\mathop{\mathrm{Ext}}} %位相空間の外部Ext
\newcommand{\Cl}{\mathop{\mathrm{Cl}}} %位相空間の閉包Cl
\newcommand{\supp}{\mathop{\mathrm{supp}}} %関数の台supp
\newcommand{\restrict}[2]{\left. #1 \right \vert_{#2}}%関数の制限 \restrict{f}{A} = f|_A
\newcommand{\Ker}{\mathop{\mathrm{Ker}}}
\newcommand{\Coker}{\mathop{\mathrm{Coker}}}
\newcommand{\coker}{\mathop{\mathrm{coker}}}
\newcommand{\Coim}{\mathop{\mathrm{Coim}}}
\newcommand{\coim}{\mathop{\mathrm{coim}}}
\newcommand{\id}{\mathop{\mathrm{id}}}
\newcommand{\Gal}{\mathop{\mathrm{Gal}}}

\newtheorem{definition}{定義}[section]

\usepackage{aliascnt}

% \newaliastheorem{(環境とカウンターの名前)}{(元となるカウンターの名前)}{(表示される文字列)}
\newcommand*{\newaliastheorem}[3]{%
  \newaliascnt{#1}{#2}%
  \newtheorem{#1}[#1]{#3}%
  \aliascntresetthe{#1}%
  \expandafter\newcommand\csname #1autorefname\endcsname{#3}%
}
\newaliastheorem{proposition}{definition}{命題} 
\newaliastheorem{theorem}{definition}{定理}
\newaliastheorem{lemma}{definition}{補題}
\newaliastheorem{corollary}{definition}{系}
\newaliastheorem{example}{definition}{例}
\newaliastheorem{practice}{definition}{演習問題}

\newtheorem*{longproof}{証明}
\newtheorem*{answer}{解答}
\newtheorem*{supplement}{補足}
\newtheorem*{remark}{注意}
%----------

%----------
%古い記法を注意するパッケージ
\RequirePackage[l2tabu, orthodox]{nag}
%----------


% 定理環境(tcolorbox)
\usepackage{tcolorbox} %箱
\tcbuselibrary{breakable,skins,theorems}
\tcolorboxenvironment{definition}{
	blanker,breakable,
	left=3mm,right=3mm,
	top=2mm,bottom=2mm,
	before skip=15pt,after skip=20pt,
	borderline ={0.5pt}{0pt}{black}
}
\newtcolorbox{emptydefinition}{
	blanker,breakable,
	left=3mm,right=3mm,
	top=2mm,bottom=2mm,
	before skip=15pt,after skip=20pt,
	borderline ={0.5pt}{0pt}{black}
}
%----------
\tcolorboxenvironment{proposition}{
	blanker,breakable,
	left=3mm,right=3mm,
	top=3mm,bottom=3mm,
	before skip=15pt,after skip=15pt,
	borderline={0.5pt}{0pt}{black}
}
\newtcolorbox{emptyproposition}{
	blanker,breakable,
	left=3mm,right=3mm,
	top=3mm,bottom=3mm,
	before skip=15pt,after skip=15pt,
	borderline={0.5pt}{0pt}{black}
}
%----------
\tcolorboxenvironment{theorem}{
	blanker,breakable,
	left=3mm,right=3mm,
	top=3mm,bottom=3mm,
    sharp corners,boxrule=0.6pt,
	before skip=15pt,after skip=15pt,
	borderline={0.5pt}{0pt}{black},
    borderline={0.5pt}{1.5pt}{black}
}
\newtcolorbox{emptytheorem}{
	blanker,breakable,
	left=3mm,right=3mm,
	top=3mm,bottom=3mm,
    sharp corners,boxrule=0.6pt,
	before skip=15pt,after skip=15pt,
	borderline={0.5pt}{0pt}{black},
    borderline={0.5pt}{1.5pt}{black}
}
%----------
\tcolorboxenvironment{lemma}{
	blanker,breakable,
	left=3mm,right=3mm,
	top=3mm,bottom=3mm,
	before skip=15pt,after skip=15pt,
	borderline={0.5pt}{0pt}{black}
}
%----------
\tcolorboxenvironment{corollary}{
	blanker,breakable,
	left=3mm,right=3mm,
	top=3mm,bottom=3mm,
	before skip=15pt,after skip=15pt,
	borderline={1.0pt}{0pt}{black,dotted}
}
\newtcolorbox{emptycorollary}{
	blanker,breakable,
	left=3mm,right=3mm,
	top=3mm,bottom=3mm,
	before skip=15pt,after skip=15pt,
	borderline={1.0pt}{0pt}{black,dotted}
}
%----------
\tcolorboxenvironment{example}{
	blanker,breakable,
	left=3mm,right=3mm,
	top=3mm,bottom=3mm,
	before skip=15pt,after skip=15pt,
	borderline={0.5pt}{0pt}{black}
}
%----------
\tcolorboxenvironment{practice}{
	blanker,breakable,
	left=3mm,right=3mm,
	top=3mm,bottom=3mm,
	before skip=15pt,after skip=15pt,
	borderline={0.5pt}{0pt}{black}
}
%----------
\tcolorboxenvironment{proof}{
	blanker,breakable,
	left=3mm,right=3mm,
	top=2mm,bottom=2mm,
	before skip=15pt,after skip=20pt,
	% borderline west={1.5pt}{0pt}{black,dotted}
	borderline vertical={1pt}{0pt}{black,dotted}
	% borderline vertical={0.8pt}{0pt}{black,dotted,arrows={Square[scale=0.5]-Square[scale=0.5]}}
	}
%----------
\tcolorboxenvironment{supplement}{
	blanker,breakable,
	left=3mm,right=3mm,
	top=2mm,bottom=2mm,
	before skip=15pt,after skip=20pt,
	% borderline west={1.5pt}{0pt}{black,dotted}
	% borderline vertical={0.5pt}{0pt}{black,arrows = {Circle[scale=0.7]-Circle[scale=0.7]}}
	borderline vertical={0.5pt}{0pt}{black}
	% borderline vertical={0.5pt}{0pt}{black},
	% borderline north={0.5pt}{0pt}{white,arrows={Circle[black,scale=0.7]-Circle[black,scale=0.7]}}
	}
%----------
\tcolorboxenvironment{remark}{
	blanker,breakable,
	left=3mm,right=3mm,
	top=1mm,bottom=1mm,
	before skip=15pt,after skip=20pt,
	% borderline west={1.5pt}{0pt}{black,dotted}
	% borderline vertical={0.5pt}{0pt}{black,arrows = {Circle[scale=0.7]-Circle[scale=0.7]}}
	borderline vertical={0.5pt}{0pt}{black}
	% borderline vertical={0.5pt}{0pt}{black},
	% borderline north={0.5pt}{0pt}{white,arrows={Circle[black,scale=0.7]-Circle[black,scale=0.7]}}
	}
    
%---------------------
 

%----------
%ハイパーリンク
% 「%」は以降の内容を「改行コードも含めて」無視するコマンド
\usepackage[%
%  dvipdfmx,% 欧文ではコメントアウトする
luatex,%
pdfencoding=auto,%
 setpagesize=false,%
 bookmarks=true,%
 bookmarksdepth=tocdepth,%
 bookmarksnumbered=true,%
 colorlinks=false,%
 pdftitle={},%
 pdfsubject={},%
 pdfauthor={},%
 pdfkeywords={}%
]{hyperref}
%------------


%参照 参照するときに自動で環境名を含んで参照する
\usepackage[nameinlink]{cleveref}
\let\normalref\ref
\renewcommand{\ref}{\cref}
\crefname{definition}{定義}{定義}
\crefname{proposition}{命題}{命題}
\crefname{theorem}{定理}{定理}
\crefname{lemma}{補題}{補題}
\crefname{corollary}{系}{系}
\crefname{example}{例}{例}
\crefname{practice}{演習問題}{演習問題}
\crefname{equation}{式}{式} 
\crefname{chapter}{第}{第}
\creflabelformat{chapter}{#2#1章#3}
\crefname{section}{第}{第}
\creflabelformat{section}{#2#1節#3}
\crefname{subsection}{第}{第}
\creflabelformat{subsection}{#2#1小節#3}
%----------

%---------------------
%章跨ぎの参照が不具合を起こすための代わり
% \mylabl でラベル付け
\newcommand{\mylabel}[1]{
\label{#1}
\hypertarget{#1}{}
}
% \myref で環境名付きリンクをつける
\newcommand{\myref}[1]{
\hyperlink{#1}{\cref*{#1}}
}
%-----------------

\usepackage{autonum} %参照した数式にだけ番号を振る
% \usepackage{docmute} %ファイル分割
% \begin{document}

% %---------- 1章1節
% \input 1-01.tex
% %---------- 1章2節
% \input 1-02.tex
% % ---------- 1章3節
% \input 1-03.tex
% % ---------- 1章4節
% \input 1-04.tex
% % ---------- 1章5節
% \input 1-05.tex
% % ---------- 1章6節
% \input 1-06.tex
% %---------- 1章7節
% \input 1-07.tex
% % ---------- 1章8節
% \input 1-08.tex
% % ---------- 1章9節
% \input 1-09.tex
% % ---------- 1章10節
% \input 1-10.tex
% % ---------- 1章11節
% \input 1-11.tex
% % ---------- 1章12節
% \input 1-12.tex
% % ---------- 参考文献
% \input reference.tex
% \end{document}
% ----------



\usepackage{bxtexlogo}
\usepackage{amsthm}
\usepackage{amsmath}
\usepackage{bbm} %小文字の黒板文字
\usepackage{physics}
\usepackage{amsfonts}
\usepackage{graphicx}
\usepackage{mathtools}
\usepackage{enumitem}
\usepackage[margin=20truemm]{geometry}
\usepackage{textcomp}
\usepackage{bm}
\usepackage{mathrsfs}
\usepackage{latexsym}
\usepackage{amssymb}
\usepackage{algorithmic}
\usepackage{algorithm}
\usepackage{tikz}
\usetikzlibrary{arrows.meta}
\usetikzlibrary{math,matrix,backgrounds}
\usetikzlibrary{angles}
\usetikzlibrary{calc}


%----------
%日本語フォント
% \usepackage[deluxe]{otf} platex用 lualatexでは動かない

%----------
%欧文フォント
\usepackage[T1]{fontenc}

%----------
%文字色
\usepackage{color}

%----------
\setlength{\parindent}{2\zw} %インデントの設定

%----------
% %参照した数式にだけ番号を振る cleverrefと併用するとうまくいかない
% \mathtoolsset{showonlyrefs=true}
%----------

%----------
%集合の中線
\newcommand{\relmiddle}[1]{\mathrel{}\middle#1\mathrel{}}
% \middle| の代わりに \relmiddle| を付ける
\newcommand{\sgn}{\mathop{\mathrm{sgn}}} %置換sgn
\newcommand{\Int}{\mathop{\mathrm{Int}}} %位相空間の内部Int
\newcommand{\Ext}{\mathop{\mathrm{Ext}}} %位相空間の外部Ext
\newcommand{\Cl}{\mathop{\mathrm{Cl}}} %位相空間の閉包Cl
\newcommand{\supp}{\mathop{\mathrm{supp}}} %関数の台supp
\newcommand{\restrict}[2]{\left. #1 \right \vert_{#2}}%関数の制限 \restrict{f}{A} = f|_A
\newcommand{\Ker}{\mathop{\mathrm{Ker}}}
\newcommand{\Coker}{\mathop{\mathrm{Coker}}}
\newcommand{\coker}{\mathop{\mathrm{coker}}}
\newcommand{\Coim}{\mathop{\mathrm{Coim}}}
\newcommand{\coim}{\mathop{\mathrm{coim}}}
\newcommand{\id}{\mathop{\mathrm{id}}}
\newcommand{\Gal}{\mathop{\mathrm{Gal}}}

\newtheorem{definition}{定義}[section]

\usepackage{aliascnt}

% \newaliastheorem{(環境とカウンターの名前)}{(元となるカウンターの名前)}{(表示される文字列)}
\newcommand*{\newaliastheorem}[3]{%
  \newaliascnt{#1}{#2}%
  \newtheorem{#1}[#1]{#3}%
  \aliascntresetthe{#1}%
  \expandafter\newcommand\csname #1autorefname\endcsname{#3}%
}
\newaliastheorem{proposition}{definition}{命題} 
\newaliastheorem{theorem}{definition}{定理}
\newaliastheorem{lemma}{definition}{補題}
\newaliastheorem{corollary}{definition}{系}
\newaliastheorem{example}{definition}{例}
\newaliastheorem{practice}{definition}{演習問題}

\newtheorem*{longproof}{証明}
\newtheorem*{answer}{解答}
\newtheorem*{supplement}{補足}
\newtheorem*{remark}{注意}
%----------

%----------
%古い記法を注意するパッケージ
\RequirePackage[l2tabu, orthodox]{nag}
%----------


% 定理環境(tcolorbox)
\usepackage{tcolorbox} %箱
\tcbuselibrary{breakable,skins,theorems}
\tcolorboxenvironment{definition}{
	blanker,breakable,
	left=3mm,right=3mm,
	top=2mm,bottom=2mm,
	before skip=15pt,after skip=20pt,
	borderline ={0.5pt}{0pt}{black}
}
\newtcolorbox{emptydefinition}{
	blanker,breakable,
	left=3mm,right=3mm,
	top=2mm,bottom=2mm,
	before skip=15pt,after skip=20pt,
	borderline ={0.5pt}{0pt}{black}
}
%----------
\tcolorboxenvironment{proposition}{
	blanker,breakable,
	left=3mm,right=3mm,
	top=3mm,bottom=3mm,
	before skip=15pt,after skip=15pt,
	borderline={0.5pt}{0pt}{black}
}
\newtcolorbox{emptyproposition}{
	blanker,breakable,
	left=3mm,right=3mm,
	top=3mm,bottom=3mm,
	before skip=15pt,after skip=15pt,
	borderline={0.5pt}{0pt}{black}
}
%----------
\tcolorboxenvironment{theorem}{
	blanker,breakable,
	left=3mm,right=3mm,
	top=3mm,bottom=3mm,
    sharp corners,boxrule=0.6pt,
	before skip=15pt,after skip=15pt,
	borderline={0.5pt}{0pt}{black},
    borderline={0.5pt}{1.5pt}{black}
}
\newtcolorbox{emptytheorem}{
	blanker,breakable,
	left=3mm,right=3mm,
	top=3mm,bottom=3mm,
    sharp corners,boxrule=0.6pt,
	before skip=15pt,after skip=15pt,
	borderline={0.5pt}{0pt}{black},
    borderline={0.5pt}{1.5pt}{black}
}
%----------
\tcolorboxenvironment{lemma}{
	blanker,breakable,
	left=3mm,right=3mm,
	top=3mm,bottom=3mm,
	before skip=15pt,after skip=15pt,
	borderline={0.5pt}{0pt}{black}
}
%----------
\tcolorboxenvironment{corollary}{
	blanker,breakable,
	left=3mm,right=3mm,
	top=3mm,bottom=3mm,
	before skip=15pt,after skip=15pt,
	borderline={1.0pt}{0pt}{black,dotted}
}
\newtcolorbox{emptycorollary}{
	blanker,breakable,
	left=3mm,right=3mm,
	top=3mm,bottom=3mm,
	before skip=15pt,after skip=15pt,
	borderline={1.0pt}{0pt}{black,dotted}
}
%----------
\tcolorboxenvironment{example}{
	blanker,breakable,
	left=3mm,right=3mm,
	top=3mm,bottom=3mm,
	before skip=15pt,after skip=15pt,
	borderline={0.5pt}{0pt}{black}
}
%----------
\tcolorboxenvironment{practice}{
	blanker,breakable,
	left=3mm,right=3mm,
	top=3mm,bottom=3mm,
	before skip=15pt,after skip=15pt,
	borderline={0.5pt}{0pt}{black}
}
%----------
\tcolorboxenvironment{proof}{
	blanker,breakable,
	left=3mm,right=3mm,
	top=2mm,bottom=2mm,
	before skip=15pt,after skip=20pt,
	% borderline west={1.5pt}{0pt}{black,dotted}
	borderline vertical={1pt}{0pt}{black,dotted}
	% borderline vertical={0.8pt}{0pt}{black,dotted,arrows={Square[scale=0.5]-Square[scale=0.5]}}
	}
%----------
\tcolorboxenvironment{supplement}{
	blanker,breakable,
	left=3mm,right=3mm,
	top=2mm,bottom=2mm,
	before skip=15pt,after skip=20pt,
	% borderline west={1.5pt}{0pt}{black,dotted}
	% borderline vertical={0.5pt}{0pt}{black,arrows = {Circle[scale=0.7]-Circle[scale=0.7]}}
	borderline vertical={0.5pt}{0pt}{black}
	% borderline vertical={0.5pt}{0pt}{black},
	% borderline north={0.5pt}{0pt}{white,arrows={Circle[black,scale=0.7]-Circle[black,scale=0.7]}}
	}
%----------
\tcolorboxenvironment{remark}{
	blanker,breakable,
	left=3mm,right=3mm,
	top=1mm,bottom=1mm,
	before skip=15pt,after skip=20pt,
	% borderline west={1.5pt}{0pt}{black,dotted}
	% borderline vertical={0.5pt}{0pt}{black,arrows = {Circle[scale=0.7]-Circle[scale=0.7]}}
	borderline vertical={0.5pt}{0pt}{black}
	% borderline vertical={0.5pt}{0pt}{black},
	% borderline north={0.5pt}{0pt}{white,arrows={Circle[black,scale=0.7]-Circle[black,scale=0.7]}}
	}
    
%---------------------
 

%----------
%ハイパーリンク
% 「%」は以降の内容を「改行コードも含めて」無視するコマンド
\usepackage[%
%  dvipdfmx,% 欧文ではコメントアウトする
luatex,%
pdfencoding=auto,%
 setpagesize=false,%
 bookmarks=true,%
 bookmarksdepth=tocdepth,%
 bookmarksnumbered=true,%
 colorlinks=false,%
 pdftitle={},%
 pdfsubject={},%
 pdfauthor={},%
 pdfkeywords={}%
]{hyperref}
%------------


%参照 参照するときに自動で環境名を含んで参照する
\usepackage[nameinlink]{cleveref}
\let\normalref\ref
\renewcommand{\ref}{\cref}
\crefname{definition}{定義}{定義}
\crefname{proposition}{命題}{命題}
\crefname{theorem}{定理}{定理}
\crefname{lemma}{補題}{補題}
\crefname{corollary}{系}{系}
\crefname{example}{例}{例}
\crefname{practice}{演習問題}{演習問題}
\crefname{equation}{式}{式} 
\crefname{chapter}{第}{第}
\creflabelformat{chapter}{#2#1章#3}
\crefname{section}{第}{第}
\creflabelformat{section}{#2#1節#3}
\crefname{subsection}{第}{第}
\creflabelformat{subsection}{#2#1小節#3}
%----------

%---------------------
%章跨ぎの参照が不具合を起こすための代わり
% \mylabl でラベル付け
\newcommand{\mylabel}[1]{
\label{#1}
\hypertarget{#1}{}
}
% \myref で環境名付きリンクをつける
\newcommand{\myref}[1]{
\hyperlink{#1}{\cref*{#1}}
}
%-----------------

\usepackage{autonum} %参照した数式にだけ番号を振る
\usepackage{docmute} %ファイル分割
\begin{document}

%\chapter{章のタイトル}
\section{H26数学必修}
\url{https://www.math.kobe-u.ac.jp/HOME/home-j/gif/h26.pdf}

\fbox{1}
(1)$\begin{pmatrix}
    x \\y
    \end{pmatrix}=(x+y)/2 \begin{pmatrix}
        1\\1
    \end{pmatrix}+(x-y)/2 \begin{pmatrix}
        1\\-1
    \end{pmatrix}$である.$x=y=0$でないなら一次従属である.$x=y=0$のときは$c\neq 0$にたいして$c \begin{pmatrix}
        x \\y
    \end{pmatrix}=0$となるから一次独立でない.

(2)$\begin{vmatrix}
    5 & -2 & x \\
    2 & -1 & y \\
    3 & 1 & z
\end{vmatrix}=\begin{vmatrix}
    1 & -2 & x \\
    0 & -1 & y\\
    0 & 11 & z-5x
\end{vmatrix}=-(z-5x)-11y$である.$5x-11y=z$が成り立つとき一次従属.

(3)$\{ u_1\}$は一次独立である.$\{ u_1,\dots,u_{k-1}\}$が一次独立と仮定する.このとき$c_1u_1+\dots+c_ku_k=0$とする.$\lambda_kc_1u_1+\dots+\lambda_kc_ku_k=0$である.また$A$をかけると$\lambda_1c_1u_1+\dots+\lambda_kc_ku_k=0$である.よって$\lambda_1c_1u_1+\dots+\lambda_kc_{k-1}u_{k-1}=\lambda_kc_1u_1+\dots+\lambda_kc_{k-1}u_{k-1}$である.一次独立性から$\lambda_ic_i=\lambda_kc_i$であり$\lambda_i\neq \lambda_k$より$c_i=0$である.このとき$c_k=0$であるから$\{u_1,\dots,u_k\}$は一次独立である.

(4)(a)$c_0x+c_1Ax+\dots c_{n-1}A^{n-1}x=0$とする.両辺に$A^{n-1}$をかけると$c_0A^{n-1}x=0$である.$A^{n-1}x\neq 0$であるから$c_0=0$である.次に$A^{n-2}$をかければ$c_1=0$が分かる.これを繰り返して$c_0=\dots=c_{n-1}=0$であるから一次独立.

(b)$n$個のベクトル$\{x,Ax,\dots,A^{n-1}x\}$が一次独立であるからこれは$\mathbb{R}^n$の基底である.よって任意の$y \in \mathbb{R}^n$は$y=c_0x+\dots+c_{n-1}A^{n-1}x$とかける.$A^ny=c_0A^nx+\dots+c_{n-1}A^{2n-1}x=0$であるから$A^n=O$である.

\fbox{2}
(1)(a)$\varphi\colon \mathbb{R}\rightarrow \mathbb{R}_+;x\mapsto e^x$は全単射である.$\varphi(x+y)=e^{x+y}=e^xe^y=\varphi(x)\varphi(y)$であるから$\varphi$は群の同型写像である.

(b)$\varphi\colon \mathbb{Q} \rightarrow \mathbb{Q}_+$が群同型写像とする.
$\varphi(1)=\alpha$とする.$\varphi(\sum\limits_{i=1}^n 1/n)=\varphi(1/n)^n=\alpha$である.すなわち$\varphi(1/n)=\alpha^{1/n}$である.$\alpha^{1/n}$が有理数にならないような$n$は存在するからこれは$\varphi$が同型写像であることに矛盾.

(2)(a)$G'$の単位元を$e'$とする.
$f \colon G \rightarrow G'$に対して$\ker f= \{ x\in G \mid f(x)=e'\}$である.$g \in G,x \in \ker f$に対して$f(g^{-1}xg)=f(g)^{-1}f(x)f(g)=f(g)^{-1}f(g)=e'$より$g^{-1}xg \in \ker f$であるから$\ker f$は正規部分群.

(b)$f(g+g')=(f_1(g+g'),f_2(g+g'))=(f_1(g)+f_1(g'),f_2(g)+f_2(g'))=(f_1(g),f_2(g))+(f_1(g'),f_2(g'))=f(g)+f(g')$である.よって$f$は準同型.

(3)$\pi_1 \colon G \rightarrow G/N_1,\pi_2 \colon G \rightarrow G/N_2$を自然な全射とする.$\pi \colon G \rightarrow G/N_1 \times G/N_2$を$\pi(x)=(\pi_1(x),\pi_2(x))$で定める.(2)から$\pi$は準同型である.$\ker \pi =\{ x \in G \mid \pi(x)=(e_1,e_2)\}=\{ x \in G \mid \pi_1(x)=e_1,\pi_2(x)=e_2\}=\{ x \in G \mid x \in N_1,x \in N_2\}=N_1 \cap N_2$である.

よって$N_1 \cap N_2$は正規部分群で準同型定理から$G/(N_1 \cap N_2)\cong G/N_1 \times G/N_2$である.右辺はアーベル群であるから,左辺もアーベル群である.

\fbox{3}
(1)(a)$x>0$で$1-1/(1+x)>0$である.よって$0<\int_0^x 1-1/(1+t)dt=x-\log(1+x)$である.

$x>0$で$x-1+1/(1+x) \ge 2\sqrt{(x+1)\frac{1}{1+x}}-2=0$であるから$0<\int_0^x t-1+1/(1+t)dt=x^2/2-x+\log(1+x)$である.
すなわち$x-\log(1+x)<x^2/2$である.

(b)$\sum\limits_{n=1}^\infty (\frac{1}{n}-\log(1+\frac{1}{n}))\le \sum\limits_{n=1}^\infty \frac{1}{2n^2}<\infty$であるから収束する.

(2)$f_x(x,y)=2xe^{x-y^2}+(x^2-8y)e^{x-y^2}=(x^2+2x-8y)e^{x-y^2}$である.$f_y(x,y)=-8e^{x-y^2}-2y(x^2-8y)e^{x-y^2}=(16y^2-2x^2y-8)e^{x-y^2}$である.
$f_x(x,y)=f_y(x,y)=0$とすると,$x^2+2x-8y=0,16y^2-2x^2y-8=0$である.
これを解く.$(x^2+2x)y=x^2y+4$より$xy=2$である.よって$x\neq 0$であるから$8y^2-4/y-4=0$である.よって$8y^3-4y-4=0$である.$8y^3-4y-4=(y-1)(8y^2+8y+4)$であり$2y^2+2y+1=2(y+1/2)^2+1/2>0$であるから$x=2,y=1$が唯一の実数解である.

$f_{xx}=(2x+2)e^{x-y^2}+(x^2+2x-8y)e^{x-y^2}=(x^2+4x-8y+2)e^{x-y^2}$,
$f_{xy}=-8e^{x-y^2}-2y(x^2+2x-8y)e^{x-y^2}=(-2x^2y-4xy+16y^2-8)e^{x-y^2}$,$f_{yy}=(32y-2x^2)e^{x-y^2}-2y(16y^2-2x^2y-8)e^{x-y^2}=(-32y^3+4x^2y^2+48y-2x^2)e^{x-y^2}$である.よって$(2,1)$におけるヘッセ行列は$\begin{pmatrix}
    6e & -8e \\
    -8e & 24e
\end{pmatrix}=80e^2>0$であり$f_{xx}|_{(2,1)}=6e>0$より$f$は$(2,1)$で極小値$f(2,1)=-4e$をとる.

(3)$\sqrt{9-x^2-y^2}=\sqrt{1+x^2+y^2}$をみたすのは$x^2+y^2=4$のときである.よって求める体積を$V$とする.$D=\{ (x,y) \mid x^2+y^2\le 4\}$とする.$V=\iint_D \sqrt{9-x^2-y^2}-\sqrt{1+x^2+y^2}dxdy$である.
$(x,y)=(r\cos \theta,r\sin\theta) \quad(0\le r \le 2,0\le 2\pi)$と極座標変換するとヤコビアンは$r$である.
\begin{align}
    V&=\int_0^{2\pi}\int_0^2 (\sqrt{9-r^2}-\sqrt{1+r^2})rdrdd\theta\\
    &=2\pi \int_0^2 (\sqrt{9-r^2}-\sqrt{1+r^2})rdr\\
    &=2\pi \qty( \int_0^2 r\sqrt{9-r^2}dr-\int_0^2 r\sqrt{1+r^2}dr)\\
    &=2\pi \qty( \qty[-\frac{1}{3}(9-r^2)^{3/2}]_0^2-\qty[\frac{1}{3}(1+r^2)^{3/2}]_0^2)\\
    &=2\pi \qty( -\frac{1}{3}(5^{3/2}-27)-\frac{1}{3}(5^{3/2}-1))\\
    &=4\pi \qty( 14-5\sqrt{5})/3
\end{align}

(4)任意の$\varepsilon$に対してある$N\in \mathbb{N}$が存在して任意の$n \ge N$に対して$\alpha-\varepsilon<a_{n+1}-a_n<\alpha+\varepsilon$である.
よって$a_n+\alpha-\varepsilon<a_{n+1}<a_n+\alpha+\varepsilon$である.
すなわち任意の$k\in \mathbb{N}$に対して
$a_{N}+k\alpha-k\varepsilon<a_{N+1}+(k-1)\alpha+(k-1)\varepsilon<\dots<a_{N+k}<a_{N}+k\alpha+k\varepsilon$がなりたつ.よって
$\frac{a_N+k\alpha-k\varepsilon}{N+k}<\frac{a_{N+k}}{N+k}<\frac{a_N+k\alpha+k\varepsilon}{N+k}$である.$k\rightarrow \infty$とすると$\alpha-\varepsilon<\lim\limits_{n\rightarrow \infty}\frac{a_n}{n}<\alpha+\varepsilon$より$\lim\limits_{n\rightarrow \infty}\frac{a_n}{n}=\alpha$である.

\fbox{4}
(1)$\{ 2,3,4\}\subset A$である.$0 \in C \in \mathcal{O}$に対して$X \setminus C$が空集合,または有限集合であるから$C$は無限集合である.よって$C$は$A$の部分集合でない.よって$A^i=\{ 2,3,4\}$である.$x \in X \setminus A$は$x \neq 0$であるから$x \in \{ x\} \subset X \setminus A$より$A^e=X \setminus A$である.$0 \in C \in \mathcal{O}$は$C \cap X \setminus A\neq \emptyset$となるから$A^f=\{ 0\}$.

(2)$x,y \in X \setminus \{0\}$に対して$x \in \{ x\} \in \mathcal{O},y\in \{ y\} \in \mathcal{O} ,\{x\} \cap \{ y\}=\emptyset$である.
$x=0,y\neq 0$に対して$x \in X\setminus \{ y\} \in \mathcal{O}$であり
$x \in \{ x\} \cap (X\setminus \{ y\})=\emptyset$である.よってハウスドルフ.

(3)$X$の開被覆$S=\{ U_\lambda \mid \lambda \in \Lambda\}$を任意にとる.
$0 \in U_{\lambda}$となる$\lambda\in \Lambda$が存在する.$0 \in U_{\lambda}$であるから$X\setminus U_{\lambda}$は有限集合か空集合である.空集合なら$\{ u_\lambda\}$が有限部分被覆となる.
有限集合なら$X\setminus U_{\lambda}=\{ x_1,\dots,x_n\}$とできる.$x_i \in U_{\lambda_i}$となる$\lambda_i \in \Lambda$が存在する.$\{ U_{\lambda_1},\dots,U_{\lambda_n},U_\lambda \}$が有限部分被覆となる.よってコンパクト.

(4)$f \colon X \rightarrow Y;n \rightarrow 1/n$とする.$f$は全単射である.$U$を$Y$の開集合とする.$0 \notin U$なら$0\notin f^{-1}(U)$より$f^{-1}(U)\in \mathcal{O}$である.$0 \in U$ならある$\varepsilon>0$が存在して$(-\varepsilon,\varepsilon)\cap Y \subset U$である.$0<1/n <\varepsilon$なる$n$は無限個存在するから$U$は無限集合である.よって$f^{-1}(U)$は$0$を含まない無限集合であるから$f^{-1}(U)\in \mathcal{O}$である.よって$f$は連続.

$0 \in B \in \mathcal{O}$に対して$Y\setminus f(B)=f(X\setminus A)=\{ 1/n_1,\dots,1/n_m\}$は閉集合.よって$f(B)$は開集合.
$0 \notin A \in \mathcal{O}$に対して$f(A)=\{ 1/n \mid n \in A\}$である.
$1/(n+1)<1/n-\varepsilon_n<1/n<1/n+\varepsilon_n<1/(n-1)$なる$\varepsilon_n$が存在する.よって$(1/n-\varepsilon_n,1/n+\varepsilon_n)\cap Y = \{ 1/n \}$より$f(A)$は開集合.よって$f$は同相.

\end{document}