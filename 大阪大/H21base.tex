\documentclass[
		book,
		head_space=20mm,
		foot_space=20mm,
		gutter=10mm,
		line_length=190mm
]{jlreq}

%----------
%LuaLaTeXで実行する!!
%----------
%各章節には以下を書く. 1-03.texのような名前にする
%----------
% \documentclass[
% 		book,
% 		head_space=20mm,
% 		foot_space=20mm,
% 		gutter=10mm,
% 		line_length=190mm
% ]{jlreq}
% 
%----------
%LuaLaTeXで実行する!!
%----------
%各章節には以下を書く. 1-03.texのような名前にする
%----------
% \documentclass[
% 		book,
% 		head_space=20mm,
% 		foot_space=20mm,
% 		gutter=10mm,
% 		line_length=190mm
% ]{jlreq}
% 
%----------
%LuaLaTeXで実行する!!
%----------
%各章節には以下を書く. 1-03.texのような名前にする
%----------
% \documentclass[
% 		book,
% 		head_space=20mm,
% 		foot_space=20mm,
% 		gutter=10mm,
% 		line_length=190mm
% ]{jlreq}
% \input {preamble.tex}
% \usepackage{docmute} %ファイル分割
% \begin{document}

% %\chapter{章のタイトル}
% \section{節のタイトル}
% no text

% \end{document}
%----------

%main.texには以下を書く
%----------
% \documentclass[
% 		book,
% 		head_space=20mm,
% 		foot_space=20mm,
% 		gutter=10mm,
% 		line_length=190mm,
%         openany
% ]{jlreq}
% \input {preamble.tex}
% \usepackage{docmute} %ファイル分割
% \begin{document}

% %---------- 1章1節
% \input 1-01.tex
% %---------- 1章2節
% \input 1-02.tex
% % ---------- 1章3節
% \input 1-03.tex
% % ---------- 1章4節
% \input 1-04.tex
% % ---------- 1章5節
% \input 1-05.tex
% % ---------- 1章6節
% \input 1-06.tex
% %---------- 1章7節
% \input 1-07.tex
% % ---------- 1章8節
% \input 1-08.tex
% % ---------- 1章9節
% \input 1-09.tex
% % ---------- 1章10節
% \input 1-10.tex
% % ---------- 1章11節
% \input 1-11.tex
% % ---------- 1章12節
% \input 1-12.tex
% % ---------- 参考文献
% \input reference.tex
% \end{document}
% ----------



\usepackage{bxtexlogo}
\usepackage{amsthm}
\usepackage{amsmath}
\usepackage{bbm} %小文字の黒板文字
\usepackage{physics}
\usepackage{amsfonts}
\usepackage{graphicx}
\usepackage{mathtools}
\usepackage{enumitem}
\usepackage[margin=20truemm]{geometry}
\usepackage{textcomp}
\usepackage{bm}
\usepackage{mathrsfs}
\usepackage{latexsym}
\usepackage{amssymb}
\usepackage{algorithmic}
\usepackage{algorithm}
\usepackage{tikz}
\usetikzlibrary{arrows.meta}
\usetikzlibrary{math,matrix,backgrounds}
\usetikzlibrary{angles}
\usetikzlibrary{calc}


%----------
%日本語フォント
% \usepackage[deluxe]{otf} platex用 lualatexでは動かない

%----------
%欧文フォント
\usepackage[T1]{fontenc}

%----------
%文字色
\usepackage{color}

%----------
\setlength{\parindent}{2\zw} %インデントの設定

%----------
% %参照した数式にだけ番号を振る cleverrefと併用するとうまくいかない
% \mathtoolsset{showonlyrefs=true}
%----------

%----------
%集合の中線
\newcommand{\relmiddle}[1]{\mathrel{}\middle#1\mathrel{}}
% \middle| の代わりに \relmiddle| を付ける
\newcommand{\sgn}{\mathop{\mathrm{sgn}}} %置換sgn
\newcommand{\Int}{\mathop{\mathrm{Int}}} %位相空間の内部Int
\newcommand{\Ext}{\mathop{\mathrm{Ext}}} %位相空間の外部Ext
\newcommand{\Cl}{\mathop{\mathrm{Cl}}} %位相空間の閉包Cl
\newcommand{\supp}{\mathop{\mathrm{supp}}} %関数の台supp
\newcommand{\restrict}[2]{\left. #1 \right \vert_{#2}}%関数の制限 \restrict{f}{A} = f|_A
\newcommand{\Ker}{\mathop{\mathrm{Ker}}}
\newcommand{\Coker}{\mathop{\mathrm{Coker}}}
\newcommand{\coker}{\mathop{\mathrm{coker}}}
\newcommand{\Coim}{\mathop{\mathrm{Coim}}}
\newcommand{\coim}{\mathop{\mathrm{coim}}}
\newcommand{\id}{\mathop{\mathrm{id}}}
\newcommand{\Gal}{\mathop{\mathrm{Gal}}}

\newtheorem{definition}{定義}[section]

\usepackage{aliascnt}

% \newaliastheorem{(環境とカウンターの名前)}{(元となるカウンターの名前)}{(表示される文字列)}
\newcommand*{\newaliastheorem}[3]{%
  \newaliascnt{#1}{#2}%
  \newtheorem{#1}[#1]{#3}%
  \aliascntresetthe{#1}%
  \expandafter\newcommand\csname #1autorefname\endcsname{#3}%
}
\newaliastheorem{proposition}{definition}{命題} 
\newaliastheorem{theorem}{definition}{定理}
\newaliastheorem{lemma}{definition}{補題}
\newaliastheorem{corollary}{definition}{系}
\newaliastheorem{example}{definition}{例}
\newaliastheorem{practice}{definition}{演習問題}

\newtheorem*{longproof}{証明}
\newtheorem*{answer}{解答}
\newtheorem*{supplement}{補足}
\newtheorem*{remark}{注意}
%----------

%----------
%古い記法を注意するパッケージ
\RequirePackage[l2tabu, orthodox]{nag}
%----------


% 定理環境(tcolorbox)
\usepackage{tcolorbox} %箱
\tcbuselibrary{breakable,skins,theorems}
\tcolorboxenvironment{definition}{
	blanker,breakable,
	left=3mm,right=3mm,
	top=2mm,bottom=2mm,
	before skip=15pt,after skip=20pt,
	borderline ={0.5pt}{0pt}{black}
}
\newtcolorbox{emptydefinition}{
	blanker,breakable,
	left=3mm,right=3mm,
	top=2mm,bottom=2mm,
	before skip=15pt,after skip=20pt,
	borderline ={0.5pt}{0pt}{black}
}
%----------
\tcolorboxenvironment{proposition}{
	blanker,breakable,
	left=3mm,right=3mm,
	top=3mm,bottom=3mm,
	before skip=15pt,after skip=15pt,
	borderline={0.5pt}{0pt}{black}
}
\newtcolorbox{emptyproposition}{
	blanker,breakable,
	left=3mm,right=3mm,
	top=3mm,bottom=3mm,
	before skip=15pt,after skip=15pt,
	borderline={0.5pt}{0pt}{black}
}
%----------
\tcolorboxenvironment{theorem}{
	blanker,breakable,
	left=3mm,right=3mm,
	top=3mm,bottom=3mm,
    sharp corners,boxrule=0.6pt,
	before skip=15pt,after skip=15pt,
	borderline={0.5pt}{0pt}{black},
    borderline={0.5pt}{1.5pt}{black}
}
\newtcolorbox{emptytheorem}{
	blanker,breakable,
	left=3mm,right=3mm,
	top=3mm,bottom=3mm,
    sharp corners,boxrule=0.6pt,
	before skip=15pt,after skip=15pt,
	borderline={0.5pt}{0pt}{black},
    borderline={0.5pt}{1.5pt}{black}
}
%----------
\tcolorboxenvironment{lemma}{
	blanker,breakable,
	left=3mm,right=3mm,
	top=3mm,bottom=3mm,
	before skip=15pt,after skip=15pt,
	borderline={0.5pt}{0pt}{black}
}
%----------
\tcolorboxenvironment{corollary}{
	blanker,breakable,
	left=3mm,right=3mm,
	top=3mm,bottom=3mm,
	before skip=15pt,after skip=15pt,
	borderline={1.0pt}{0pt}{black,dotted}
}
\newtcolorbox{emptycorollary}{
	blanker,breakable,
	left=3mm,right=3mm,
	top=3mm,bottom=3mm,
	before skip=15pt,after skip=15pt,
	borderline={1.0pt}{0pt}{black,dotted}
}
%----------
\tcolorboxenvironment{example}{
	blanker,breakable,
	left=3mm,right=3mm,
	top=3mm,bottom=3mm,
	before skip=15pt,after skip=15pt,
	borderline={0.5pt}{0pt}{black}
}
%----------
\tcolorboxenvironment{practice}{
	blanker,breakable,
	left=3mm,right=3mm,
	top=3mm,bottom=3mm,
	before skip=15pt,after skip=15pt,
	borderline={0.5pt}{0pt}{black}
}
%----------
\tcolorboxenvironment{proof}{
	blanker,breakable,
	left=3mm,right=3mm,
	top=2mm,bottom=2mm,
	before skip=15pt,after skip=20pt,
	% borderline west={1.5pt}{0pt}{black,dotted}
	borderline vertical={1pt}{0pt}{black,dotted}
	% borderline vertical={0.8pt}{0pt}{black,dotted,arrows={Square[scale=0.5]-Square[scale=0.5]}}
	}
%----------
\tcolorboxenvironment{supplement}{
	blanker,breakable,
	left=3mm,right=3mm,
	top=2mm,bottom=2mm,
	before skip=15pt,after skip=20pt,
	% borderline west={1.5pt}{0pt}{black,dotted}
	% borderline vertical={0.5pt}{0pt}{black,arrows = {Circle[scale=0.7]-Circle[scale=0.7]}}
	borderline vertical={0.5pt}{0pt}{black}
	% borderline vertical={0.5pt}{0pt}{black},
	% borderline north={0.5pt}{0pt}{white,arrows={Circle[black,scale=0.7]-Circle[black,scale=0.7]}}
	}
%----------
\tcolorboxenvironment{remark}{
	blanker,breakable,
	left=3mm,right=3mm,
	top=1mm,bottom=1mm,
	before skip=15pt,after skip=20pt,
	% borderline west={1.5pt}{0pt}{black,dotted}
	% borderline vertical={0.5pt}{0pt}{black,arrows = {Circle[scale=0.7]-Circle[scale=0.7]}}
	borderline vertical={0.5pt}{0pt}{black}
	% borderline vertical={0.5pt}{0pt}{black},
	% borderline north={0.5pt}{0pt}{white,arrows={Circle[black,scale=0.7]-Circle[black,scale=0.7]}}
	}
    
%---------------------
 

%----------
%ハイパーリンク
% 「%」は以降の内容を「改行コードも含めて」無視するコマンド
\usepackage[%
%  dvipdfmx,% 欧文ではコメントアウトする
luatex,%
pdfencoding=auto,%
 setpagesize=false,%
 bookmarks=true,%
 bookmarksdepth=tocdepth,%
 bookmarksnumbered=true,%
 colorlinks=false,%
 pdftitle={},%
 pdfsubject={},%
 pdfauthor={},%
 pdfkeywords={}%
]{hyperref}
%------------


%参照 参照するときに自動で環境名を含んで参照する
\usepackage[nameinlink]{cleveref}
\let\normalref\ref
\renewcommand{\ref}{\cref}
\crefname{definition}{定義}{定義}
\crefname{proposition}{命題}{命題}
\crefname{theorem}{定理}{定理}
\crefname{lemma}{補題}{補題}
\crefname{corollary}{系}{系}
\crefname{example}{例}{例}
\crefname{practice}{演習問題}{演習問題}
\crefname{equation}{式}{式} 
\crefname{chapter}{第}{第}
\creflabelformat{chapter}{#2#1章#3}
\crefname{section}{第}{第}
\creflabelformat{section}{#2#1節#3}
\crefname{subsection}{第}{第}
\creflabelformat{subsection}{#2#1小節#3}
%----------

%---------------------
%章跨ぎの参照が不具合を起こすための代わり
% \mylabl でラベル付け
\newcommand{\mylabel}[1]{
\label{#1}
\hypertarget{#1}{}
}
% \myref で環境名付きリンクをつける
\newcommand{\myref}[1]{
\hyperlink{#1}{\cref*{#1}}
}
%-----------------

\usepackage{autonum} %参照した数式にだけ番号を振る
% \usepackage{docmute} %ファイル分割
% \begin{document}

% %\chapter{章のタイトル}
% \section{節のタイトル}
% no text

% \end{document}
%----------

%main.texには以下を書く
%----------
% \documentclass[
% 		book,
% 		head_space=20mm,
% 		foot_space=20mm,
% 		gutter=10mm,
% 		line_length=190mm,
%         openany
% ]{jlreq}
% 
%----------
%LuaLaTeXで実行する!!
%----------
%各章節には以下を書く. 1-03.texのような名前にする
%----------
% \documentclass[
% 		book,
% 		head_space=20mm,
% 		foot_space=20mm,
% 		gutter=10mm,
% 		line_length=190mm
% ]{jlreq}
% \input {preamble.tex}
% \usepackage{docmute} %ファイル分割
% \begin{document}

% %\chapter{章のタイトル}
% \section{節のタイトル}
% no text

% \end{document}
%----------

%main.texには以下を書く
%----------
% \documentclass[
% 		book,
% 		head_space=20mm,
% 		foot_space=20mm,
% 		gutter=10mm,
% 		line_length=190mm,
%         openany
% ]{jlreq}
% \input {preamble.tex}
% \usepackage{docmute} %ファイル分割
% \begin{document}

% %---------- 1章1節
% \input 1-01.tex
% %---------- 1章2節
% \input 1-02.tex
% % ---------- 1章3節
% \input 1-03.tex
% % ---------- 1章4節
% \input 1-04.tex
% % ---------- 1章5節
% \input 1-05.tex
% % ---------- 1章6節
% \input 1-06.tex
% %---------- 1章7節
% \input 1-07.tex
% % ---------- 1章8節
% \input 1-08.tex
% % ---------- 1章9節
% \input 1-09.tex
% % ---------- 1章10節
% \input 1-10.tex
% % ---------- 1章11節
% \input 1-11.tex
% % ---------- 1章12節
% \input 1-12.tex
% % ---------- 参考文献
% \input reference.tex
% \end{document}
% ----------



\usepackage{bxtexlogo}
\usepackage{amsthm}
\usepackage{amsmath}
\usepackage{bbm} %小文字の黒板文字
\usepackage{physics}
\usepackage{amsfonts}
\usepackage{graphicx}
\usepackage{mathtools}
\usepackage{enumitem}
\usepackage[margin=20truemm]{geometry}
\usepackage{textcomp}
\usepackage{bm}
\usepackage{mathrsfs}
\usepackage{latexsym}
\usepackage{amssymb}
\usepackage{algorithmic}
\usepackage{algorithm}
\usepackage{tikz}
\usetikzlibrary{arrows.meta}
\usetikzlibrary{math,matrix,backgrounds}
\usetikzlibrary{angles}
\usetikzlibrary{calc}


%----------
%日本語フォント
% \usepackage[deluxe]{otf} platex用 lualatexでは動かない

%----------
%欧文フォント
\usepackage[T1]{fontenc}

%----------
%文字色
\usepackage{color}

%----------
\setlength{\parindent}{2\zw} %インデントの設定

%----------
% %参照した数式にだけ番号を振る cleverrefと併用するとうまくいかない
% \mathtoolsset{showonlyrefs=true}
%----------

%----------
%集合の中線
\newcommand{\relmiddle}[1]{\mathrel{}\middle#1\mathrel{}}
% \middle| の代わりに \relmiddle| を付ける
\newcommand{\sgn}{\mathop{\mathrm{sgn}}} %置換sgn
\newcommand{\Int}{\mathop{\mathrm{Int}}} %位相空間の内部Int
\newcommand{\Ext}{\mathop{\mathrm{Ext}}} %位相空間の外部Ext
\newcommand{\Cl}{\mathop{\mathrm{Cl}}} %位相空間の閉包Cl
\newcommand{\supp}{\mathop{\mathrm{supp}}} %関数の台supp
\newcommand{\restrict}[2]{\left. #1 \right \vert_{#2}}%関数の制限 \restrict{f}{A} = f|_A
\newcommand{\Ker}{\mathop{\mathrm{Ker}}}
\newcommand{\Coker}{\mathop{\mathrm{Coker}}}
\newcommand{\coker}{\mathop{\mathrm{coker}}}
\newcommand{\Coim}{\mathop{\mathrm{Coim}}}
\newcommand{\coim}{\mathop{\mathrm{coim}}}
\newcommand{\id}{\mathop{\mathrm{id}}}
\newcommand{\Gal}{\mathop{\mathrm{Gal}}}

\newtheorem{definition}{定義}[section]

\usepackage{aliascnt}

% \newaliastheorem{(環境とカウンターの名前)}{(元となるカウンターの名前)}{(表示される文字列)}
\newcommand*{\newaliastheorem}[3]{%
  \newaliascnt{#1}{#2}%
  \newtheorem{#1}[#1]{#3}%
  \aliascntresetthe{#1}%
  \expandafter\newcommand\csname #1autorefname\endcsname{#3}%
}
\newaliastheorem{proposition}{definition}{命題} 
\newaliastheorem{theorem}{definition}{定理}
\newaliastheorem{lemma}{definition}{補題}
\newaliastheorem{corollary}{definition}{系}
\newaliastheorem{example}{definition}{例}
\newaliastheorem{practice}{definition}{演習問題}

\newtheorem*{longproof}{証明}
\newtheorem*{answer}{解答}
\newtheorem*{supplement}{補足}
\newtheorem*{remark}{注意}
%----------

%----------
%古い記法を注意するパッケージ
\RequirePackage[l2tabu, orthodox]{nag}
%----------


% 定理環境(tcolorbox)
\usepackage{tcolorbox} %箱
\tcbuselibrary{breakable,skins,theorems}
\tcolorboxenvironment{definition}{
	blanker,breakable,
	left=3mm,right=3mm,
	top=2mm,bottom=2mm,
	before skip=15pt,after skip=20pt,
	borderline ={0.5pt}{0pt}{black}
}
\newtcolorbox{emptydefinition}{
	blanker,breakable,
	left=3mm,right=3mm,
	top=2mm,bottom=2mm,
	before skip=15pt,after skip=20pt,
	borderline ={0.5pt}{0pt}{black}
}
%----------
\tcolorboxenvironment{proposition}{
	blanker,breakable,
	left=3mm,right=3mm,
	top=3mm,bottom=3mm,
	before skip=15pt,after skip=15pt,
	borderline={0.5pt}{0pt}{black}
}
\newtcolorbox{emptyproposition}{
	blanker,breakable,
	left=3mm,right=3mm,
	top=3mm,bottom=3mm,
	before skip=15pt,after skip=15pt,
	borderline={0.5pt}{0pt}{black}
}
%----------
\tcolorboxenvironment{theorem}{
	blanker,breakable,
	left=3mm,right=3mm,
	top=3mm,bottom=3mm,
    sharp corners,boxrule=0.6pt,
	before skip=15pt,after skip=15pt,
	borderline={0.5pt}{0pt}{black},
    borderline={0.5pt}{1.5pt}{black}
}
\newtcolorbox{emptytheorem}{
	blanker,breakable,
	left=3mm,right=3mm,
	top=3mm,bottom=3mm,
    sharp corners,boxrule=0.6pt,
	before skip=15pt,after skip=15pt,
	borderline={0.5pt}{0pt}{black},
    borderline={0.5pt}{1.5pt}{black}
}
%----------
\tcolorboxenvironment{lemma}{
	blanker,breakable,
	left=3mm,right=3mm,
	top=3mm,bottom=3mm,
	before skip=15pt,after skip=15pt,
	borderline={0.5pt}{0pt}{black}
}
%----------
\tcolorboxenvironment{corollary}{
	blanker,breakable,
	left=3mm,right=3mm,
	top=3mm,bottom=3mm,
	before skip=15pt,after skip=15pt,
	borderline={1.0pt}{0pt}{black,dotted}
}
\newtcolorbox{emptycorollary}{
	blanker,breakable,
	left=3mm,right=3mm,
	top=3mm,bottom=3mm,
	before skip=15pt,after skip=15pt,
	borderline={1.0pt}{0pt}{black,dotted}
}
%----------
\tcolorboxenvironment{example}{
	blanker,breakable,
	left=3mm,right=3mm,
	top=3mm,bottom=3mm,
	before skip=15pt,after skip=15pt,
	borderline={0.5pt}{0pt}{black}
}
%----------
\tcolorboxenvironment{practice}{
	blanker,breakable,
	left=3mm,right=3mm,
	top=3mm,bottom=3mm,
	before skip=15pt,after skip=15pt,
	borderline={0.5pt}{0pt}{black}
}
%----------
\tcolorboxenvironment{proof}{
	blanker,breakable,
	left=3mm,right=3mm,
	top=2mm,bottom=2mm,
	before skip=15pt,after skip=20pt,
	% borderline west={1.5pt}{0pt}{black,dotted}
	borderline vertical={1pt}{0pt}{black,dotted}
	% borderline vertical={0.8pt}{0pt}{black,dotted,arrows={Square[scale=0.5]-Square[scale=0.5]}}
	}
%----------
\tcolorboxenvironment{supplement}{
	blanker,breakable,
	left=3mm,right=3mm,
	top=2mm,bottom=2mm,
	before skip=15pt,after skip=20pt,
	% borderline west={1.5pt}{0pt}{black,dotted}
	% borderline vertical={0.5pt}{0pt}{black,arrows = {Circle[scale=0.7]-Circle[scale=0.7]}}
	borderline vertical={0.5pt}{0pt}{black}
	% borderline vertical={0.5pt}{0pt}{black},
	% borderline north={0.5pt}{0pt}{white,arrows={Circle[black,scale=0.7]-Circle[black,scale=0.7]}}
	}
%----------
\tcolorboxenvironment{remark}{
	blanker,breakable,
	left=3mm,right=3mm,
	top=1mm,bottom=1mm,
	before skip=15pt,after skip=20pt,
	% borderline west={1.5pt}{0pt}{black,dotted}
	% borderline vertical={0.5pt}{0pt}{black,arrows = {Circle[scale=0.7]-Circle[scale=0.7]}}
	borderline vertical={0.5pt}{0pt}{black}
	% borderline vertical={0.5pt}{0pt}{black},
	% borderline north={0.5pt}{0pt}{white,arrows={Circle[black,scale=0.7]-Circle[black,scale=0.7]}}
	}
    
%---------------------
 

%----------
%ハイパーリンク
% 「%」は以降の内容を「改行コードも含めて」無視するコマンド
\usepackage[%
%  dvipdfmx,% 欧文ではコメントアウトする
luatex,%
pdfencoding=auto,%
 setpagesize=false,%
 bookmarks=true,%
 bookmarksdepth=tocdepth,%
 bookmarksnumbered=true,%
 colorlinks=false,%
 pdftitle={},%
 pdfsubject={},%
 pdfauthor={},%
 pdfkeywords={}%
]{hyperref}
%------------


%参照 参照するときに自動で環境名を含んで参照する
\usepackage[nameinlink]{cleveref}
\let\normalref\ref
\renewcommand{\ref}{\cref}
\crefname{definition}{定義}{定義}
\crefname{proposition}{命題}{命題}
\crefname{theorem}{定理}{定理}
\crefname{lemma}{補題}{補題}
\crefname{corollary}{系}{系}
\crefname{example}{例}{例}
\crefname{practice}{演習問題}{演習問題}
\crefname{equation}{式}{式} 
\crefname{chapter}{第}{第}
\creflabelformat{chapter}{#2#1章#3}
\crefname{section}{第}{第}
\creflabelformat{section}{#2#1節#3}
\crefname{subsection}{第}{第}
\creflabelformat{subsection}{#2#1小節#3}
%----------

%---------------------
%章跨ぎの参照が不具合を起こすための代わり
% \mylabl でラベル付け
\newcommand{\mylabel}[1]{
\label{#1}
\hypertarget{#1}{}
}
% \myref で環境名付きリンクをつける
\newcommand{\myref}[1]{
\hyperlink{#1}{\cref*{#1}}
}
%-----------------

\usepackage{autonum} %参照した数式にだけ番号を振る
% \usepackage{docmute} %ファイル分割
% \begin{document}

% %---------- 1章1節
% \input 1-01.tex
% %---------- 1章2節
% \input 1-02.tex
% % ---------- 1章3節
% \input 1-03.tex
% % ---------- 1章4節
% \input 1-04.tex
% % ---------- 1章5節
% \input 1-05.tex
% % ---------- 1章6節
% \input 1-06.tex
% %---------- 1章7節
% \input 1-07.tex
% % ---------- 1章8節
% \input 1-08.tex
% % ---------- 1章9節
% \input 1-09.tex
% % ---------- 1章10節
% \input 1-10.tex
% % ---------- 1章11節
% \input 1-11.tex
% % ---------- 1章12節
% \input 1-12.tex
% % ---------- 参考文献
% \input reference.tex
% \end{document}
% ----------



\usepackage{bxtexlogo}
\usepackage{amsthm}
\usepackage{amsmath}
\usepackage{bbm} %小文字の黒板文字
\usepackage{physics}
\usepackage{amsfonts}
\usepackage{graphicx}
\usepackage{mathtools}
\usepackage{enumitem}
\usepackage[margin=20truemm]{geometry}
\usepackage{textcomp}
\usepackage{bm}
\usepackage{mathrsfs}
\usepackage{latexsym}
\usepackage{amssymb}
\usepackage{algorithmic}
\usepackage{algorithm}
\usepackage{tikz}
\usetikzlibrary{arrows.meta}
\usetikzlibrary{math,matrix,backgrounds}
\usetikzlibrary{angles}
\usetikzlibrary{calc}


%----------
%日本語フォント
% \usepackage[deluxe]{otf} platex用 lualatexでは動かない

%----------
%欧文フォント
\usepackage[T1]{fontenc}

%----------
%文字色
\usepackage{color}

%----------
\setlength{\parindent}{2\zw} %インデントの設定

%----------
% %参照した数式にだけ番号を振る cleverrefと併用するとうまくいかない
% \mathtoolsset{showonlyrefs=true}
%----------

%----------
%集合の中線
\newcommand{\relmiddle}[1]{\mathrel{}\middle#1\mathrel{}}
% \middle| の代わりに \relmiddle| を付ける
\newcommand{\sgn}{\mathop{\mathrm{sgn}}} %置換sgn
\newcommand{\Int}{\mathop{\mathrm{Int}}} %位相空間の内部Int
\newcommand{\Ext}{\mathop{\mathrm{Ext}}} %位相空間の外部Ext
\newcommand{\Cl}{\mathop{\mathrm{Cl}}} %位相空間の閉包Cl
\newcommand{\supp}{\mathop{\mathrm{supp}}} %関数の台supp
\newcommand{\restrict}[2]{\left. #1 \right \vert_{#2}}%関数の制限 \restrict{f}{A} = f|_A
\newcommand{\Ker}{\mathop{\mathrm{Ker}}}
\newcommand{\Coker}{\mathop{\mathrm{Coker}}}
\newcommand{\coker}{\mathop{\mathrm{coker}}}
\newcommand{\Coim}{\mathop{\mathrm{Coim}}}
\newcommand{\coim}{\mathop{\mathrm{coim}}}
\newcommand{\id}{\mathop{\mathrm{id}}}
\newcommand{\Gal}{\mathop{\mathrm{Gal}}}

\newtheorem{definition}{定義}[section]

\usepackage{aliascnt}

% \newaliastheorem{(環境とカウンターの名前)}{(元となるカウンターの名前)}{(表示される文字列)}
\newcommand*{\newaliastheorem}[3]{%
  \newaliascnt{#1}{#2}%
  \newtheorem{#1}[#1]{#3}%
  \aliascntresetthe{#1}%
  \expandafter\newcommand\csname #1autorefname\endcsname{#3}%
}
\newaliastheorem{proposition}{definition}{命題} 
\newaliastheorem{theorem}{definition}{定理}
\newaliastheorem{lemma}{definition}{補題}
\newaliastheorem{corollary}{definition}{系}
\newaliastheorem{example}{definition}{例}
\newaliastheorem{practice}{definition}{演習問題}

\newtheorem*{longproof}{証明}
\newtheorem*{answer}{解答}
\newtheorem*{supplement}{補足}
\newtheorem*{remark}{注意}
%----------

%----------
%古い記法を注意するパッケージ
\RequirePackage[l2tabu, orthodox]{nag}
%----------


% 定理環境(tcolorbox)
\usepackage{tcolorbox} %箱
\tcbuselibrary{breakable,skins,theorems}
\tcolorboxenvironment{definition}{
	blanker,breakable,
	left=3mm,right=3mm,
	top=2mm,bottom=2mm,
	before skip=15pt,after skip=20pt,
	borderline ={0.5pt}{0pt}{black}
}
\newtcolorbox{emptydefinition}{
	blanker,breakable,
	left=3mm,right=3mm,
	top=2mm,bottom=2mm,
	before skip=15pt,after skip=20pt,
	borderline ={0.5pt}{0pt}{black}
}
%----------
\tcolorboxenvironment{proposition}{
	blanker,breakable,
	left=3mm,right=3mm,
	top=3mm,bottom=3mm,
	before skip=15pt,after skip=15pt,
	borderline={0.5pt}{0pt}{black}
}
\newtcolorbox{emptyproposition}{
	blanker,breakable,
	left=3mm,right=3mm,
	top=3mm,bottom=3mm,
	before skip=15pt,after skip=15pt,
	borderline={0.5pt}{0pt}{black}
}
%----------
\tcolorboxenvironment{theorem}{
	blanker,breakable,
	left=3mm,right=3mm,
	top=3mm,bottom=3mm,
    sharp corners,boxrule=0.6pt,
	before skip=15pt,after skip=15pt,
	borderline={0.5pt}{0pt}{black},
    borderline={0.5pt}{1.5pt}{black}
}
\newtcolorbox{emptytheorem}{
	blanker,breakable,
	left=3mm,right=3mm,
	top=3mm,bottom=3mm,
    sharp corners,boxrule=0.6pt,
	before skip=15pt,after skip=15pt,
	borderline={0.5pt}{0pt}{black},
    borderline={0.5pt}{1.5pt}{black}
}
%----------
\tcolorboxenvironment{lemma}{
	blanker,breakable,
	left=3mm,right=3mm,
	top=3mm,bottom=3mm,
	before skip=15pt,after skip=15pt,
	borderline={0.5pt}{0pt}{black}
}
%----------
\tcolorboxenvironment{corollary}{
	blanker,breakable,
	left=3mm,right=3mm,
	top=3mm,bottom=3mm,
	before skip=15pt,after skip=15pt,
	borderline={1.0pt}{0pt}{black,dotted}
}
\newtcolorbox{emptycorollary}{
	blanker,breakable,
	left=3mm,right=3mm,
	top=3mm,bottom=3mm,
	before skip=15pt,after skip=15pt,
	borderline={1.0pt}{0pt}{black,dotted}
}
%----------
\tcolorboxenvironment{example}{
	blanker,breakable,
	left=3mm,right=3mm,
	top=3mm,bottom=3mm,
	before skip=15pt,after skip=15pt,
	borderline={0.5pt}{0pt}{black}
}
%----------
\tcolorboxenvironment{practice}{
	blanker,breakable,
	left=3mm,right=3mm,
	top=3mm,bottom=3mm,
	before skip=15pt,after skip=15pt,
	borderline={0.5pt}{0pt}{black}
}
%----------
\tcolorboxenvironment{proof}{
	blanker,breakable,
	left=3mm,right=3mm,
	top=2mm,bottom=2mm,
	before skip=15pt,after skip=20pt,
	% borderline west={1.5pt}{0pt}{black,dotted}
	borderline vertical={1pt}{0pt}{black,dotted}
	% borderline vertical={0.8pt}{0pt}{black,dotted,arrows={Square[scale=0.5]-Square[scale=0.5]}}
	}
%----------
\tcolorboxenvironment{supplement}{
	blanker,breakable,
	left=3mm,right=3mm,
	top=2mm,bottom=2mm,
	before skip=15pt,after skip=20pt,
	% borderline west={1.5pt}{0pt}{black,dotted}
	% borderline vertical={0.5pt}{0pt}{black,arrows = {Circle[scale=0.7]-Circle[scale=0.7]}}
	borderline vertical={0.5pt}{0pt}{black}
	% borderline vertical={0.5pt}{0pt}{black},
	% borderline north={0.5pt}{0pt}{white,arrows={Circle[black,scale=0.7]-Circle[black,scale=0.7]}}
	}
%----------
\tcolorboxenvironment{remark}{
	blanker,breakable,
	left=3mm,right=3mm,
	top=1mm,bottom=1mm,
	before skip=15pt,after skip=20pt,
	% borderline west={1.5pt}{0pt}{black,dotted}
	% borderline vertical={0.5pt}{0pt}{black,arrows = {Circle[scale=0.7]-Circle[scale=0.7]}}
	borderline vertical={0.5pt}{0pt}{black}
	% borderline vertical={0.5pt}{0pt}{black},
	% borderline north={0.5pt}{0pt}{white,arrows={Circle[black,scale=0.7]-Circle[black,scale=0.7]}}
	}
    
%---------------------
 

%----------
%ハイパーリンク
% 「%」は以降の内容を「改行コードも含めて」無視するコマンド
\usepackage[%
%  dvipdfmx,% 欧文ではコメントアウトする
luatex,%
pdfencoding=auto,%
 setpagesize=false,%
 bookmarks=true,%
 bookmarksdepth=tocdepth,%
 bookmarksnumbered=true,%
 colorlinks=false,%
 pdftitle={},%
 pdfsubject={},%
 pdfauthor={},%
 pdfkeywords={}%
]{hyperref}
%------------


%参照 参照するときに自動で環境名を含んで参照する
\usepackage[nameinlink]{cleveref}
\let\normalref\ref
\renewcommand{\ref}{\cref}
\crefname{definition}{定義}{定義}
\crefname{proposition}{命題}{命題}
\crefname{theorem}{定理}{定理}
\crefname{lemma}{補題}{補題}
\crefname{corollary}{系}{系}
\crefname{example}{例}{例}
\crefname{practice}{演習問題}{演習問題}
\crefname{equation}{式}{式} 
\crefname{chapter}{第}{第}
\creflabelformat{chapter}{#2#1章#3}
\crefname{section}{第}{第}
\creflabelformat{section}{#2#1節#3}
\crefname{subsection}{第}{第}
\creflabelformat{subsection}{#2#1小節#3}
%----------

%---------------------
%章跨ぎの参照が不具合を起こすための代わり
% \mylabl でラベル付け
\newcommand{\mylabel}[1]{
\label{#1}
\hypertarget{#1}{}
}
% \myref で環境名付きリンクをつける
\newcommand{\myref}[1]{
\hyperlink{#1}{\cref*{#1}}
}
%-----------------

\usepackage{autonum} %参照した数式にだけ番号を振る
% \usepackage{docmute} %ファイル分割
% \begin{document}

% %\chapter{章のタイトル}
% \section{節のタイトル}
% no text

% \end{document}
%----------

%main.texには以下を書く
%----------
% \documentclass[
% 		book,
% 		head_space=20mm,
% 		foot_space=20mm,
% 		gutter=10mm,
% 		line_length=190mm,
%         openany
% ]{jlreq}
% 
%----------
%LuaLaTeXで実行する!!
%----------
%各章節には以下を書く. 1-03.texのような名前にする
%----------
% \documentclass[
% 		book,
% 		head_space=20mm,
% 		foot_space=20mm,
% 		gutter=10mm,
% 		line_length=190mm
% ]{jlreq}
% 
%----------
%LuaLaTeXで実行する!!
%----------
%各章節には以下を書く. 1-03.texのような名前にする
%----------
% \documentclass[
% 		book,
% 		head_space=20mm,
% 		foot_space=20mm,
% 		gutter=10mm,
% 		line_length=190mm
% ]{jlreq}
% \input {preamble.tex}
% \usepackage{docmute} %ファイル分割
% \begin{document}

% %\chapter{章のタイトル}
% \section{節のタイトル}
% no text

% \end{document}
%----------

%main.texには以下を書く
%----------
% \documentclass[
% 		book,
% 		head_space=20mm,
% 		foot_space=20mm,
% 		gutter=10mm,
% 		line_length=190mm,
%         openany
% ]{jlreq}
% \input {preamble.tex}
% \usepackage{docmute} %ファイル分割
% \begin{document}

% %---------- 1章1節
% \input 1-01.tex
% %---------- 1章2節
% \input 1-02.tex
% % ---------- 1章3節
% \input 1-03.tex
% % ---------- 1章4節
% \input 1-04.tex
% % ---------- 1章5節
% \input 1-05.tex
% % ---------- 1章6節
% \input 1-06.tex
% %---------- 1章7節
% \input 1-07.tex
% % ---------- 1章8節
% \input 1-08.tex
% % ---------- 1章9節
% \input 1-09.tex
% % ---------- 1章10節
% \input 1-10.tex
% % ---------- 1章11節
% \input 1-11.tex
% % ---------- 1章12節
% \input 1-12.tex
% % ---------- 参考文献
% \input reference.tex
% \end{document}
% ----------



\usepackage{bxtexlogo}
\usepackage{amsthm}
\usepackage{amsmath}
\usepackage{bbm} %小文字の黒板文字
\usepackage{physics}
\usepackage{amsfonts}
\usepackage{graphicx}
\usepackage{mathtools}
\usepackage{enumitem}
\usepackage[margin=20truemm]{geometry}
\usepackage{textcomp}
\usepackage{bm}
\usepackage{mathrsfs}
\usepackage{latexsym}
\usepackage{amssymb}
\usepackage{algorithmic}
\usepackage{algorithm}
\usepackage{tikz}
\usetikzlibrary{arrows.meta}
\usetikzlibrary{math,matrix,backgrounds}
\usetikzlibrary{angles}
\usetikzlibrary{calc}


%----------
%日本語フォント
% \usepackage[deluxe]{otf} platex用 lualatexでは動かない

%----------
%欧文フォント
\usepackage[T1]{fontenc}

%----------
%文字色
\usepackage{color}

%----------
\setlength{\parindent}{2\zw} %インデントの設定

%----------
% %参照した数式にだけ番号を振る cleverrefと併用するとうまくいかない
% \mathtoolsset{showonlyrefs=true}
%----------

%----------
%集合の中線
\newcommand{\relmiddle}[1]{\mathrel{}\middle#1\mathrel{}}
% \middle| の代わりに \relmiddle| を付ける
\newcommand{\sgn}{\mathop{\mathrm{sgn}}} %置換sgn
\newcommand{\Int}{\mathop{\mathrm{Int}}} %位相空間の内部Int
\newcommand{\Ext}{\mathop{\mathrm{Ext}}} %位相空間の外部Ext
\newcommand{\Cl}{\mathop{\mathrm{Cl}}} %位相空間の閉包Cl
\newcommand{\supp}{\mathop{\mathrm{supp}}} %関数の台supp
\newcommand{\restrict}[2]{\left. #1 \right \vert_{#2}}%関数の制限 \restrict{f}{A} = f|_A
\newcommand{\Ker}{\mathop{\mathrm{Ker}}}
\newcommand{\Coker}{\mathop{\mathrm{Coker}}}
\newcommand{\coker}{\mathop{\mathrm{coker}}}
\newcommand{\Coim}{\mathop{\mathrm{Coim}}}
\newcommand{\coim}{\mathop{\mathrm{coim}}}
\newcommand{\id}{\mathop{\mathrm{id}}}
\newcommand{\Gal}{\mathop{\mathrm{Gal}}}

\newtheorem{definition}{定義}[section]

\usepackage{aliascnt}

% \newaliastheorem{(環境とカウンターの名前)}{(元となるカウンターの名前)}{(表示される文字列)}
\newcommand*{\newaliastheorem}[3]{%
  \newaliascnt{#1}{#2}%
  \newtheorem{#1}[#1]{#3}%
  \aliascntresetthe{#1}%
  \expandafter\newcommand\csname #1autorefname\endcsname{#3}%
}
\newaliastheorem{proposition}{definition}{命題} 
\newaliastheorem{theorem}{definition}{定理}
\newaliastheorem{lemma}{definition}{補題}
\newaliastheorem{corollary}{definition}{系}
\newaliastheorem{example}{definition}{例}
\newaliastheorem{practice}{definition}{演習問題}

\newtheorem*{longproof}{証明}
\newtheorem*{answer}{解答}
\newtheorem*{supplement}{補足}
\newtheorem*{remark}{注意}
%----------

%----------
%古い記法を注意するパッケージ
\RequirePackage[l2tabu, orthodox]{nag}
%----------


% 定理環境(tcolorbox)
\usepackage{tcolorbox} %箱
\tcbuselibrary{breakable,skins,theorems}
\tcolorboxenvironment{definition}{
	blanker,breakable,
	left=3mm,right=3mm,
	top=2mm,bottom=2mm,
	before skip=15pt,after skip=20pt,
	borderline ={0.5pt}{0pt}{black}
}
\newtcolorbox{emptydefinition}{
	blanker,breakable,
	left=3mm,right=3mm,
	top=2mm,bottom=2mm,
	before skip=15pt,after skip=20pt,
	borderline ={0.5pt}{0pt}{black}
}
%----------
\tcolorboxenvironment{proposition}{
	blanker,breakable,
	left=3mm,right=3mm,
	top=3mm,bottom=3mm,
	before skip=15pt,after skip=15pt,
	borderline={0.5pt}{0pt}{black}
}
\newtcolorbox{emptyproposition}{
	blanker,breakable,
	left=3mm,right=3mm,
	top=3mm,bottom=3mm,
	before skip=15pt,after skip=15pt,
	borderline={0.5pt}{0pt}{black}
}
%----------
\tcolorboxenvironment{theorem}{
	blanker,breakable,
	left=3mm,right=3mm,
	top=3mm,bottom=3mm,
    sharp corners,boxrule=0.6pt,
	before skip=15pt,after skip=15pt,
	borderline={0.5pt}{0pt}{black},
    borderline={0.5pt}{1.5pt}{black}
}
\newtcolorbox{emptytheorem}{
	blanker,breakable,
	left=3mm,right=3mm,
	top=3mm,bottom=3mm,
    sharp corners,boxrule=0.6pt,
	before skip=15pt,after skip=15pt,
	borderline={0.5pt}{0pt}{black},
    borderline={0.5pt}{1.5pt}{black}
}
%----------
\tcolorboxenvironment{lemma}{
	blanker,breakable,
	left=3mm,right=3mm,
	top=3mm,bottom=3mm,
	before skip=15pt,after skip=15pt,
	borderline={0.5pt}{0pt}{black}
}
%----------
\tcolorboxenvironment{corollary}{
	blanker,breakable,
	left=3mm,right=3mm,
	top=3mm,bottom=3mm,
	before skip=15pt,after skip=15pt,
	borderline={1.0pt}{0pt}{black,dotted}
}
\newtcolorbox{emptycorollary}{
	blanker,breakable,
	left=3mm,right=3mm,
	top=3mm,bottom=3mm,
	before skip=15pt,after skip=15pt,
	borderline={1.0pt}{0pt}{black,dotted}
}
%----------
\tcolorboxenvironment{example}{
	blanker,breakable,
	left=3mm,right=3mm,
	top=3mm,bottom=3mm,
	before skip=15pt,after skip=15pt,
	borderline={0.5pt}{0pt}{black}
}
%----------
\tcolorboxenvironment{practice}{
	blanker,breakable,
	left=3mm,right=3mm,
	top=3mm,bottom=3mm,
	before skip=15pt,after skip=15pt,
	borderline={0.5pt}{0pt}{black}
}
%----------
\tcolorboxenvironment{proof}{
	blanker,breakable,
	left=3mm,right=3mm,
	top=2mm,bottom=2mm,
	before skip=15pt,after skip=20pt,
	% borderline west={1.5pt}{0pt}{black,dotted}
	borderline vertical={1pt}{0pt}{black,dotted}
	% borderline vertical={0.8pt}{0pt}{black,dotted,arrows={Square[scale=0.5]-Square[scale=0.5]}}
	}
%----------
\tcolorboxenvironment{supplement}{
	blanker,breakable,
	left=3mm,right=3mm,
	top=2mm,bottom=2mm,
	before skip=15pt,after skip=20pt,
	% borderline west={1.5pt}{0pt}{black,dotted}
	% borderline vertical={0.5pt}{0pt}{black,arrows = {Circle[scale=0.7]-Circle[scale=0.7]}}
	borderline vertical={0.5pt}{0pt}{black}
	% borderline vertical={0.5pt}{0pt}{black},
	% borderline north={0.5pt}{0pt}{white,arrows={Circle[black,scale=0.7]-Circle[black,scale=0.7]}}
	}
%----------
\tcolorboxenvironment{remark}{
	blanker,breakable,
	left=3mm,right=3mm,
	top=1mm,bottom=1mm,
	before skip=15pt,after skip=20pt,
	% borderline west={1.5pt}{0pt}{black,dotted}
	% borderline vertical={0.5pt}{0pt}{black,arrows = {Circle[scale=0.7]-Circle[scale=0.7]}}
	borderline vertical={0.5pt}{0pt}{black}
	% borderline vertical={0.5pt}{0pt}{black},
	% borderline north={0.5pt}{0pt}{white,arrows={Circle[black,scale=0.7]-Circle[black,scale=0.7]}}
	}
    
%---------------------
 

%----------
%ハイパーリンク
% 「%」は以降の内容を「改行コードも含めて」無視するコマンド
\usepackage[%
%  dvipdfmx,% 欧文ではコメントアウトする
luatex,%
pdfencoding=auto,%
 setpagesize=false,%
 bookmarks=true,%
 bookmarksdepth=tocdepth,%
 bookmarksnumbered=true,%
 colorlinks=false,%
 pdftitle={},%
 pdfsubject={},%
 pdfauthor={},%
 pdfkeywords={}%
]{hyperref}
%------------


%参照 参照するときに自動で環境名を含んで参照する
\usepackage[nameinlink]{cleveref}
\let\normalref\ref
\renewcommand{\ref}{\cref}
\crefname{definition}{定義}{定義}
\crefname{proposition}{命題}{命題}
\crefname{theorem}{定理}{定理}
\crefname{lemma}{補題}{補題}
\crefname{corollary}{系}{系}
\crefname{example}{例}{例}
\crefname{practice}{演習問題}{演習問題}
\crefname{equation}{式}{式} 
\crefname{chapter}{第}{第}
\creflabelformat{chapter}{#2#1章#3}
\crefname{section}{第}{第}
\creflabelformat{section}{#2#1節#3}
\crefname{subsection}{第}{第}
\creflabelformat{subsection}{#2#1小節#3}
%----------

%---------------------
%章跨ぎの参照が不具合を起こすための代わり
% \mylabl でラベル付け
\newcommand{\mylabel}[1]{
\label{#1}
\hypertarget{#1}{}
}
% \myref で環境名付きリンクをつける
\newcommand{\myref}[1]{
\hyperlink{#1}{\cref*{#1}}
}
%-----------------

\usepackage{autonum} %参照した数式にだけ番号を振る
% \usepackage{docmute} %ファイル分割
% \begin{document}

% %\chapter{章のタイトル}
% \section{節のタイトル}
% no text

% \end{document}
%----------

%main.texには以下を書く
%----------
% \documentclass[
% 		book,
% 		head_space=20mm,
% 		foot_space=20mm,
% 		gutter=10mm,
% 		line_length=190mm,
%         openany
% ]{jlreq}
% 
%----------
%LuaLaTeXで実行する!!
%----------
%各章節には以下を書く. 1-03.texのような名前にする
%----------
% \documentclass[
% 		book,
% 		head_space=20mm,
% 		foot_space=20mm,
% 		gutter=10mm,
% 		line_length=190mm
% ]{jlreq}
% \input {preamble.tex}
% \usepackage{docmute} %ファイル分割
% \begin{document}

% %\chapter{章のタイトル}
% \section{節のタイトル}
% no text

% \end{document}
%----------

%main.texには以下を書く
%----------
% \documentclass[
% 		book,
% 		head_space=20mm,
% 		foot_space=20mm,
% 		gutter=10mm,
% 		line_length=190mm,
%         openany
% ]{jlreq}
% \input {preamble.tex}
% \usepackage{docmute} %ファイル分割
% \begin{document}

% %---------- 1章1節
% \input 1-01.tex
% %---------- 1章2節
% \input 1-02.tex
% % ---------- 1章3節
% \input 1-03.tex
% % ---------- 1章4節
% \input 1-04.tex
% % ---------- 1章5節
% \input 1-05.tex
% % ---------- 1章6節
% \input 1-06.tex
% %---------- 1章7節
% \input 1-07.tex
% % ---------- 1章8節
% \input 1-08.tex
% % ---------- 1章9節
% \input 1-09.tex
% % ---------- 1章10節
% \input 1-10.tex
% % ---------- 1章11節
% \input 1-11.tex
% % ---------- 1章12節
% \input 1-12.tex
% % ---------- 参考文献
% \input reference.tex
% \end{document}
% ----------



\usepackage{bxtexlogo}
\usepackage{amsthm}
\usepackage{amsmath}
\usepackage{bbm} %小文字の黒板文字
\usepackage{physics}
\usepackage{amsfonts}
\usepackage{graphicx}
\usepackage{mathtools}
\usepackage{enumitem}
\usepackage[margin=20truemm]{geometry}
\usepackage{textcomp}
\usepackage{bm}
\usepackage{mathrsfs}
\usepackage{latexsym}
\usepackage{amssymb}
\usepackage{algorithmic}
\usepackage{algorithm}
\usepackage{tikz}
\usetikzlibrary{arrows.meta}
\usetikzlibrary{math,matrix,backgrounds}
\usetikzlibrary{angles}
\usetikzlibrary{calc}


%----------
%日本語フォント
% \usepackage[deluxe]{otf} platex用 lualatexでは動かない

%----------
%欧文フォント
\usepackage[T1]{fontenc}

%----------
%文字色
\usepackage{color}

%----------
\setlength{\parindent}{2\zw} %インデントの設定

%----------
% %参照した数式にだけ番号を振る cleverrefと併用するとうまくいかない
% \mathtoolsset{showonlyrefs=true}
%----------

%----------
%集合の中線
\newcommand{\relmiddle}[1]{\mathrel{}\middle#1\mathrel{}}
% \middle| の代わりに \relmiddle| を付ける
\newcommand{\sgn}{\mathop{\mathrm{sgn}}} %置換sgn
\newcommand{\Int}{\mathop{\mathrm{Int}}} %位相空間の内部Int
\newcommand{\Ext}{\mathop{\mathrm{Ext}}} %位相空間の外部Ext
\newcommand{\Cl}{\mathop{\mathrm{Cl}}} %位相空間の閉包Cl
\newcommand{\supp}{\mathop{\mathrm{supp}}} %関数の台supp
\newcommand{\restrict}[2]{\left. #1 \right \vert_{#2}}%関数の制限 \restrict{f}{A} = f|_A
\newcommand{\Ker}{\mathop{\mathrm{Ker}}}
\newcommand{\Coker}{\mathop{\mathrm{Coker}}}
\newcommand{\coker}{\mathop{\mathrm{coker}}}
\newcommand{\Coim}{\mathop{\mathrm{Coim}}}
\newcommand{\coim}{\mathop{\mathrm{coim}}}
\newcommand{\id}{\mathop{\mathrm{id}}}
\newcommand{\Gal}{\mathop{\mathrm{Gal}}}

\newtheorem{definition}{定義}[section]

\usepackage{aliascnt}

% \newaliastheorem{(環境とカウンターの名前)}{(元となるカウンターの名前)}{(表示される文字列)}
\newcommand*{\newaliastheorem}[3]{%
  \newaliascnt{#1}{#2}%
  \newtheorem{#1}[#1]{#3}%
  \aliascntresetthe{#1}%
  \expandafter\newcommand\csname #1autorefname\endcsname{#3}%
}
\newaliastheorem{proposition}{definition}{命題} 
\newaliastheorem{theorem}{definition}{定理}
\newaliastheorem{lemma}{definition}{補題}
\newaliastheorem{corollary}{definition}{系}
\newaliastheorem{example}{definition}{例}
\newaliastheorem{practice}{definition}{演習問題}

\newtheorem*{longproof}{証明}
\newtheorem*{answer}{解答}
\newtheorem*{supplement}{補足}
\newtheorem*{remark}{注意}
%----------

%----------
%古い記法を注意するパッケージ
\RequirePackage[l2tabu, orthodox]{nag}
%----------


% 定理環境(tcolorbox)
\usepackage{tcolorbox} %箱
\tcbuselibrary{breakable,skins,theorems}
\tcolorboxenvironment{definition}{
	blanker,breakable,
	left=3mm,right=3mm,
	top=2mm,bottom=2mm,
	before skip=15pt,after skip=20pt,
	borderline ={0.5pt}{0pt}{black}
}
\newtcolorbox{emptydefinition}{
	blanker,breakable,
	left=3mm,right=3mm,
	top=2mm,bottom=2mm,
	before skip=15pt,after skip=20pt,
	borderline ={0.5pt}{0pt}{black}
}
%----------
\tcolorboxenvironment{proposition}{
	blanker,breakable,
	left=3mm,right=3mm,
	top=3mm,bottom=3mm,
	before skip=15pt,after skip=15pt,
	borderline={0.5pt}{0pt}{black}
}
\newtcolorbox{emptyproposition}{
	blanker,breakable,
	left=3mm,right=3mm,
	top=3mm,bottom=3mm,
	before skip=15pt,after skip=15pt,
	borderline={0.5pt}{0pt}{black}
}
%----------
\tcolorboxenvironment{theorem}{
	blanker,breakable,
	left=3mm,right=3mm,
	top=3mm,bottom=3mm,
    sharp corners,boxrule=0.6pt,
	before skip=15pt,after skip=15pt,
	borderline={0.5pt}{0pt}{black},
    borderline={0.5pt}{1.5pt}{black}
}
\newtcolorbox{emptytheorem}{
	blanker,breakable,
	left=3mm,right=3mm,
	top=3mm,bottom=3mm,
    sharp corners,boxrule=0.6pt,
	before skip=15pt,after skip=15pt,
	borderline={0.5pt}{0pt}{black},
    borderline={0.5pt}{1.5pt}{black}
}
%----------
\tcolorboxenvironment{lemma}{
	blanker,breakable,
	left=3mm,right=3mm,
	top=3mm,bottom=3mm,
	before skip=15pt,after skip=15pt,
	borderline={0.5pt}{0pt}{black}
}
%----------
\tcolorboxenvironment{corollary}{
	blanker,breakable,
	left=3mm,right=3mm,
	top=3mm,bottom=3mm,
	before skip=15pt,after skip=15pt,
	borderline={1.0pt}{0pt}{black,dotted}
}
\newtcolorbox{emptycorollary}{
	blanker,breakable,
	left=3mm,right=3mm,
	top=3mm,bottom=3mm,
	before skip=15pt,after skip=15pt,
	borderline={1.0pt}{0pt}{black,dotted}
}
%----------
\tcolorboxenvironment{example}{
	blanker,breakable,
	left=3mm,right=3mm,
	top=3mm,bottom=3mm,
	before skip=15pt,after skip=15pt,
	borderline={0.5pt}{0pt}{black}
}
%----------
\tcolorboxenvironment{practice}{
	blanker,breakable,
	left=3mm,right=3mm,
	top=3mm,bottom=3mm,
	before skip=15pt,after skip=15pt,
	borderline={0.5pt}{0pt}{black}
}
%----------
\tcolorboxenvironment{proof}{
	blanker,breakable,
	left=3mm,right=3mm,
	top=2mm,bottom=2mm,
	before skip=15pt,after skip=20pt,
	% borderline west={1.5pt}{0pt}{black,dotted}
	borderline vertical={1pt}{0pt}{black,dotted}
	% borderline vertical={0.8pt}{0pt}{black,dotted,arrows={Square[scale=0.5]-Square[scale=0.5]}}
	}
%----------
\tcolorboxenvironment{supplement}{
	blanker,breakable,
	left=3mm,right=3mm,
	top=2mm,bottom=2mm,
	before skip=15pt,after skip=20pt,
	% borderline west={1.5pt}{0pt}{black,dotted}
	% borderline vertical={0.5pt}{0pt}{black,arrows = {Circle[scale=0.7]-Circle[scale=0.7]}}
	borderline vertical={0.5pt}{0pt}{black}
	% borderline vertical={0.5pt}{0pt}{black},
	% borderline north={0.5pt}{0pt}{white,arrows={Circle[black,scale=0.7]-Circle[black,scale=0.7]}}
	}
%----------
\tcolorboxenvironment{remark}{
	blanker,breakable,
	left=3mm,right=3mm,
	top=1mm,bottom=1mm,
	before skip=15pt,after skip=20pt,
	% borderline west={1.5pt}{0pt}{black,dotted}
	% borderline vertical={0.5pt}{0pt}{black,arrows = {Circle[scale=0.7]-Circle[scale=0.7]}}
	borderline vertical={0.5pt}{0pt}{black}
	% borderline vertical={0.5pt}{0pt}{black},
	% borderline north={0.5pt}{0pt}{white,arrows={Circle[black,scale=0.7]-Circle[black,scale=0.7]}}
	}
    
%---------------------
 

%----------
%ハイパーリンク
% 「%」は以降の内容を「改行コードも含めて」無視するコマンド
\usepackage[%
%  dvipdfmx,% 欧文ではコメントアウトする
luatex,%
pdfencoding=auto,%
 setpagesize=false,%
 bookmarks=true,%
 bookmarksdepth=tocdepth,%
 bookmarksnumbered=true,%
 colorlinks=false,%
 pdftitle={},%
 pdfsubject={},%
 pdfauthor={},%
 pdfkeywords={}%
]{hyperref}
%------------


%参照 参照するときに自動で環境名を含んで参照する
\usepackage[nameinlink]{cleveref}
\let\normalref\ref
\renewcommand{\ref}{\cref}
\crefname{definition}{定義}{定義}
\crefname{proposition}{命題}{命題}
\crefname{theorem}{定理}{定理}
\crefname{lemma}{補題}{補題}
\crefname{corollary}{系}{系}
\crefname{example}{例}{例}
\crefname{practice}{演習問題}{演習問題}
\crefname{equation}{式}{式} 
\crefname{chapter}{第}{第}
\creflabelformat{chapter}{#2#1章#3}
\crefname{section}{第}{第}
\creflabelformat{section}{#2#1節#3}
\crefname{subsection}{第}{第}
\creflabelformat{subsection}{#2#1小節#3}
%----------

%---------------------
%章跨ぎの参照が不具合を起こすための代わり
% \mylabl でラベル付け
\newcommand{\mylabel}[1]{
\label{#1}
\hypertarget{#1}{}
}
% \myref で環境名付きリンクをつける
\newcommand{\myref}[1]{
\hyperlink{#1}{\cref*{#1}}
}
%-----------------

\usepackage{autonum} %参照した数式にだけ番号を振る
% \usepackage{docmute} %ファイル分割
% \begin{document}

% %---------- 1章1節
% \input 1-01.tex
% %---------- 1章2節
% \input 1-02.tex
% % ---------- 1章3節
% \input 1-03.tex
% % ---------- 1章4節
% \input 1-04.tex
% % ---------- 1章5節
% \input 1-05.tex
% % ---------- 1章6節
% \input 1-06.tex
% %---------- 1章7節
% \input 1-07.tex
% % ---------- 1章8節
% \input 1-08.tex
% % ---------- 1章9節
% \input 1-09.tex
% % ---------- 1章10節
% \input 1-10.tex
% % ---------- 1章11節
% \input 1-11.tex
% % ---------- 1章12節
% \input 1-12.tex
% % ---------- 参考文献
% \input reference.tex
% \end{document}
% ----------



\usepackage{bxtexlogo}
\usepackage{amsthm}
\usepackage{amsmath}
\usepackage{bbm} %小文字の黒板文字
\usepackage{physics}
\usepackage{amsfonts}
\usepackage{graphicx}
\usepackage{mathtools}
\usepackage{enumitem}
\usepackage[margin=20truemm]{geometry}
\usepackage{textcomp}
\usepackage{bm}
\usepackage{mathrsfs}
\usepackage{latexsym}
\usepackage{amssymb}
\usepackage{algorithmic}
\usepackage{algorithm}
\usepackage{tikz}
\usetikzlibrary{arrows.meta}
\usetikzlibrary{math,matrix,backgrounds}
\usetikzlibrary{angles}
\usetikzlibrary{calc}


%----------
%日本語フォント
% \usepackage[deluxe]{otf} platex用 lualatexでは動かない

%----------
%欧文フォント
\usepackage[T1]{fontenc}

%----------
%文字色
\usepackage{color}

%----------
\setlength{\parindent}{2\zw} %インデントの設定

%----------
% %参照した数式にだけ番号を振る cleverrefと併用するとうまくいかない
% \mathtoolsset{showonlyrefs=true}
%----------

%----------
%集合の中線
\newcommand{\relmiddle}[1]{\mathrel{}\middle#1\mathrel{}}
% \middle| の代わりに \relmiddle| を付ける
\newcommand{\sgn}{\mathop{\mathrm{sgn}}} %置換sgn
\newcommand{\Int}{\mathop{\mathrm{Int}}} %位相空間の内部Int
\newcommand{\Ext}{\mathop{\mathrm{Ext}}} %位相空間の外部Ext
\newcommand{\Cl}{\mathop{\mathrm{Cl}}} %位相空間の閉包Cl
\newcommand{\supp}{\mathop{\mathrm{supp}}} %関数の台supp
\newcommand{\restrict}[2]{\left. #1 \right \vert_{#2}}%関数の制限 \restrict{f}{A} = f|_A
\newcommand{\Ker}{\mathop{\mathrm{Ker}}}
\newcommand{\Coker}{\mathop{\mathrm{Coker}}}
\newcommand{\coker}{\mathop{\mathrm{coker}}}
\newcommand{\Coim}{\mathop{\mathrm{Coim}}}
\newcommand{\coim}{\mathop{\mathrm{coim}}}
\newcommand{\id}{\mathop{\mathrm{id}}}
\newcommand{\Gal}{\mathop{\mathrm{Gal}}}

\newtheorem{definition}{定義}[section]

\usepackage{aliascnt}

% \newaliastheorem{(環境とカウンターの名前)}{(元となるカウンターの名前)}{(表示される文字列)}
\newcommand*{\newaliastheorem}[3]{%
  \newaliascnt{#1}{#2}%
  \newtheorem{#1}[#1]{#3}%
  \aliascntresetthe{#1}%
  \expandafter\newcommand\csname #1autorefname\endcsname{#3}%
}
\newaliastheorem{proposition}{definition}{命題} 
\newaliastheorem{theorem}{definition}{定理}
\newaliastheorem{lemma}{definition}{補題}
\newaliastheorem{corollary}{definition}{系}
\newaliastheorem{example}{definition}{例}
\newaliastheorem{practice}{definition}{演習問題}

\newtheorem*{longproof}{証明}
\newtheorem*{answer}{解答}
\newtheorem*{supplement}{補足}
\newtheorem*{remark}{注意}
%----------

%----------
%古い記法を注意するパッケージ
\RequirePackage[l2tabu, orthodox]{nag}
%----------


% 定理環境(tcolorbox)
\usepackage{tcolorbox} %箱
\tcbuselibrary{breakable,skins,theorems}
\tcolorboxenvironment{definition}{
	blanker,breakable,
	left=3mm,right=3mm,
	top=2mm,bottom=2mm,
	before skip=15pt,after skip=20pt,
	borderline ={0.5pt}{0pt}{black}
}
\newtcolorbox{emptydefinition}{
	blanker,breakable,
	left=3mm,right=3mm,
	top=2mm,bottom=2mm,
	before skip=15pt,after skip=20pt,
	borderline ={0.5pt}{0pt}{black}
}
%----------
\tcolorboxenvironment{proposition}{
	blanker,breakable,
	left=3mm,right=3mm,
	top=3mm,bottom=3mm,
	before skip=15pt,after skip=15pt,
	borderline={0.5pt}{0pt}{black}
}
\newtcolorbox{emptyproposition}{
	blanker,breakable,
	left=3mm,right=3mm,
	top=3mm,bottom=3mm,
	before skip=15pt,after skip=15pt,
	borderline={0.5pt}{0pt}{black}
}
%----------
\tcolorboxenvironment{theorem}{
	blanker,breakable,
	left=3mm,right=3mm,
	top=3mm,bottom=3mm,
    sharp corners,boxrule=0.6pt,
	before skip=15pt,after skip=15pt,
	borderline={0.5pt}{0pt}{black},
    borderline={0.5pt}{1.5pt}{black}
}
\newtcolorbox{emptytheorem}{
	blanker,breakable,
	left=3mm,right=3mm,
	top=3mm,bottom=3mm,
    sharp corners,boxrule=0.6pt,
	before skip=15pt,after skip=15pt,
	borderline={0.5pt}{0pt}{black},
    borderline={0.5pt}{1.5pt}{black}
}
%----------
\tcolorboxenvironment{lemma}{
	blanker,breakable,
	left=3mm,right=3mm,
	top=3mm,bottom=3mm,
	before skip=15pt,after skip=15pt,
	borderline={0.5pt}{0pt}{black}
}
%----------
\tcolorboxenvironment{corollary}{
	blanker,breakable,
	left=3mm,right=3mm,
	top=3mm,bottom=3mm,
	before skip=15pt,after skip=15pt,
	borderline={1.0pt}{0pt}{black,dotted}
}
\newtcolorbox{emptycorollary}{
	blanker,breakable,
	left=3mm,right=3mm,
	top=3mm,bottom=3mm,
	before skip=15pt,after skip=15pt,
	borderline={1.0pt}{0pt}{black,dotted}
}
%----------
\tcolorboxenvironment{example}{
	blanker,breakable,
	left=3mm,right=3mm,
	top=3mm,bottom=3mm,
	before skip=15pt,after skip=15pt,
	borderline={0.5pt}{0pt}{black}
}
%----------
\tcolorboxenvironment{practice}{
	blanker,breakable,
	left=3mm,right=3mm,
	top=3mm,bottom=3mm,
	before skip=15pt,after skip=15pt,
	borderline={0.5pt}{0pt}{black}
}
%----------
\tcolorboxenvironment{proof}{
	blanker,breakable,
	left=3mm,right=3mm,
	top=2mm,bottom=2mm,
	before skip=15pt,after skip=20pt,
	% borderline west={1.5pt}{0pt}{black,dotted}
	borderline vertical={1pt}{0pt}{black,dotted}
	% borderline vertical={0.8pt}{0pt}{black,dotted,arrows={Square[scale=0.5]-Square[scale=0.5]}}
	}
%----------
\tcolorboxenvironment{supplement}{
	blanker,breakable,
	left=3mm,right=3mm,
	top=2mm,bottom=2mm,
	before skip=15pt,after skip=20pt,
	% borderline west={1.5pt}{0pt}{black,dotted}
	% borderline vertical={0.5pt}{0pt}{black,arrows = {Circle[scale=0.7]-Circle[scale=0.7]}}
	borderline vertical={0.5pt}{0pt}{black}
	% borderline vertical={0.5pt}{0pt}{black},
	% borderline north={0.5pt}{0pt}{white,arrows={Circle[black,scale=0.7]-Circle[black,scale=0.7]}}
	}
%----------
\tcolorboxenvironment{remark}{
	blanker,breakable,
	left=3mm,right=3mm,
	top=1mm,bottom=1mm,
	before skip=15pt,after skip=20pt,
	% borderline west={1.5pt}{0pt}{black,dotted}
	% borderline vertical={0.5pt}{0pt}{black,arrows = {Circle[scale=0.7]-Circle[scale=0.7]}}
	borderline vertical={0.5pt}{0pt}{black}
	% borderline vertical={0.5pt}{0pt}{black},
	% borderline north={0.5pt}{0pt}{white,arrows={Circle[black,scale=0.7]-Circle[black,scale=0.7]}}
	}
    
%---------------------
 

%----------
%ハイパーリンク
% 「%」は以降の内容を「改行コードも含めて」無視するコマンド
\usepackage[%
%  dvipdfmx,% 欧文ではコメントアウトする
luatex,%
pdfencoding=auto,%
 setpagesize=false,%
 bookmarks=true,%
 bookmarksdepth=tocdepth,%
 bookmarksnumbered=true,%
 colorlinks=false,%
 pdftitle={},%
 pdfsubject={},%
 pdfauthor={},%
 pdfkeywords={}%
]{hyperref}
%------------


%参照 参照するときに自動で環境名を含んで参照する
\usepackage[nameinlink]{cleveref}
\let\normalref\ref
\renewcommand{\ref}{\cref}
\crefname{definition}{定義}{定義}
\crefname{proposition}{命題}{命題}
\crefname{theorem}{定理}{定理}
\crefname{lemma}{補題}{補題}
\crefname{corollary}{系}{系}
\crefname{example}{例}{例}
\crefname{practice}{演習問題}{演習問題}
\crefname{equation}{式}{式} 
\crefname{chapter}{第}{第}
\creflabelformat{chapter}{#2#1章#3}
\crefname{section}{第}{第}
\creflabelformat{section}{#2#1節#3}
\crefname{subsection}{第}{第}
\creflabelformat{subsection}{#2#1小節#3}
%----------

%---------------------
%章跨ぎの参照が不具合を起こすための代わり
% \mylabl でラベル付け
\newcommand{\mylabel}[1]{
\label{#1}
\hypertarget{#1}{}
}
% \myref で環境名付きリンクをつける
\newcommand{\myref}[1]{
\hyperlink{#1}{\cref*{#1}}
}
%-----------------

\usepackage{autonum} %参照した数式にだけ番号を振る
% \usepackage{docmute} %ファイル分割
% \begin{document}

% %---------- 1章1節
% \input 1-01.tex
% %---------- 1章2節
% \input 1-02.tex
% % ---------- 1章3節
% \input 1-03.tex
% % ---------- 1章4節
% \input 1-04.tex
% % ---------- 1章5節
% \input 1-05.tex
% % ---------- 1章6節
% \input 1-06.tex
% %---------- 1章7節
% \input 1-07.tex
% % ---------- 1章8節
% \input 1-08.tex
% % ---------- 1章9節
% \input 1-09.tex
% % ---------- 1章10節
% \input 1-10.tex
% % ---------- 1章11節
% \input 1-11.tex
% % ---------- 1章12節
% \input 1-12.tex
% % ---------- 参考文献
% \input reference.tex
% \end{document}
% ----------



\usepackage{bxtexlogo}
\usepackage{amsthm}
\usepackage{amsmath}
\usepackage{bbm} %小文字の黒板文字
\usepackage{physics}
\usepackage{amsfonts}
\usepackage{graphicx}
\usepackage{mathtools}
\usepackage{enumitem}
\usepackage[margin=20truemm]{geometry}
\usepackage{textcomp}
\usepackage{bm}
\usepackage{mathrsfs}
\usepackage{latexsym}
\usepackage{amssymb}
\usepackage{algorithmic}
\usepackage{algorithm}
\usepackage{tikz}
\usetikzlibrary{arrows.meta}
\usetikzlibrary{math,matrix,backgrounds}
\usetikzlibrary{angles}
\usetikzlibrary{calc}


%----------
%日本語フォント
% \usepackage[deluxe]{otf} platex用 lualatexでは動かない

%----------
%欧文フォント
\usepackage[T1]{fontenc}

%----------
%文字色
\usepackage{color}

%----------
\setlength{\parindent}{2\zw} %インデントの設定

%----------
% %参照した数式にだけ番号を振る cleverrefと併用するとうまくいかない
% \mathtoolsset{showonlyrefs=true}
%----------

%----------
%集合の中線
\newcommand{\relmiddle}[1]{\mathrel{}\middle#1\mathrel{}}
% \middle| の代わりに \relmiddle| を付ける
\newcommand{\sgn}{\mathop{\mathrm{sgn}}} %置換sgn
\newcommand{\Int}{\mathop{\mathrm{Int}}} %位相空間の内部Int
\newcommand{\Ext}{\mathop{\mathrm{Ext}}} %位相空間の外部Ext
\newcommand{\Cl}{\mathop{\mathrm{Cl}}} %位相空間の閉包Cl
\newcommand{\supp}{\mathop{\mathrm{supp}}} %関数の台supp
\newcommand{\restrict}[2]{\left. #1 \right \vert_{#2}}%関数の制限 \restrict{f}{A} = f|_A
\newcommand{\Ker}{\mathop{\mathrm{Ker}}}
\newcommand{\Coker}{\mathop{\mathrm{Coker}}}
\newcommand{\coker}{\mathop{\mathrm{coker}}}
\newcommand{\Coim}{\mathop{\mathrm{Coim}}}
\newcommand{\coim}{\mathop{\mathrm{coim}}}
\newcommand{\id}{\mathop{\mathrm{id}}}
\newcommand{\Gal}{\mathop{\mathrm{Gal}}}

\newtheorem{definition}{定義}[section]

\usepackage{aliascnt}

% \newaliastheorem{(環境とカウンターの名前)}{(元となるカウンターの名前)}{(表示される文字列)}
\newcommand*{\newaliastheorem}[3]{%
  \newaliascnt{#1}{#2}%
  \newtheorem{#1}[#1]{#3}%
  \aliascntresetthe{#1}%
  \expandafter\newcommand\csname #1autorefname\endcsname{#3}%
}
\newaliastheorem{proposition}{definition}{命題} 
\newaliastheorem{theorem}{definition}{定理}
\newaliastheorem{lemma}{definition}{補題}
\newaliastheorem{corollary}{definition}{系}
\newaliastheorem{example}{definition}{例}
\newaliastheorem{practice}{definition}{演習問題}

\newtheorem*{longproof}{証明}
\newtheorem*{answer}{解答}
\newtheorem*{supplement}{補足}
\newtheorem*{remark}{注意}
%----------

%----------
%古い記法を注意するパッケージ
\RequirePackage[l2tabu, orthodox]{nag}
%----------


% 定理環境(tcolorbox)
\usepackage{tcolorbox} %箱
\tcbuselibrary{breakable,skins,theorems}
\tcolorboxenvironment{definition}{
	blanker,breakable,
	left=3mm,right=3mm,
	top=2mm,bottom=2mm,
	before skip=15pt,after skip=20pt,
	borderline ={0.5pt}{0pt}{black}
}
\newtcolorbox{emptydefinition}{
	blanker,breakable,
	left=3mm,right=3mm,
	top=2mm,bottom=2mm,
	before skip=15pt,after skip=20pt,
	borderline ={0.5pt}{0pt}{black}
}
%----------
\tcolorboxenvironment{proposition}{
	blanker,breakable,
	left=3mm,right=3mm,
	top=3mm,bottom=3mm,
	before skip=15pt,after skip=15pt,
	borderline={0.5pt}{0pt}{black}
}
\newtcolorbox{emptyproposition}{
	blanker,breakable,
	left=3mm,right=3mm,
	top=3mm,bottom=3mm,
	before skip=15pt,after skip=15pt,
	borderline={0.5pt}{0pt}{black}
}
%----------
\tcolorboxenvironment{theorem}{
	blanker,breakable,
	left=3mm,right=3mm,
	top=3mm,bottom=3mm,
    sharp corners,boxrule=0.6pt,
	before skip=15pt,after skip=15pt,
	borderline={0.5pt}{0pt}{black},
    borderline={0.5pt}{1.5pt}{black}
}
\newtcolorbox{emptytheorem}{
	blanker,breakable,
	left=3mm,right=3mm,
	top=3mm,bottom=3mm,
    sharp corners,boxrule=0.6pt,
	before skip=15pt,after skip=15pt,
	borderline={0.5pt}{0pt}{black},
    borderline={0.5pt}{1.5pt}{black}
}
%----------
\tcolorboxenvironment{lemma}{
	blanker,breakable,
	left=3mm,right=3mm,
	top=3mm,bottom=3mm,
	before skip=15pt,after skip=15pt,
	borderline={0.5pt}{0pt}{black}
}
%----------
\tcolorboxenvironment{corollary}{
	blanker,breakable,
	left=3mm,right=3mm,
	top=3mm,bottom=3mm,
	before skip=15pt,after skip=15pt,
	borderline={1.0pt}{0pt}{black,dotted}
}
\newtcolorbox{emptycorollary}{
	blanker,breakable,
	left=3mm,right=3mm,
	top=3mm,bottom=3mm,
	before skip=15pt,after skip=15pt,
	borderline={1.0pt}{0pt}{black,dotted}
}
%----------
\tcolorboxenvironment{example}{
	blanker,breakable,
	left=3mm,right=3mm,
	top=3mm,bottom=3mm,
	before skip=15pt,after skip=15pt,
	borderline={0.5pt}{0pt}{black}
}
%----------
\tcolorboxenvironment{practice}{
	blanker,breakable,
	left=3mm,right=3mm,
	top=3mm,bottom=3mm,
	before skip=15pt,after skip=15pt,
	borderline={0.5pt}{0pt}{black}
}
%----------
\tcolorboxenvironment{proof}{
	blanker,breakable,
	left=3mm,right=3mm,
	top=2mm,bottom=2mm,
	before skip=15pt,after skip=20pt,
	% borderline west={1.5pt}{0pt}{black,dotted}
	borderline vertical={1pt}{0pt}{black,dotted}
	% borderline vertical={0.8pt}{0pt}{black,dotted,arrows={Square[scale=0.5]-Square[scale=0.5]}}
	}
%----------
\tcolorboxenvironment{supplement}{
	blanker,breakable,
	left=3mm,right=3mm,
	top=2mm,bottom=2mm,
	before skip=15pt,after skip=20pt,
	% borderline west={1.5pt}{0pt}{black,dotted}
	% borderline vertical={0.5pt}{0pt}{black,arrows = {Circle[scale=0.7]-Circle[scale=0.7]}}
	borderline vertical={0.5pt}{0pt}{black}
	% borderline vertical={0.5pt}{0pt}{black},
	% borderline north={0.5pt}{0pt}{white,arrows={Circle[black,scale=0.7]-Circle[black,scale=0.7]}}
	}
%----------
\tcolorboxenvironment{remark}{
	blanker,breakable,
	left=3mm,right=3mm,
	top=1mm,bottom=1mm,
	before skip=15pt,after skip=20pt,
	% borderline west={1.5pt}{0pt}{black,dotted}
	% borderline vertical={0.5pt}{0pt}{black,arrows = {Circle[scale=0.7]-Circle[scale=0.7]}}
	borderline vertical={0.5pt}{0pt}{black}
	% borderline vertical={0.5pt}{0pt}{black},
	% borderline north={0.5pt}{0pt}{white,arrows={Circle[black,scale=0.7]-Circle[black,scale=0.7]}}
	}
    
%---------------------
 

%----------
%ハイパーリンク
% 「%」は以降の内容を「改行コードも含めて」無視するコマンド
\usepackage[%
%  dvipdfmx,% 欧文ではコメントアウトする
luatex,%
pdfencoding=auto,%
 setpagesize=false,%
 bookmarks=true,%
 bookmarksdepth=tocdepth,%
 bookmarksnumbered=true,%
 colorlinks=false,%
 pdftitle={},%
 pdfsubject={},%
 pdfauthor={},%
 pdfkeywords={}%
]{hyperref}
%------------


%参照 参照するときに自動で環境名を含んで参照する
\usepackage[nameinlink]{cleveref}
\let\normalref\ref
\renewcommand{\ref}{\cref}
\crefname{definition}{定義}{定義}
\crefname{proposition}{命題}{命題}
\crefname{theorem}{定理}{定理}
\crefname{lemma}{補題}{補題}
\crefname{corollary}{系}{系}
\crefname{example}{例}{例}
\crefname{practice}{演習問題}{演習問題}
\crefname{equation}{式}{式} 
\crefname{chapter}{第}{第}
\creflabelformat{chapter}{#2#1章#3}
\crefname{section}{第}{第}
\creflabelformat{section}{#2#1節#3}
\crefname{subsection}{第}{第}
\creflabelformat{subsection}{#2#1小節#3}
%----------

%---------------------
%章跨ぎの参照が不具合を起こすための代わり
% \mylabl でラベル付け
\newcommand{\mylabel}[1]{
\label{#1}
\hypertarget{#1}{}
}
% \myref で環境名付きリンクをつける
\newcommand{\myref}[1]{
\hyperlink{#1}{\cref*{#1}}
}
%-----------------

\usepackage{autonum} %参照した数式にだけ番号を振る
\usepackage{docmute} %ファイル分割
\begin{document}

%\chapter{章のタイトル}
\section{H21数学A}
\fbox{1}
(1)
\begin{align}
	\lim\limits_{n\to \infty}\frac{f\left( \frac{1}{n} \right)-f(0)}{\frac{1}{n}-0}=\lim\limits_{n\to \infty}nf\left( \frac{1}{n} \right)=f'(0)
\end{align}であるから$-f'(0)/2>0$に対してある$N \in \mathbb{N}$が存在して$n>N$なら$nf(\frac{1}{n})<f'(0)+(-f'(0)/2)=f'(0)/2$である.

(2)$\pi>x>0$で$\frac{1}{1+x}<1$より$\log(1+x)=\int_0^1\frac{1}{1+t}dt<\int_0^11dt=t$である.したがって$\log(1+x)<x$である.またテイラーの定理から$\cos x=1-\frac{x^2}{2}+\frac{x^4}{4!}+R(x)$である.$R(x)=\frac{(\cos^{(5)}s)}{5!}x^5\quad(0<s<x<\pi)$である.$(\cos^{(5)}s)=-\sin s<0$であるから$R(x)<0$である.したがって$\cos x<1-\frac{x^2}{2}+\frac{x^4}{4!}$である.

以上より \begin{align}
	\log(\cos (\frac{1}{\sqrt{n}}))&\le\log(1-\frac{1}{2n}+\frac{1}{24n^2}) \le-\frac{1}{2n}+\frac{1}{24n^2}<-\frac{1}{n}\\
\end{align}であるから発散する.

\fbox{2}
(1)略

(2)$\sigma(u)=u-2\frac{(u,u)}{(u,u)}u=-u$より$-1$は固有値である.
$\sigma(x)=-x$とすると,$x-2\frac{(x,u)}{(u,u)}u=-x$であるから$x=\frac{(x,u)}{(u,u)}u$である.すなわち$x \in \mathrm{spam(u)} $である.よって$W(-1)=\mathrm{spam(u)}$である.

(3)$\sigma\circ f(u)=f \circ\sigma(u)=f(-u)=-f(u)$であるから$f(u)\in W(-1)=\mathrm{spam(u)}$である.したがって$u$は$f$の固有ベクトルである.

(4)$\sigma(e_1)=e_1-2\frac{(e_1,u)}{(u,u)}u=e_1-\frac{2}{3}u=\frac{1}{3}e_1+\frac{2}{3}e_3-\frac{2}{3}e_4,\sigma(e_2)=e_2,\sigma(e_3)=e_3+\frac{2}{3}u=\frac{2}{3}e_1+\frac{1}{3}e_3+\frac{2}{3}e_4 ,\sigma(e_4)=e_4-\frac{2}{3}u=-\frac{2}{3}e_1+\frac{2}{3}e_3+\frac{1}{3}e_4$である.
よって表現行列は$\begin{pmatrix}
	\frac{1}{3} & 0 & \frac{2}{3} & -\frac{2}{3} \\
	0 & 1 & 0 & 0 \\
	\frac{2}{3} & 0 & \frac{1}{3} & \frac{2}{3} \\
	\frac{-2}{3} & 0 & \frac{2}{3} & \frac{1}{3}
\end{pmatrix}$である.

\fbox{3}
(1)ハウスドルフ空間$(X,\mathcal{O})$のコンパクト部分集合$C$をとる.$y \in X \setminus C$と$x\in C$について$x \in U_x,y\in V_x,U_x \cap V_x=\emptyset$となる開集合$U_x,V_x\in \mathcal{O}$が存在する.$\bigcup\limits_{x\in C }U_x\supset C$より有限部分集合$X' \subset X$が存在して$\bigcup\limits_{x\in X'}U_x\supset C$である.$V=\bigcap\limits_{x\in X'}V_x$とすれば$V$は$y$の開近傍で$V\cap C=\emptyset$である.したがって$X\setminus C$は開集合である.

(2)$f\colon X \times Y \rightarrow \mathbb{R};(x,y)\mapsto x-y$とすると$f$は連続である.よって$f^{-1}(0)=F$は閉集合である.
$g\colon F \rightarrow \mathbb{R};(x,x)\mapsto x$とすると$g$は連続である.$\sup g(x)=1$より$g$は最大値をもたない.したがって$F$はコンパクトでない.

\fbox{4}
(1)$zx+iy$とする.$e^z=-e^{-z}$より$e^x=|e^z|=|-e^{-z}|=e^{-x}$であるから$x=-x$である.したがって$x=0$である.$e^{iy}=-e^{-iy}$より$\cos y=-\cos y$である.よって$y=\frac{i\pi}{2}+n\pi\quad(n\in \mathbb{Z})$である.

(2)$|e^{-iz}|=|e^{s-i\pm{R}}|=e^s\le e^\pi$である.$z=R+is$のとき$|e^z+e^{-z}|\ge |e^z|-|e^{-z}|=e^R-e^{-R}$であり,$z=-R+is$のとき$|e^z+e^{-z}|\ge |e^{-z}|-|e^{z}|=e^R-e^{-R}$である.
よって$\sup\limits_{z\in \Gamma_R}|e^{-iz}/(e^z+e^{-z})|\le \frac{e^\pi}{e^R-e^{-R}}$である.

(3)$f(z)=e^{-iz}/(e^z+e^{-z})$とする.(1)で求めた点以外で$f$は正則である.したがって積分経路$C$を4点$-R,R,R+i\pi,-R+i\pi$を結んでできる長方形を反時計回りに進むとすると,留数定理から$\int_C f(z)dz=2\pi i(\mathrm{Res}(f,i\pi/2))$である.
$\lim\limits_{z\to i\pi/2}\frac{z-1\pi/2}{e^z+e^{-z}}=\lim\limits_{z\to i\pi/2}\frac{1}{e^z-e^{-z}}=\frac{1}{2i}$であるから$\int_C f(z)dz=\pi e^{\pi/2}$である.

また \begin{align}
	\int_{R+i\pi}^{-R+i\pi}f(z)dz&=\int_{R}^{-R}\frac{e^{-i(x+i\pi)}}{e^{x+i\pi}+e^{-x-i\pi}}dx=\int_{-R}^{R}\frac{e^{\pi}e^{-ix}}{e^{x}+e^{-x}}dx\\
	\left| \int_{R}^{R+i\pi}f(z)dz \right|&\le \int_{R}^{R+i\pi}\left| \frac{e^{-iz}}{e^z+e^{-z}} \right|dz\le \pi e^\pi/(e^R-e^{-R})\rightarrow 0\quad (R\to \infty)\\
	\left| \int_{-R+i\pi}^{-R}f(z)dz \right|&\le \int_{-R+i\pi}^{-R}\left| \frac{e^{-iz}}{e^z+e^{-z}} \right|dz\le \pi e^\pi/(e^R-e^{-R})\rightarrow 0\quad (R\to \infty)
\end{align}である.よって$\pi e^{\pi/2}=\int_{-\infty}^\infty \cos x/(e^{x}+e^{-x})dx+e^{\pi}\int_{-\infty}^\infty e^{-ix}/(e^{x}+e^{-x})dx$である.
よって$\int_{-\infty}^\infty \cos x/(e^{x}+e^{-x})dx=\frac{\pi }{e^{\pi/2}+e^{-\pi/2}}$である.
	





\end{document}