\documentclass[
		book,
		head_space=20mm,
		foot_space=20mm,
		gutter=10mm,
		line_length=190mm
]{jlreq}

%----------
%LuaLaTeXで実行する!!
%----------
%各章節には以下を書く. 1-03.texのような名前にする
%----------
% \documentclass[
% 		book,
% 		head_space=20mm,
% 		foot_space=20mm,
% 		gutter=10mm,
% 		line_length=190mm
% ]{jlreq}
% 
%----------
%LuaLaTeXで実行する!!
%----------
%各章節には以下を書く. 1-03.texのような名前にする
%----------
% \documentclass[
% 		book,
% 		head_space=20mm,
% 		foot_space=20mm,
% 		gutter=10mm,
% 		line_length=190mm
% ]{jlreq}
% 
%----------
%LuaLaTeXで実行する!!
%----------
%各章節には以下を書く. 1-03.texのような名前にする
%----------
% \documentclass[
% 		book,
% 		head_space=20mm,
% 		foot_space=20mm,
% 		gutter=10mm,
% 		line_length=190mm
% ]{jlreq}
% \input {preamble.tex}
% \usepackage{docmute} %ファイル分割
% \begin{document}

% %\chapter{章のタイトル}
% \section{節のタイトル}
% no text

% \end{document}
%----------

%main.texには以下を書く
%----------
% \documentclass[
% 		book,
% 		head_space=20mm,
% 		foot_space=20mm,
% 		gutter=10mm,
% 		line_length=190mm,
%         openany
% ]{jlreq}
% \input {preamble.tex}
% \usepackage{docmute} %ファイル分割
% \begin{document}

% %---------- 1章1節
% \input 1-01.tex
% %---------- 1章2節
% \input 1-02.tex
% % ---------- 1章3節
% \input 1-03.tex
% % ---------- 1章4節
% \input 1-04.tex
% % ---------- 1章5節
% \input 1-05.tex
% % ---------- 1章6節
% \input 1-06.tex
% %---------- 1章7節
% \input 1-07.tex
% % ---------- 1章8節
% \input 1-08.tex
% % ---------- 1章9節
% \input 1-09.tex
% % ---------- 1章10節
% \input 1-10.tex
% % ---------- 1章11節
% \input 1-11.tex
% % ---------- 1章12節
% \input 1-12.tex
% % ---------- 参考文献
% \input reference.tex
% \end{document}
% ----------



\usepackage{bxtexlogo}
\usepackage{amsthm}
\usepackage{amsmath}
\usepackage{bbm} %小文字の黒板文字
\usepackage{physics}
\usepackage{amsfonts}
\usepackage{graphicx}
\usepackage{mathtools}
\usepackage{enumitem}
\usepackage[margin=20truemm]{geometry}
\usepackage{textcomp}
\usepackage{bm}
\usepackage{mathrsfs}
\usepackage{latexsym}
\usepackage{amssymb}
\usepackage{algorithmic}
\usepackage{algorithm}
\usepackage{tikz}
\usetikzlibrary{arrows.meta}
\usetikzlibrary{math,matrix,backgrounds}
\usetikzlibrary{angles}
\usetikzlibrary{calc}


%----------
%日本語フォント
% \usepackage[deluxe]{otf} platex用 lualatexでは動かない

%----------
%欧文フォント
\usepackage[T1]{fontenc}

%----------
%文字色
\usepackage{color}

%----------
\setlength{\parindent}{2\zw} %インデントの設定

%----------
% %参照した数式にだけ番号を振る cleverrefと併用するとうまくいかない
% \mathtoolsset{showonlyrefs=true}
%----------

%----------
%集合の中線
\newcommand{\relmiddle}[1]{\mathrel{}\middle#1\mathrel{}}
% \middle| の代わりに \relmiddle| を付ける
\newcommand{\sgn}{\mathop{\mathrm{sgn}}} %置換sgn
\newcommand{\Int}{\mathop{\mathrm{Int}}} %位相空間の内部Int
\newcommand{\Ext}{\mathop{\mathrm{Ext}}} %位相空間の外部Ext
\newcommand{\Cl}{\mathop{\mathrm{Cl}}} %位相空間の閉包Cl
\newcommand{\supp}{\mathop{\mathrm{supp}}} %関数の台supp
\newcommand{\restrict}[2]{\left. #1 \right \vert_{#2}}%関数の制限 \restrict{f}{A} = f|_A
\newcommand{\Ker}{\mathop{\mathrm{Ker}}}
\newcommand{\Coker}{\mathop{\mathrm{Coker}}}
\newcommand{\coker}{\mathop{\mathrm{coker}}}
\newcommand{\Coim}{\mathop{\mathrm{Coim}}}
\newcommand{\coim}{\mathop{\mathrm{coim}}}
\newcommand{\id}{\mathop{\mathrm{id}}}
\newcommand{\Gal}{\mathop{\mathrm{Gal}}}

\newtheorem{definition}{定義}[section]

\usepackage{aliascnt}

% \newaliastheorem{(環境とカウンターの名前)}{(元となるカウンターの名前)}{(表示される文字列)}
\newcommand*{\newaliastheorem}[3]{%
  \newaliascnt{#1}{#2}%
  \newtheorem{#1}[#1]{#3}%
  \aliascntresetthe{#1}%
  \expandafter\newcommand\csname #1autorefname\endcsname{#3}%
}
\newaliastheorem{proposition}{definition}{命題} 
\newaliastheorem{theorem}{definition}{定理}
\newaliastheorem{lemma}{definition}{補題}
\newaliastheorem{corollary}{definition}{系}
\newaliastheorem{example}{definition}{例}
\newaliastheorem{practice}{definition}{演習問題}

\newtheorem*{longproof}{証明}
\newtheorem*{answer}{解答}
\newtheorem*{supplement}{補足}
\newtheorem*{remark}{注意}
%----------

%----------
%古い記法を注意するパッケージ
\RequirePackage[l2tabu, orthodox]{nag}
%----------


% 定理環境(tcolorbox)
\usepackage{tcolorbox} %箱
\tcbuselibrary{breakable,skins,theorems}
\tcolorboxenvironment{definition}{
	blanker,breakable,
	left=3mm,right=3mm,
	top=2mm,bottom=2mm,
	before skip=15pt,after skip=20pt,
	borderline ={0.5pt}{0pt}{black}
}
\newtcolorbox{emptydefinition}{
	blanker,breakable,
	left=3mm,right=3mm,
	top=2mm,bottom=2mm,
	before skip=15pt,after skip=20pt,
	borderline ={0.5pt}{0pt}{black}
}
%----------
\tcolorboxenvironment{proposition}{
	blanker,breakable,
	left=3mm,right=3mm,
	top=3mm,bottom=3mm,
	before skip=15pt,after skip=15pt,
	borderline={0.5pt}{0pt}{black}
}
\newtcolorbox{emptyproposition}{
	blanker,breakable,
	left=3mm,right=3mm,
	top=3mm,bottom=3mm,
	before skip=15pt,after skip=15pt,
	borderline={0.5pt}{0pt}{black}
}
%----------
\tcolorboxenvironment{theorem}{
	blanker,breakable,
	left=3mm,right=3mm,
	top=3mm,bottom=3mm,
    sharp corners,boxrule=0.6pt,
	before skip=15pt,after skip=15pt,
	borderline={0.5pt}{0pt}{black},
    borderline={0.5pt}{1.5pt}{black}
}
\newtcolorbox{emptytheorem}{
	blanker,breakable,
	left=3mm,right=3mm,
	top=3mm,bottom=3mm,
    sharp corners,boxrule=0.6pt,
	before skip=15pt,after skip=15pt,
	borderline={0.5pt}{0pt}{black},
    borderline={0.5pt}{1.5pt}{black}
}
%----------
\tcolorboxenvironment{lemma}{
	blanker,breakable,
	left=3mm,right=3mm,
	top=3mm,bottom=3mm,
	before skip=15pt,after skip=15pt,
	borderline={0.5pt}{0pt}{black}
}
%----------
\tcolorboxenvironment{corollary}{
	blanker,breakable,
	left=3mm,right=3mm,
	top=3mm,bottom=3mm,
	before skip=15pt,after skip=15pt,
	borderline={1.0pt}{0pt}{black,dotted}
}
\newtcolorbox{emptycorollary}{
	blanker,breakable,
	left=3mm,right=3mm,
	top=3mm,bottom=3mm,
	before skip=15pt,after skip=15pt,
	borderline={1.0pt}{0pt}{black,dotted}
}
%----------
\tcolorboxenvironment{example}{
	blanker,breakable,
	left=3mm,right=3mm,
	top=3mm,bottom=3mm,
	before skip=15pt,after skip=15pt,
	borderline={0.5pt}{0pt}{black}
}
%----------
\tcolorboxenvironment{practice}{
	blanker,breakable,
	left=3mm,right=3mm,
	top=3mm,bottom=3mm,
	before skip=15pt,after skip=15pt,
	borderline={0.5pt}{0pt}{black}
}
%----------
\tcolorboxenvironment{proof}{
	blanker,breakable,
	left=3mm,right=3mm,
	top=2mm,bottom=2mm,
	before skip=15pt,after skip=20pt,
	% borderline west={1.5pt}{0pt}{black,dotted}
	borderline vertical={1pt}{0pt}{black,dotted}
	% borderline vertical={0.8pt}{0pt}{black,dotted,arrows={Square[scale=0.5]-Square[scale=0.5]}}
	}
%----------
\tcolorboxenvironment{supplement}{
	blanker,breakable,
	left=3mm,right=3mm,
	top=2mm,bottom=2mm,
	before skip=15pt,after skip=20pt,
	% borderline west={1.5pt}{0pt}{black,dotted}
	% borderline vertical={0.5pt}{0pt}{black,arrows = {Circle[scale=0.7]-Circle[scale=0.7]}}
	borderline vertical={0.5pt}{0pt}{black}
	% borderline vertical={0.5pt}{0pt}{black},
	% borderline north={0.5pt}{0pt}{white,arrows={Circle[black,scale=0.7]-Circle[black,scale=0.7]}}
	}
%----------
\tcolorboxenvironment{remark}{
	blanker,breakable,
	left=3mm,right=3mm,
	top=1mm,bottom=1mm,
	before skip=15pt,after skip=20pt,
	% borderline west={1.5pt}{0pt}{black,dotted}
	% borderline vertical={0.5pt}{0pt}{black,arrows = {Circle[scale=0.7]-Circle[scale=0.7]}}
	borderline vertical={0.5pt}{0pt}{black}
	% borderline vertical={0.5pt}{0pt}{black},
	% borderline north={0.5pt}{0pt}{white,arrows={Circle[black,scale=0.7]-Circle[black,scale=0.7]}}
	}
    
%---------------------
 

%----------
%ハイパーリンク
% 「%」は以降の内容を「改行コードも含めて」無視するコマンド
\usepackage[%
%  dvipdfmx,% 欧文ではコメントアウトする
luatex,%
pdfencoding=auto,%
 setpagesize=false,%
 bookmarks=true,%
 bookmarksdepth=tocdepth,%
 bookmarksnumbered=true,%
 colorlinks=false,%
 pdftitle={},%
 pdfsubject={},%
 pdfauthor={},%
 pdfkeywords={}%
]{hyperref}
%------------


%参照 参照するときに自動で環境名を含んで参照する
\usepackage[nameinlink]{cleveref}
\let\normalref\ref
\renewcommand{\ref}{\cref}
\crefname{definition}{定義}{定義}
\crefname{proposition}{命題}{命題}
\crefname{theorem}{定理}{定理}
\crefname{lemma}{補題}{補題}
\crefname{corollary}{系}{系}
\crefname{example}{例}{例}
\crefname{practice}{演習問題}{演習問題}
\crefname{equation}{式}{式} 
\crefname{chapter}{第}{第}
\creflabelformat{chapter}{#2#1章#3}
\crefname{section}{第}{第}
\creflabelformat{section}{#2#1節#3}
\crefname{subsection}{第}{第}
\creflabelformat{subsection}{#2#1小節#3}
%----------

%---------------------
%章跨ぎの参照が不具合を起こすための代わり
% \mylabl でラベル付け
\newcommand{\mylabel}[1]{
\label{#1}
\hypertarget{#1}{}
}
% \myref で環境名付きリンクをつける
\newcommand{\myref}[1]{
\hyperlink{#1}{\cref*{#1}}
}
%-----------------

\usepackage{autonum} %参照した数式にだけ番号を振る
% \usepackage{docmute} %ファイル分割
% \begin{document}

% %\chapter{章のタイトル}
% \section{節のタイトル}
% no text

% \end{document}
%----------

%main.texには以下を書く
%----------
% \documentclass[
% 		book,
% 		head_space=20mm,
% 		foot_space=20mm,
% 		gutter=10mm,
% 		line_length=190mm,
%         openany
% ]{jlreq}
% 
%----------
%LuaLaTeXで実行する!!
%----------
%各章節には以下を書く. 1-03.texのような名前にする
%----------
% \documentclass[
% 		book,
% 		head_space=20mm,
% 		foot_space=20mm,
% 		gutter=10mm,
% 		line_length=190mm
% ]{jlreq}
% \input {preamble.tex}
% \usepackage{docmute} %ファイル分割
% \begin{document}

% %\chapter{章のタイトル}
% \section{節のタイトル}
% no text

% \end{document}
%----------

%main.texには以下を書く
%----------
% \documentclass[
% 		book,
% 		head_space=20mm,
% 		foot_space=20mm,
% 		gutter=10mm,
% 		line_length=190mm,
%         openany
% ]{jlreq}
% \input {preamble.tex}
% \usepackage{docmute} %ファイル分割
% \begin{document}

% %---------- 1章1節
% \input 1-01.tex
% %---------- 1章2節
% \input 1-02.tex
% % ---------- 1章3節
% \input 1-03.tex
% % ---------- 1章4節
% \input 1-04.tex
% % ---------- 1章5節
% \input 1-05.tex
% % ---------- 1章6節
% \input 1-06.tex
% %---------- 1章7節
% \input 1-07.tex
% % ---------- 1章8節
% \input 1-08.tex
% % ---------- 1章9節
% \input 1-09.tex
% % ---------- 1章10節
% \input 1-10.tex
% % ---------- 1章11節
% \input 1-11.tex
% % ---------- 1章12節
% \input 1-12.tex
% % ---------- 参考文献
% \input reference.tex
% \end{document}
% ----------



\usepackage{bxtexlogo}
\usepackage{amsthm}
\usepackage{amsmath}
\usepackage{bbm} %小文字の黒板文字
\usepackage{physics}
\usepackage{amsfonts}
\usepackage{graphicx}
\usepackage{mathtools}
\usepackage{enumitem}
\usepackage[margin=20truemm]{geometry}
\usepackage{textcomp}
\usepackage{bm}
\usepackage{mathrsfs}
\usepackage{latexsym}
\usepackage{amssymb}
\usepackage{algorithmic}
\usepackage{algorithm}
\usepackage{tikz}
\usetikzlibrary{arrows.meta}
\usetikzlibrary{math,matrix,backgrounds}
\usetikzlibrary{angles}
\usetikzlibrary{calc}


%----------
%日本語フォント
% \usepackage[deluxe]{otf} platex用 lualatexでは動かない

%----------
%欧文フォント
\usepackage[T1]{fontenc}

%----------
%文字色
\usepackage{color}

%----------
\setlength{\parindent}{2\zw} %インデントの設定

%----------
% %参照した数式にだけ番号を振る cleverrefと併用するとうまくいかない
% \mathtoolsset{showonlyrefs=true}
%----------

%----------
%集合の中線
\newcommand{\relmiddle}[1]{\mathrel{}\middle#1\mathrel{}}
% \middle| の代わりに \relmiddle| を付ける
\newcommand{\sgn}{\mathop{\mathrm{sgn}}} %置換sgn
\newcommand{\Int}{\mathop{\mathrm{Int}}} %位相空間の内部Int
\newcommand{\Ext}{\mathop{\mathrm{Ext}}} %位相空間の外部Ext
\newcommand{\Cl}{\mathop{\mathrm{Cl}}} %位相空間の閉包Cl
\newcommand{\supp}{\mathop{\mathrm{supp}}} %関数の台supp
\newcommand{\restrict}[2]{\left. #1 \right \vert_{#2}}%関数の制限 \restrict{f}{A} = f|_A
\newcommand{\Ker}{\mathop{\mathrm{Ker}}}
\newcommand{\Coker}{\mathop{\mathrm{Coker}}}
\newcommand{\coker}{\mathop{\mathrm{coker}}}
\newcommand{\Coim}{\mathop{\mathrm{Coim}}}
\newcommand{\coim}{\mathop{\mathrm{coim}}}
\newcommand{\id}{\mathop{\mathrm{id}}}
\newcommand{\Gal}{\mathop{\mathrm{Gal}}}

\newtheorem{definition}{定義}[section]

\usepackage{aliascnt}

% \newaliastheorem{(環境とカウンターの名前)}{(元となるカウンターの名前)}{(表示される文字列)}
\newcommand*{\newaliastheorem}[3]{%
  \newaliascnt{#1}{#2}%
  \newtheorem{#1}[#1]{#3}%
  \aliascntresetthe{#1}%
  \expandafter\newcommand\csname #1autorefname\endcsname{#3}%
}
\newaliastheorem{proposition}{definition}{命題} 
\newaliastheorem{theorem}{definition}{定理}
\newaliastheorem{lemma}{definition}{補題}
\newaliastheorem{corollary}{definition}{系}
\newaliastheorem{example}{definition}{例}
\newaliastheorem{practice}{definition}{演習問題}

\newtheorem*{longproof}{証明}
\newtheorem*{answer}{解答}
\newtheorem*{supplement}{補足}
\newtheorem*{remark}{注意}
%----------

%----------
%古い記法を注意するパッケージ
\RequirePackage[l2tabu, orthodox]{nag}
%----------


% 定理環境(tcolorbox)
\usepackage{tcolorbox} %箱
\tcbuselibrary{breakable,skins,theorems}
\tcolorboxenvironment{definition}{
	blanker,breakable,
	left=3mm,right=3mm,
	top=2mm,bottom=2mm,
	before skip=15pt,after skip=20pt,
	borderline ={0.5pt}{0pt}{black}
}
\newtcolorbox{emptydefinition}{
	blanker,breakable,
	left=3mm,right=3mm,
	top=2mm,bottom=2mm,
	before skip=15pt,after skip=20pt,
	borderline ={0.5pt}{0pt}{black}
}
%----------
\tcolorboxenvironment{proposition}{
	blanker,breakable,
	left=3mm,right=3mm,
	top=3mm,bottom=3mm,
	before skip=15pt,after skip=15pt,
	borderline={0.5pt}{0pt}{black}
}
\newtcolorbox{emptyproposition}{
	blanker,breakable,
	left=3mm,right=3mm,
	top=3mm,bottom=3mm,
	before skip=15pt,after skip=15pt,
	borderline={0.5pt}{0pt}{black}
}
%----------
\tcolorboxenvironment{theorem}{
	blanker,breakable,
	left=3mm,right=3mm,
	top=3mm,bottom=3mm,
    sharp corners,boxrule=0.6pt,
	before skip=15pt,after skip=15pt,
	borderline={0.5pt}{0pt}{black},
    borderline={0.5pt}{1.5pt}{black}
}
\newtcolorbox{emptytheorem}{
	blanker,breakable,
	left=3mm,right=3mm,
	top=3mm,bottom=3mm,
    sharp corners,boxrule=0.6pt,
	before skip=15pt,after skip=15pt,
	borderline={0.5pt}{0pt}{black},
    borderline={0.5pt}{1.5pt}{black}
}
%----------
\tcolorboxenvironment{lemma}{
	blanker,breakable,
	left=3mm,right=3mm,
	top=3mm,bottom=3mm,
	before skip=15pt,after skip=15pt,
	borderline={0.5pt}{0pt}{black}
}
%----------
\tcolorboxenvironment{corollary}{
	blanker,breakable,
	left=3mm,right=3mm,
	top=3mm,bottom=3mm,
	before skip=15pt,after skip=15pt,
	borderline={1.0pt}{0pt}{black,dotted}
}
\newtcolorbox{emptycorollary}{
	blanker,breakable,
	left=3mm,right=3mm,
	top=3mm,bottom=3mm,
	before skip=15pt,after skip=15pt,
	borderline={1.0pt}{0pt}{black,dotted}
}
%----------
\tcolorboxenvironment{example}{
	blanker,breakable,
	left=3mm,right=3mm,
	top=3mm,bottom=3mm,
	before skip=15pt,after skip=15pt,
	borderline={0.5pt}{0pt}{black}
}
%----------
\tcolorboxenvironment{practice}{
	blanker,breakable,
	left=3mm,right=3mm,
	top=3mm,bottom=3mm,
	before skip=15pt,after skip=15pt,
	borderline={0.5pt}{0pt}{black}
}
%----------
\tcolorboxenvironment{proof}{
	blanker,breakable,
	left=3mm,right=3mm,
	top=2mm,bottom=2mm,
	before skip=15pt,after skip=20pt,
	% borderline west={1.5pt}{0pt}{black,dotted}
	borderline vertical={1pt}{0pt}{black,dotted}
	% borderline vertical={0.8pt}{0pt}{black,dotted,arrows={Square[scale=0.5]-Square[scale=0.5]}}
	}
%----------
\tcolorboxenvironment{supplement}{
	blanker,breakable,
	left=3mm,right=3mm,
	top=2mm,bottom=2mm,
	before skip=15pt,after skip=20pt,
	% borderline west={1.5pt}{0pt}{black,dotted}
	% borderline vertical={0.5pt}{0pt}{black,arrows = {Circle[scale=0.7]-Circle[scale=0.7]}}
	borderline vertical={0.5pt}{0pt}{black}
	% borderline vertical={0.5pt}{0pt}{black},
	% borderline north={0.5pt}{0pt}{white,arrows={Circle[black,scale=0.7]-Circle[black,scale=0.7]}}
	}
%----------
\tcolorboxenvironment{remark}{
	blanker,breakable,
	left=3mm,right=3mm,
	top=1mm,bottom=1mm,
	before skip=15pt,after skip=20pt,
	% borderline west={1.5pt}{0pt}{black,dotted}
	% borderline vertical={0.5pt}{0pt}{black,arrows = {Circle[scale=0.7]-Circle[scale=0.7]}}
	borderline vertical={0.5pt}{0pt}{black}
	% borderline vertical={0.5pt}{0pt}{black},
	% borderline north={0.5pt}{0pt}{white,arrows={Circle[black,scale=0.7]-Circle[black,scale=0.7]}}
	}
    
%---------------------
 

%----------
%ハイパーリンク
% 「%」は以降の内容を「改行コードも含めて」無視するコマンド
\usepackage[%
%  dvipdfmx,% 欧文ではコメントアウトする
luatex,%
pdfencoding=auto,%
 setpagesize=false,%
 bookmarks=true,%
 bookmarksdepth=tocdepth,%
 bookmarksnumbered=true,%
 colorlinks=false,%
 pdftitle={},%
 pdfsubject={},%
 pdfauthor={},%
 pdfkeywords={}%
]{hyperref}
%------------


%参照 参照するときに自動で環境名を含んで参照する
\usepackage[nameinlink]{cleveref}
\let\normalref\ref
\renewcommand{\ref}{\cref}
\crefname{definition}{定義}{定義}
\crefname{proposition}{命題}{命題}
\crefname{theorem}{定理}{定理}
\crefname{lemma}{補題}{補題}
\crefname{corollary}{系}{系}
\crefname{example}{例}{例}
\crefname{practice}{演習問題}{演習問題}
\crefname{equation}{式}{式} 
\crefname{chapter}{第}{第}
\creflabelformat{chapter}{#2#1章#3}
\crefname{section}{第}{第}
\creflabelformat{section}{#2#1節#3}
\crefname{subsection}{第}{第}
\creflabelformat{subsection}{#2#1小節#3}
%----------

%---------------------
%章跨ぎの参照が不具合を起こすための代わり
% \mylabl でラベル付け
\newcommand{\mylabel}[1]{
\label{#1}
\hypertarget{#1}{}
}
% \myref で環境名付きリンクをつける
\newcommand{\myref}[1]{
\hyperlink{#1}{\cref*{#1}}
}
%-----------------

\usepackage{autonum} %参照した数式にだけ番号を振る
% \usepackage{docmute} %ファイル分割
% \begin{document}

% %---------- 1章1節
% \input 1-01.tex
% %---------- 1章2節
% \input 1-02.tex
% % ---------- 1章3節
% \input 1-03.tex
% % ---------- 1章4節
% \input 1-04.tex
% % ---------- 1章5節
% \input 1-05.tex
% % ---------- 1章6節
% \input 1-06.tex
% %---------- 1章7節
% \input 1-07.tex
% % ---------- 1章8節
% \input 1-08.tex
% % ---------- 1章9節
% \input 1-09.tex
% % ---------- 1章10節
% \input 1-10.tex
% % ---------- 1章11節
% \input 1-11.tex
% % ---------- 1章12節
% \input 1-12.tex
% % ---------- 参考文献
% \input reference.tex
% \end{document}
% ----------



\usepackage{bxtexlogo}
\usepackage{amsthm}
\usepackage{amsmath}
\usepackage{bbm} %小文字の黒板文字
\usepackage{physics}
\usepackage{amsfonts}
\usepackage{graphicx}
\usepackage{mathtools}
\usepackage{enumitem}
\usepackage[margin=20truemm]{geometry}
\usepackage{textcomp}
\usepackage{bm}
\usepackage{mathrsfs}
\usepackage{latexsym}
\usepackage{amssymb}
\usepackage{algorithmic}
\usepackage{algorithm}
\usepackage{tikz}
\usetikzlibrary{arrows.meta}
\usetikzlibrary{math,matrix,backgrounds}
\usetikzlibrary{angles}
\usetikzlibrary{calc}


%----------
%日本語フォント
% \usepackage[deluxe]{otf} platex用 lualatexでは動かない

%----------
%欧文フォント
\usepackage[T1]{fontenc}

%----------
%文字色
\usepackage{color}

%----------
\setlength{\parindent}{2\zw} %インデントの設定

%----------
% %参照した数式にだけ番号を振る cleverrefと併用するとうまくいかない
% \mathtoolsset{showonlyrefs=true}
%----------

%----------
%集合の中線
\newcommand{\relmiddle}[1]{\mathrel{}\middle#1\mathrel{}}
% \middle| の代わりに \relmiddle| を付ける
\newcommand{\sgn}{\mathop{\mathrm{sgn}}} %置換sgn
\newcommand{\Int}{\mathop{\mathrm{Int}}} %位相空間の内部Int
\newcommand{\Ext}{\mathop{\mathrm{Ext}}} %位相空間の外部Ext
\newcommand{\Cl}{\mathop{\mathrm{Cl}}} %位相空間の閉包Cl
\newcommand{\supp}{\mathop{\mathrm{supp}}} %関数の台supp
\newcommand{\restrict}[2]{\left. #1 \right \vert_{#2}}%関数の制限 \restrict{f}{A} = f|_A
\newcommand{\Ker}{\mathop{\mathrm{Ker}}}
\newcommand{\Coker}{\mathop{\mathrm{Coker}}}
\newcommand{\coker}{\mathop{\mathrm{coker}}}
\newcommand{\Coim}{\mathop{\mathrm{Coim}}}
\newcommand{\coim}{\mathop{\mathrm{coim}}}
\newcommand{\id}{\mathop{\mathrm{id}}}
\newcommand{\Gal}{\mathop{\mathrm{Gal}}}

\newtheorem{definition}{定義}[section]

\usepackage{aliascnt}

% \newaliastheorem{(環境とカウンターの名前)}{(元となるカウンターの名前)}{(表示される文字列)}
\newcommand*{\newaliastheorem}[3]{%
  \newaliascnt{#1}{#2}%
  \newtheorem{#1}[#1]{#3}%
  \aliascntresetthe{#1}%
  \expandafter\newcommand\csname #1autorefname\endcsname{#3}%
}
\newaliastheorem{proposition}{definition}{命題} 
\newaliastheorem{theorem}{definition}{定理}
\newaliastheorem{lemma}{definition}{補題}
\newaliastheorem{corollary}{definition}{系}
\newaliastheorem{example}{definition}{例}
\newaliastheorem{practice}{definition}{演習問題}

\newtheorem*{longproof}{証明}
\newtheorem*{answer}{解答}
\newtheorem*{supplement}{補足}
\newtheorem*{remark}{注意}
%----------

%----------
%古い記法を注意するパッケージ
\RequirePackage[l2tabu, orthodox]{nag}
%----------


% 定理環境(tcolorbox)
\usepackage{tcolorbox} %箱
\tcbuselibrary{breakable,skins,theorems}
\tcolorboxenvironment{definition}{
	blanker,breakable,
	left=3mm,right=3mm,
	top=2mm,bottom=2mm,
	before skip=15pt,after skip=20pt,
	borderline ={0.5pt}{0pt}{black}
}
\newtcolorbox{emptydefinition}{
	blanker,breakable,
	left=3mm,right=3mm,
	top=2mm,bottom=2mm,
	before skip=15pt,after skip=20pt,
	borderline ={0.5pt}{0pt}{black}
}
%----------
\tcolorboxenvironment{proposition}{
	blanker,breakable,
	left=3mm,right=3mm,
	top=3mm,bottom=3mm,
	before skip=15pt,after skip=15pt,
	borderline={0.5pt}{0pt}{black}
}
\newtcolorbox{emptyproposition}{
	blanker,breakable,
	left=3mm,right=3mm,
	top=3mm,bottom=3mm,
	before skip=15pt,after skip=15pt,
	borderline={0.5pt}{0pt}{black}
}
%----------
\tcolorboxenvironment{theorem}{
	blanker,breakable,
	left=3mm,right=3mm,
	top=3mm,bottom=3mm,
    sharp corners,boxrule=0.6pt,
	before skip=15pt,after skip=15pt,
	borderline={0.5pt}{0pt}{black},
    borderline={0.5pt}{1.5pt}{black}
}
\newtcolorbox{emptytheorem}{
	blanker,breakable,
	left=3mm,right=3mm,
	top=3mm,bottom=3mm,
    sharp corners,boxrule=0.6pt,
	before skip=15pt,after skip=15pt,
	borderline={0.5pt}{0pt}{black},
    borderline={0.5pt}{1.5pt}{black}
}
%----------
\tcolorboxenvironment{lemma}{
	blanker,breakable,
	left=3mm,right=3mm,
	top=3mm,bottom=3mm,
	before skip=15pt,after skip=15pt,
	borderline={0.5pt}{0pt}{black}
}
%----------
\tcolorboxenvironment{corollary}{
	blanker,breakable,
	left=3mm,right=3mm,
	top=3mm,bottom=3mm,
	before skip=15pt,after skip=15pt,
	borderline={1.0pt}{0pt}{black,dotted}
}
\newtcolorbox{emptycorollary}{
	blanker,breakable,
	left=3mm,right=3mm,
	top=3mm,bottom=3mm,
	before skip=15pt,after skip=15pt,
	borderline={1.0pt}{0pt}{black,dotted}
}
%----------
\tcolorboxenvironment{example}{
	blanker,breakable,
	left=3mm,right=3mm,
	top=3mm,bottom=3mm,
	before skip=15pt,after skip=15pt,
	borderline={0.5pt}{0pt}{black}
}
%----------
\tcolorboxenvironment{practice}{
	blanker,breakable,
	left=3mm,right=3mm,
	top=3mm,bottom=3mm,
	before skip=15pt,after skip=15pt,
	borderline={0.5pt}{0pt}{black}
}
%----------
\tcolorboxenvironment{proof}{
	blanker,breakable,
	left=3mm,right=3mm,
	top=2mm,bottom=2mm,
	before skip=15pt,after skip=20pt,
	% borderline west={1.5pt}{0pt}{black,dotted}
	borderline vertical={1pt}{0pt}{black,dotted}
	% borderline vertical={0.8pt}{0pt}{black,dotted,arrows={Square[scale=0.5]-Square[scale=0.5]}}
	}
%----------
\tcolorboxenvironment{supplement}{
	blanker,breakable,
	left=3mm,right=3mm,
	top=2mm,bottom=2mm,
	before skip=15pt,after skip=20pt,
	% borderline west={1.5pt}{0pt}{black,dotted}
	% borderline vertical={0.5pt}{0pt}{black,arrows = {Circle[scale=0.7]-Circle[scale=0.7]}}
	borderline vertical={0.5pt}{0pt}{black}
	% borderline vertical={0.5pt}{0pt}{black},
	% borderline north={0.5pt}{0pt}{white,arrows={Circle[black,scale=0.7]-Circle[black,scale=0.7]}}
	}
%----------
\tcolorboxenvironment{remark}{
	blanker,breakable,
	left=3mm,right=3mm,
	top=1mm,bottom=1mm,
	before skip=15pt,after skip=20pt,
	% borderline west={1.5pt}{0pt}{black,dotted}
	% borderline vertical={0.5pt}{0pt}{black,arrows = {Circle[scale=0.7]-Circle[scale=0.7]}}
	borderline vertical={0.5pt}{0pt}{black}
	% borderline vertical={0.5pt}{0pt}{black},
	% borderline north={0.5pt}{0pt}{white,arrows={Circle[black,scale=0.7]-Circle[black,scale=0.7]}}
	}
    
%---------------------
 

%----------
%ハイパーリンク
% 「%」は以降の内容を「改行コードも含めて」無視するコマンド
\usepackage[%
%  dvipdfmx,% 欧文ではコメントアウトする
luatex,%
pdfencoding=auto,%
 setpagesize=false,%
 bookmarks=true,%
 bookmarksdepth=tocdepth,%
 bookmarksnumbered=true,%
 colorlinks=false,%
 pdftitle={},%
 pdfsubject={},%
 pdfauthor={},%
 pdfkeywords={}%
]{hyperref}
%------------


%参照 参照するときに自動で環境名を含んで参照する
\usepackage[nameinlink]{cleveref}
\let\normalref\ref
\renewcommand{\ref}{\cref}
\crefname{definition}{定義}{定義}
\crefname{proposition}{命題}{命題}
\crefname{theorem}{定理}{定理}
\crefname{lemma}{補題}{補題}
\crefname{corollary}{系}{系}
\crefname{example}{例}{例}
\crefname{practice}{演習問題}{演習問題}
\crefname{equation}{式}{式} 
\crefname{chapter}{第}{第}
\creflabelformat{chapter}{#2#1章#3}
\crefname{section}{第}{第}
\creflabelformat{section}{#2#1節#3}
\crefname{subsection}{第}{第}
\creflabelformat{subsection}{#2#1小節#3}
%----------

%---------------------
%章跨ぎの参照が不具合を起こすための代わり
% \mylabl でラベル付け
\newcommand{\mylabel}[1]{
\label{#1}
\hypertarget{#1}{}
}
% \myref で環境名付きリンクをつける
\newcommand{\myref}[1]{
\hyperlink{#1}{\cref*{#1}}
}
%-----------------

\usepackage{autonum} %参照した数式にだけ番号を振る
% \usepackage{docmute} %ファイル分割
% \begin{document}

% %\chapter{章のタイトル}
% \section{節のタイトル}
% no text

% \end{document}
%----------

%main.texには以下を書く
%----------
% \documentclass[
% 		book,
% 		head_space=20mm,
% 		foot_space=20mm,
% 		gutter=10mm,
% 		line_length=190mm,
%         openany
% ]{jlreq}
% 
%----------
%LuaLaTeXで実行する!!
%----------
%各章節には以下を書く. 1-03.texのような名前にする
%----------
% \documentclass[
% 		book,
% 		head_space=20mm,
% 		foot_space=20mm,
% 		gutter=10mm,
% 		line_length=190mm
% ]{jlreq}
% 
%----------
%LuaLaTeXで実行する!!
%----------
%各章節には以下を書く. 1-03.texのような名前にする
%----------
% \documentclass[
% 		book,
% 		head_space=20mm,
% 		foot_space=20mm,
% 		gutter=10mm,
% 		line_length=190mm
% ]{jlreq}
% \input {preamble.tex}
% \usepackage{docmute} %ファイル分割
% \begin{document}

% %\chapter{章のタイトル}
% \section{節のタイトル}
% no text

% \end{document}
%----------

%main.texには以下を書く
%----------
% \documentclass[
% 		book,
% 		head_space=20mm,
% 		foot_space=20mm,
% 		gutter=10mm,
% 		line_length=190mm,
%         openany
% ]{jlreq}
% \input {preamble.tex}
% \usepackage{docmute} %ファイル分割
% \begin{document}

% %---------- 1章1節
% \input 1-01.tex
% %---------- 1章2節
% \input 1-02.tex
% % ---------- 1章3節
% \input 1-03.tex
% % ---------- 1章4節
% \input 1-04.tex
% % ---------- 1章5節
% \input 1-05.tex
% % ---------- 1章6節
% \input 1-06.tex
% %---------- 1章7節
% \input 1-07.tex
% % ---------- 1章8節
% \input 1-08.tex
% % ---------- 1章9節
% \input 1-09.tex
% % ---------- 1章10節
% \input 1-10.tex
% % ---------- 1章11節
% \input 1-11.tex
% % ---------- 1章12節
% \input 1-12.tex
% % ---------- 参考文献
% \input reference.tex
% \end{document}
% ----------



\usepackage{bxtexlogo}
\usepackage{amsthm}
\usepackage{amsmath}
\usepackage{bbm} %小文字の黒板文字
\usepackage{physics}
\usepackage{amsfonts}
\usepackage{graphicx}
\usepackage{mathtools}
\usepackage{enumitem}
\usepackage[margin=20truemm]{geometry}
\usepackage{textcomp}
\usepackage{bm}
\usepackage{mathrsfs}
\usepackage{latexsym}
\usepackage{amssymb}
\usepackage{algorithmic}
\usepackage{algorithm}
\usepackage{tikz}
\usetikzlibrary{arrows.meta}
\usetikzlibrary{math,matrix,backgrounds}
\usetikzlibrary{angles}
\usetikzlibrary{calc}


%----------
%日本語フォント
% \usepackage[deluxe]{otf} platex用 lualatexでは動かない

%----------
%欧文フォント
\usepackage[T1]{fontenc}

%----------
%文字色
\usepackage{color}

%----------
\setlength{\parindent}{2\zw} %インデントの設定

%----------
% %参照した数式にだけ番号を振る cleverrefと併用するとうまくいかない
% \mathtoolsset{showonlyrefs=true}
%----------

%----------
%集合の中線
\newcommand{\relmiddle}[1]{\mathrel{}\middle#1\mathrel{}}
% \middle| の代わりに \relmiddle| を付ける
\newcommand{\sgn}{\mathop{\mathrm{sgn}}} %置換sgn
\newcommand{\Int}{\mathop{\mathrm{Int}}} %位相空間の内部Int
\newcommand{\Ext}{\mathop{\mathrm{Ext}}} %位相空間の外部Ext
\newcommand{\Cl}{\mathop{\mathrm{Cl}}} %位相空間の閉包Cl
\newcommand{\supp}{\mathop{\mathrm{supp}}} %関数の台supp
\newcommand{\restrict}[2]{\left. #1 \right \vert_{#2}}%関数の制限 \restrict{f}{A} = f|_A
\newcommand{\Ker}{\mathop{\mathrm{Ker}}}
\newcommand{\Coker}{\mathop{\mathrm{Coker}}}
\newcommand{\coker}{\mathop{\mathrm{coker}}}
\newcommand{\Coim}{\mathop{\mathrm{Coim}}}
\newcommand{\coim}{\mathop{\mathrm{coim}}}
\newcommand{\id}{\mathop{\mathrm{id}}}
\newcommand{\Gal}{\mathop{\mathrm{Gal}}}

\newtheorem{definition}{定義}[section]

\usepackage{aliascnt}

% \newaliastheorem{(環境とカウンターの名前)}{(元となるカウンターの名前)}{(表示される文字列)}
\newcommand*{\newaliastheorem}[3]{%
  \newaliascnt{#1}{#2}%
  \newtheorem{#1}[#1]{#3}%
  \aliascntresetthe{#1}%
  \expandafter\newcommand\csname #1autorefname\endcsname{#3}%
}
\newaliastheorem{proposition}{definition}{命題} 
\newaliastheorem{theorem}{definition}{定理}
\newaliastheorem{lemma}{definition}{補題}
\newaliastheorem{corollary}{definition}{系}
\newaliastheorem{example}{definition}{例}
\newaliastheorem{practice}{definition}{演習問題}

\newtheorem*{longproof}{証明}
\newtheorem*{answer}{解答}
\newtheorem*{supplement}{補足}
\newtheorem*{remark}{注意}
%----------

%----------
%古い記法を注意するパッケージ
\RequirePackage[l2tabu, orthodox]{nag}
%----------


% 定理環境(tcolorbox)
\usepackage{tcolorbox} %箱
\tcbuselibrary{breakable,skins,theorems}
\tcolorboxenvironment{definition}{
	blanker,breakable,
	left=3mm,right=3mm,
	top=2mm,bottom=2mm,
	before skip=15pt,after skip=20pt,
	borderline ={0.5pt}{0pt}{black}
}
\newtcolorbox{emptydefinition}{
	blanker,breakable,
	left=3mm,right=3mm,
	top=2mm,bottom=2mm,
	before skip=15pt,after skip=20pt,
	borderline ={0.5pt}{0pt}{black}
}
%----------
\tcolorboxenvironment{proposition}{
	blanker,breakable,
	left=3mm,right=3mm,
	top=3mm,bottom=3mm,
	before skip=15pt,after skip=15pt,
	borderline={0.5pt}{0pt}{black}
}
\newtcolorbox{emptyproposition}{
	blanker,breakable,
	left=3mm,right=3mm,
	top=3mm,bottom=3mm,
	before skip=15pt,after skip=15pt,
	borderline={0.5pt}{0pt}{black}
}
%----------
\tcolorboxenvironment{theorem}{
	blanker,breakable,
	left=3mm,right=3mm,
	top=3mm,bottom=3mm,
    sharp corners,boxrule=0.6pt,
	before skip=15pt,after skip=15pt,
	borderline={0.5pt}{0pt}{black},
    borderline={0.5pt}{1.5pt}{black}
}
\newtcolorbox{emptytheorem}{
	blanker,breakable,
	left=3mm,right=3mm,
	top=3mm,bottom=3mm,
    sharp corners,boxrule=0.6pt,
	before skip=15pt,after skip=15pt,
	borderline={0.5pt}{0pt}{black},
    borderline={0.5pt}{1.5pt}{black}
}
%----------
\tcolorboxenvironment{lemma}{
	blanker,breakable,
	left=3mm,right=3mm,
	top=3mm,bottom=3mm,
	before skip=15pt,after skip=15pt,
	borderline={0.5pt}{0pt}{black}
}
%----------
\tcolorboxenvironment{corollary}{
	blanker,breakable,
	left=3mm,right=3mm,
	top=3mm,bottom=3mm,
	before skip=15pt,after skip=15pt,
	borderline={1.0pt}{0pt}{black,dotted}
}
\newtcolorbox{emptycorollary}{
	blanker,breakable,
	left=3mm,right=3mm,
	top=3mm,bottom=3mm,
	before skip=15pt,after skip=15pt,
	borderline={1.0pt}{0pt}{black,dotted}
}
%----------
\tcolorboxenvironment{example}{
	blanker,breakable,
	left=3mm,right=3mm,
	top=3mm,bottom=3mm,
	before skip=15pt,after skip=15pt,
	borderline={0.5pt}{0pt}{black}
}
%----------
\tcolorboxenvironment{practice}{
	blanker,breakable,
	left=3mm,right=3mm,
	top=3mm,bottom=3mm,
	before skip=15pt,after skip=15pt,
	borderline={0.5pt}{0pt}{black}
}
%----------
\tcolorboxenvironment{proof}{
	blanker,breakable,
	left=3mm,right=3mm,
	top=2mm,bottom=2mm,
	before skip=15pt,after skip=20pt,
	% borderline west={1.5pt}{0pt}{black,dotted}
	borderline vertical={1pt}{0pt}{black,dotted}
	% borderline vertical={0.8pt}{0pt}{black,dotted,arrows={Square[scale=0.5]-Square[scale=0.5]}}
	}
%----------
\tcolorboxenvironment{supplement}{
	blanker,breakable,
	left=3mm,right=3mm,
	top=2mm,bottom=2mm,
	before skip=15pt,after skip=20pt,
	% borderline west={1.5pt}{0pt}{black,dotted}
	% borderline vertical={0.5pt}{0pt}{black,arrows = {Circle[scale=0.7]-Circle[scale=0.7]}}
	borderline vertical={0.5pt}{0pt}{black}
	% borderline vertical={0.5pt}{0pt}{black},
	% borderline north={0.5pt}{0pt}{white,arrows={Circle[black,scale=0.7]-Circle[black,scale=0.7]}}
	}
%----------
\tcolorboxenvironment{remark}{
	blanker,breakable,
	left=3mm,right=3mm,
	top=1mm,bottom=1mm,
	before skip=15pt,after skip=20pt,
	% borderline west={1.5pt}{0pt}{black,dotted}
	% borderline vertical={0.5pt}{0pt}{black,arrows = {Circle[scale=0.7]-Circle[scale=0.7]}}
	borderline vertical={0.5pt}{0pt}{black}
	% borderline vertical={0.5pt}{0pt}{black},
	% borderline north={0.5pt}{0pt}{white,arrows={Circle[black,scale=0.7]-Circle[black,scale=0.7]}}
	}
    
%---------------------
 

%----------
%ハイパーリンク
% 「%」は以降の内容を「改行コードも含めて」無視するコマンド
\usepackage[%
%  dvipdfmx,% 欧文ではコメントアウトする
luatex,%
pdfencoding=auto,%
 setpagesize=false,%
 bookmarks=true,%
 bookmarksdepth=tocdepth,%
 bookmarksnumbered=true,%
 colorlinks=false,%
 pdftitle={},%
 pdfsubject={},%
 pdfauthor={},%
 pdfkeywords={}%
]{hyperref}
%------------


%参照 参照するときに自動で環境名を含んで参照する
\usepackage[nameinlink]{cleveref}
\let\normalref\ref
\renewcommand{\ref}{\cref}
\crefname{definition}{定義}{定義}
\crefname{proposition}{命題}{命題}
\crefname{theorem}{定理}{定理}
\crefname{lemma}{補題}{補題}
\crefname{corollary}{系}{系}
\crefname{example}{例}{例}
\crefname{practice}{演習問題}{演習問題}
\crefname{equation}{式}{式} 
\crefname{chapter}{第}{第}
\creflabelformat{chapter}{#2#1章#3}
\crefname{section}{第}{第}
\creflabelformat{section}{#2#1節#3}
\crefname{subsection}{第}{第}
\creflabelformat{subsection}{#2#1小節#3}
%----------

%---------------------
%章跨ぎの参照が不具合を起こすための代わり
% \mylabl でラベル付け
\newcommand{\mylabel}[1]{
\label{#1}
\hypertarget{#1}{}
}
% \myref で環境名付きリンクをつける
\newcommand{\myref}[1]{
\hyperlink{#1}{\cref*{#1}}
}
%-----------------

\usepackage{autonum} %参照した数式にだけ番号を振る
% \usepackage{docmute} %ファイル分割
% \begin{document}

% %\chapter{章のタイトル}
% \section{節のタイトル}
% no text

% \end{document}
%----------

%main.texには以下を書く
%----------
% \documentclass[
% 		book,
% 		head_space=20mm,
% 		foot_space=20mm,
% 		gutter=10mm,
% 		line_length=190mm,
%         openany
% ]{jlreq}
% 
%----------
%LuaLaTeXで実行する!!
%----------
%各章節には以下を書く. 1-03.texのような名前にする
%----------
% \documentclass[
% 		book,
% 		head_space=20mm,
% 		foot_space=20mm,
% 		gutter=10mm,
% 		line_length=190mm
% ]{jlreq}
% \input {preamble.tex}
% \usepackage{docmute} %ファイル分割
% \begin{document}

% %\chapter{章のタイトル}
% \section{節のタイトル}
% no text

% \end{document}
%----------

%main.texには以下を書く
%----------
% \documentclass[
% 		book,
% 		head_space=20mm,
% 		foot_space=20mm,
% 		gutter=10mm,
% 		line_length=190mm,
%         openany
% ]{jlreq}
% \input {preamble.tex}
% \usepackage{docmute} %ファイル分割
% \begin{document}

% %---------- 1章1節
% \input 1-01.tex
% %---------- 1章2節
% \input 1-02.tex
% % ---------- 1章3節
% \input 1-03.tex
% % ---------- 1章4節
% \input 1-04.tex
% % ---------- 1章5節
% \input 1-05.tex
% % ---------- 1章6節
% \input 1-06.tex
% %---------- 1章7節
% \input 1-07.tex
% % ---------- 1章8節
% \input 1-08.tex
% % ---------- 1章9節
% \input 1-09.tex
% % ---------- 1章10節
% \input 1-10.tex
% % ---------- 1章11節
% \input 1-11.tex
% % ---------- 1章12節
% \input 1-12.tex
% % ---------- 参考文献
% \input reference.tex
% \end{document}
% ----------



\usepackage{bxtexlogo}
\usepackage{amsthm}
\usepackage{amsmath}
\usepackage{bbm} %小文字の黒板文字
\usepackage{physics}
\usepackage{amsfonts}
\usepackage{graphicx}
\usepackage{mathtools}
\usepackage{enumitem}
\usepackage[margin=20truemm]{geometry}
\usepackage{textcomp}
\usepackage{bm}
\usepackage{mathrsfs}
\usepackage{latexsym}
\usepackage{amssymb}
\usepackage{algorithmic}
\usepackage{algorithm}
\usepackage{tikz}
\usetikzlibrary{arrows.meta}
\usetikzlibrary{math,matrix,backgrounds}
\usetikzlibrary{angles}
\usetikzlibrary{calc}


%----------
%日本語フォント
% \usepackage[deluxe]{otf} platex用 lualatexでは動かない

%----------
%欧文フォント
\usepackage[T1]{fontenc}

%----------
%文字色
\usepackage{color}

%----------
\setlength{\parindent}{2\zw} %インデントの設定

%----------
% %参照した数式にだけ番号を振る cleverrefと併用するとうまくいかない
% \mathtoolsset{showonlyrefs=true}
%----------

%----------
%集合の中線
\newcommand{\relmiddle}[1]{\mathrel{}\middle#1\mathrel{}}
% \middle| の代わりに \relmiddle| を付ける
\newcommand{\sgn}{\mathop{\mathrm{sgn}}} %置換sgn
\newcommand{\Int}{\mathop{\mathrm{Int}}} %位相空間の内部Int
\newcommand{\Ext}{\mathop{\mathrm{Ext}}} %位相空間の外部Ext
\newcommand{\Cl}{\mathop{\mathrm{Cl}}} %位相空間の閉包Cl
\newcommand{\supp}{\mathop{\mathrm{supp}}} %関数の台supp
\newcommand{\restrict}[2]{\left. #1 \right \vert_{#2}}%関数の制限 \restrict{f}{A} = f|_A
\newcommand{\Ker}{\mathop{\mathrm{Ker}}}
\newcommand{\Coker}{\mathop{\mathrm{Coker}}}
\newcommand{\coker}{\mathop{\mathrm{coker}}}
\newcommand{\Coim}{\mathop{\mathrm{Coim}}}
\newcommand{\coim}{\mathop{\mathrm{coim}}}
\newcommand{\id}{\mathop{\mathrm{id}}}
\newcommand{\Gal}{\mathop{\mathrm{Gal}}}

\newtheorem{definition}{定義}[section]

\usepackage{aliascnt}

% \newaliastheorem{(環境とカウンターの名前)}{(元となるカウンターの名前)}{(表示される文字列)}
\newcommand*{\newaliastheorem}[3]{%
  \newaliascnt{#1}{#2}%
  \newtheorem{#1}[#1]{#3}%
  \aliascntresetthe{#1}%
  \expandafter\newcommand\csname #1autorefname\endcsname{#3}%
}
\newaliastheorem{proposition}{definition}{命題} 
\newaliastheorem{theorem}{definition}{定理}
\newaliastheorem{lemma}{definition}{補題}
\newaliastheorem{corollary}{definition}{系}
\newaliastheorem{example}{definition}{例}
\newaliastheorem{practice}{definition}{演習問題}

\newtheorem*{longproof}{証明}
\newtheorem*{answer}{解答}
\newtheorem*{supplement}{補足}
\newtheorem*{remark}{注意}
%----------

%----------
%古い記法を注意するパッケージ
\RequirePackage[l2tabu, orthodox]{nag}
%----------


% 定理環境(tcolorbox)
\usepackage{tcolorbox} %箱
\tcbuselibrary{breakable,skins,theorems}
\tcolorboxenvironment{definition}{
	blanker,breakable,
	left=3mm,right=3mm,
	top=2mm,bottom=2mm,
	before skip=15pt,after skip=20pt,
	borderline ={0.5pt}{0pt}{black}
}
\newtcolorbox{emptydefinition}{
	blanker,breakable,
	left=3mm,right=3mm,
	top=2mm,bottom=2mm,
	before skip=15pt,after skip=20pt,
	borderline ={0.5pt}{0pt}{black}
}
%----------
\tcolorboxenvironment{proposition}{
	blanker,breakable,
	left=3mm,right=3mm,
	top=3mm,bottom=3mm,
	before skip=15pt,after skip=15pt,
	borderline={0.5pt}{0pt}{black}
}
\newtcolorbox{emptyproposition}{
	blanker,breakable,
	left=3mm,right=3mm,
	top=3mm,bottom=3mm,
	before skip=15pt,after skip=15pt,
	borderline={0.5pt}{0pt}{black}
}
%----------
\tcolorboxenvironment{theorem}{
	blanker,breakable,
	left=3mm,right=3mm,
	top=3mm,bottom=3mm,
    sharp corners,boxrule=0.6pt,
	before skip=15pt,after skip=15pt,
	borderline={0.5pt}{0pt}{black},
    borderline={0.5pt}{1.5pt}{black}
}
\newtcolorbox{emptytheorem}{
	blanker,breakable,
	left=3mm,right=3mm,
	top=3mm,bottom=3mm,
    sharp corners,boxrule=0.6pt,
	before skip=15pt,after skip=15pt,
	borderline={0.5pt}{0pt}{black},
    borderline={0.5pt}{1.5pt}{black}
}
%----------
\tcolorboxenvironment{lemma}{
	blanker,breakable,
	left=3mm,right=3mm,
	top=3mm,bottom=3mm,
	before skip=15pt,after skip=15pt,
	borderline={0.5pt}{0pt}{black}
}
%----------
\tcolorboxenvironment{corollary}{
	blanker,breakable,
	left=3mm,right=3mm,
	top=3mm,bottom=3mm,
	before skip=15pt,after skip=15pt,
	borderline={1.0pt}{0pt}{black,dotted}
}
\newtcolorbox{emptycorollary}{
	blanker,breakable,
	left=3mm,right=3mm,
	top=3mm,bottom=3mm,
	before skip=15pt,after skip=15pt,
	borderline={1.0pt}{0pt}{black,dotted}
}
%----------
\tcolorboxenvironment{example}{
	blanker,breakable,
	left=3mm,right=3mm,
	top=3mm,bottom=3mm,
	before skip=15pt,after skip=15pt,
	borderline={0.5pt}{0pt}{black}
}
%----------
\tcolorboxenvironment{practice}{
	blanker,breakable,
	left=3mm,right=3mm,
	top=3mm,bottom=3mm,
	before skip=15pt,after skip=15pt,
	borderline={0.5pt}{0pt}{black}
}
%----------
\tcolorboxenvironment{proof}{
	blanker,breakable,
	left=3mm,right=3mm,
	top=2mm,bottom=2mm,
	before skip=15pt,after skip=20pt,
	% borderline west={1.5pt}{0pt}{black,dotted}
	borderline vertical={1pt}{0pt}{black,dotted}
	% borderline vertical={0.8pt}{0pt}{black,dotted,arrows={Square[scale=0.5]-Square[scale=0.5]}}
	}
%----------
\tcolorboxenvironment{supplement}{
	blanker,breakable,
	left=3mm,right=3mm,
	top=2mm,bottom=2mm,
	before skip=15pt,after skip=20pt,
	% borderline west={1.5pt}{0pt}{black,dotted}
	% borderline vertical={0.5pt}{0pt}{black,arrows = {Circle[scale=0.7]-Circle[scale=0.7]}}
	borderline vertical={0.5pt}{0pt}{black}
	% borderline vertical={0.5pt}{0pt}{black},
	% borderline north={0.5pt}{0pt}{white,arrows={Circle[black,scale=0.7]-Circle[black,scale=0.7]}}
	}
%----------
\tcolorboxenvironment{remark}{
	blanker,breakable,
	left=3mm,right=3mm,
	top=1mm,bottom=1mm,
	before skip=15pt,after skip=20pt,
	% borderline west={1.5pt}{0pt}{black,dotted}
	% borderline vertical={0.5pt}{0pt}{black,arrows = {Circle[scale=0.7]-Circle[scale=0.7]}}
	borderline vertical={0.5pt}{0pt}{black}
	% borderline vertical={0.5pt}{0pt}{black},
	% borderline north={0.5pt}{0pt}{white,arrows={Circle[black,scale=0.7]-Circle[black,scale=0.7]}}
	}
    
%---------------------
 

%----------
%ハイパーリンク
% 「%」は以降の内容を「改行コードも含めて」無視するコマンド
\usepackage[%
%  dvipdfmx,% 欧文ではコメントアウトする
luatex,%
pdfencoding=auto,%
 setpagesize=false,%
 bookmarks=true,%
 bookmarksdepth=tocdepth,%
 bookmarksnumbered=true,%
 colorlinks=false,%
 pdftitle={},%
 pdfsubject={},%
 pdfauthor={},%
 pdfkeywords={}%
]{hyperref}
%------------


%参照 参照するときに自動で環境名を含んで参照する
\usepackage[nameinlink]{cleveref}
\let\normalref\ref
\renewcommand{\ref}{\cref}
\crefname{definition}{定義}{定義}
\crefname{proposition}{命題}{命題}
\crefname{theorem}{定理}{定理}
\crefname{lemma}{補題}{補題}
\crefname{corollary}{系}{系}
\crefname{example}{例}{例}
\crefname{practice}{演習問題}{演習問題}
\crefname{equation}{式}{式} 
\crefname{chapter}{第}{第}
\creflabelformat{chapter}{#2#1章#3}
\crefname{section}{第}{第}
\creflabelformat{section}{#2#1節#3}
\crefname{subsection}{第}{第}
\creflabelformat{subsection}{#2#1小節#3}
%----------

%---------------------
%章跨ぎの参照が不具合を起こすための代わり
% \mylabl でラベル付け
\newcommand{\mylabel}[1]{
\label{#1}
\hypertarget{#1}{}
}
% \myref で環境名付きリンクをつける
\newcommand{\myref}[1]{
\hyperlink{#1}{\cref*{#1}}
}
%-----------------

\usepackage{autonum} %参照した数式にだけ番号を振る
% \usepackage{docmute} %ファイル分割
% \begin{document}

% %---------- 1章1節
% \input 1-01.tex
% %---------- 1章2節
% \input 1-02.tex
% % ---------- 1章3節
% \input 1-03.tex
% % ---------- 1章4節
% \input 1-04.tex
% % ---------- 1章5節
% \input 1-05.tex
% % ---------- 1章6節
% \input 1-06.tex
% %---------- 1章7節
% \input 1-07.tex
% % ---------- 1章8節
% \input 1-08.tex
% % ---------- 1章9節
% \input 1-09.tex
% % ---------- 1章10節
% \input 1-10.tex
% % ---------- 1章11節
% \input 1-11.tex
% % ---------- 1章12節
% \input 1-12.tex
% % ---------- 参考文献
% \input reference.tex
% \end{document}
% ----------



\usepackage{bxtexlogo}
\usepackage{amsthm}
\usepackage{amsmath}
\usepackage{bbm} %小文字の黒板文字
\usepackage{physics}
\usepackage{amsfonts}
\usepackage{graphicx}
\usepackage{mathtools}
\usepackage{enumitem}
\usepackage[margin=20truemm]{geometry}
\usepackage{textcomp}
\usepackage{bm}
\usepackage{mathrsfs}
\usepackage{latexsym}
\usepackage{amssymb}
\usepackage{algorithmic}
\usepackage{algorithm}
\usepackage{tikz}
\usetikzlibrary{arrows.meta}
\usetikzlibrary{math,matrix,backgrounds}
\usetikzlibrary{angles}
\usetikzlibrary{calc}


%----------
%日本語フォント
% \usepackage[deluxe]{otf} platex用 lualatexでは動かない

%----------
%欧文フォント
\usepackage[T1]{fontenc}

%----------
%文字色
\usepackage{color}

%----------
\setlength{\parindent}{2\zw} %インデントの設定

%----------
% %参照した数式にだけ番号を振る cleverrefと併用するとうまくいかない
% \mathtoolsset{showonlyrefs=true}
%----------

%----------
%集合の中線
\newcommand{\relmiddle}[1]{\mathrel{}\middle#1\mathrel{}}
% \middle| の代わりに \relmiddle| を付ける
\newcommand{\sgn}{\mathop{\mathrm{sgn}}} %置換sgn
\newcommand{\Int}{\mathop{\mathrm{Int}}} %位相空間の内部Int
\newcommand{\Ext}{\mathop{\mathrm{Ext}}} %位相空間の外部Ext
\newcommand{\Cl}{\mathop{\mathrm{Cl}}} %位相空間の閉包Cl
\newcommand{\supp}{\mathop{\mathrm{supp}}} %関数の台supp
\newcommand{\restrict}[2]{\left. #1 \right \vert_{#2}}%関数の制限 \restrict{f}{A} = f|_A
\newcommand{\Ker}{\mathop{\mathrm{Ker}}}
\newcommand{\Coker}{\mathop{\mathrm{Coker}}}
\newcommand{\coker}{\mathop{\mathrm{coker}}}
\newcommand{\Coim}{\mathop{\mathrm{Coim}}}
\newcommand{\coim}{\mathop{\mathrm{coim}}}
\newcommand{\id}{\mathop{\mathrm{id}}}
\newcommand{\Gal}{\mathop{\mathrm{Gal}}}

\newtheorem{definition}{定義}[section]

\usepackage{aliascnt}

% \newaliastheorem{(環境とカウンターの名前)}{(元となるカウンターの名前)}{(表示される文字列)}
\newcommand*{\newaliastheorem}[3]{%
  \newaliascnt{#1}{#2}%
  \newtheorem{#1}[#1]{#3}%
  \aliascntresetthe{#1}%
  \expandafter\newcommand\csname #1autorefname\endcsname{#3}%
}
\newaliastheorem{proposition}{definition}{命題} 
\newaliastheorem{theorem}{definition}{定理}
\newaliastheorem{lemma}{definition}{補題}
\newaliastheorem{corollary}{definition}{系}
\newaliastheorem{example}{definition}{例}
\newaliastheorem{practice}{definition}{演習問題}

\newtheorem*{longproof}{証明}
\newtheorem*{answer}{解答}
\newtheorem*{supplement}{補足}
\newtheorem*{remark}{注意}
%----------

%----------
%古い記法を注意するパッケージ
\RequirePackage[l2tabu, orthodox]{nag}
%----------


% 定理環境(tcolorbox)
\usepackage{tcolorbox} %箱
\tcbuselibrary{breakable,skins,theorems}
\tcolorboxenvironment{definition}{
	blanker,breakable,
	left=3mm,right=3mm,
	top=2mm,bottom=2mm,
	before skip=15pt,after skip=20pt,
	borderline ={0.5pt}{0pt}{black}
}
\newtcolorbox{emptydefinition}{
	blanker,breakable,
	left=3mm,right=3mm,
	top=2mm,bottom=2mm,
	before skip=15pt,after skip=20pt,
	borderline ={0.5pt}{0pt}{black}
}
%----------
\tcolorboxenvironment{proposition}{
	blanker,breakable,
	left=3mm,right=3mm,
	top=3mm,bottom=3mm,
	before skip=15pt,after skip=15pt,
	borderline={0.5pt}{0pt}{black}
}
\newtcolorbox{emptyproposition}{
	blanker,breakable,
	left=3mm,right=3mm,
	top=3mm,bottom=3mm,
	before skip=15pt,after skip=15pt,
	borderline={0.5pt}{0pt}{black}
}
%----------
\tcolorboxenvironment{theorem}{
	blanker,breakable,
	left=3mm,right=3mm,
	top=3mm,bottom=3mm,
    sharp corners,boxrule=0.6pt,
	before skip=15pt,after skip=15pt,
	borderline={0.5pt}{0pt}{black},
    borderline={0.5pt}{1.5pt}{black}
}
\newtcolorbox{emptytheorem}{
	blanker,breakable,
	left=3mm,right=3mm,
	top=3mm,bottom=3mm,
    sharp corners,boxrule=0.6pt,
	before skip=15pt,after skip=15pt,
	borderline={0.5pt}{0pt}{black},
    borderline={0.5pt}{1.5pt}{black}
}
%----------
\tcolorboxenvironment{lemma}{
	blanker,breakable,
	left=3mm,right=3mm,
	top=3mm,bottom=3mm,
	before skip=15pt,after skip=15pt,
	borderline={0.5pt}{0pt}{black}
}
%----------
\tcolorboxenvironment{corollary}{
	blanker,breakable,
	left=3mm,right=3mm,
	top=3mm,bottom=3mm,
	before skip=15pt,after skip=15pt,
	borderline={1.0pt}{0pt}{black,dotted}
}
\newtcolorbox{emptycorollary}{
	blanker,breakable,
	left=3mm,right=3mm,
	top=3mm,bottom=3mm,
	before skip=15pt,after skip=15pt,
	borderline={1.0pt}{0pt}{black,dotted}
}
%----------
\tcolorboxenvironment{example}{
	blanker,breakable,
	left=3mm,right=3mm,
	top=3mm,bottom=3mm,
	before skip=15pt,after skip=15pt,
	borderline={0.5pt}{0pt}{black}
}
%----------
\tcolorboxenvironment{practice}{
	blanker,breakable,
	left=3mm,right=3mm,
	top=3mm,bottom=3mm,
	before skip=15pt,after skip=15pt,
	borderline={0.5pt}{0pt}{black}
}
%----------
\tcolorboxenvironment{proof}{
	blanker,breakable,
	left=3mm,right=3mm,
	top=2mm,bottom=2mm,
	before skip=15pt,after skip=20pt,
	% borderline west={1.5pt}{0pt}{black,dotted}
	borderline vertical={1pt}{0pt}{black,dotted}
	% borderline vertical={0.8pt}{0pt}{black,dotted,arrows={Square[scale=0.5]-Square[scale=0.5]}}
	}
%----------
\tcolorboxenvironment{supplement}{
	blanker,breakable,
	left=3mm,right=3mm,
	top=2mm,bottom=2mm,
	before skip=15pt,after skip=20pt,
	% borderline west={1.5pt}{0pt}{black,dotted}
	% borderline vertical={0.5pt}{0pt}{black,arrows = {Circle[scale=0.7]-Circle[scale=0.7]}}
	borderline vertical={0.5pt}{0pt}{black}
	% borderline vertical={0.5pt}{0pt}{black},
	% borderline north={0.5pt}{0pt}{white,arrows={Circle[black,scale=0.7]-Circle[black,scale=0.7]}}
	}
%----------
\tcolorboxenvironment{remark}{
	blanker,breakable,
	left=3mm,right=3mm,
	top=1mm,bottom=1mm,
	before skip=15pt,after skip=20pt,
	% borderline west={1.5pt}{0pt}{black,dotted}
	% borderline vertical={0.5pt}{0pt}{black,arrows = {Circle[scale=0.7]-Circle[scale=0.7]}}
	borderline vertical={0.5pt}{0pt}{black}
	% borderline vertical={0.5pt}{0pt}{black},
	% borderline north={0.5pt}{0pt}{white,arrows={Circle[black,scale=0.7]-Circle[black,scale=0.7]}}
	}
    
%---------------------
 

%----------
%ハイパーリンク
% 「%」は以降の内容を「改行コードも含めて」無視するコマンド
\usepackage[%
%  dvipdfmx,% 欧文ではコメントアウトする
luatex,%
pdfencoding=auto,%
 setpagesize=false,%
 bookmarks=true,%
 bookmarksdepth=tocdepth,%
 bookmarksnumbered=true,%
 colorlinks=false,%
 pdftitle={},%
 pdfsubject={},%
 pdfauthor={},%
 pdfkeywords={}%
]{hyperref}
%------------


%参照 参照するときに自動で環境名を含んで参照する
\usepackage[nameinlink]{cleveref}
\let\normalref\ref
\renewcommand{\ref}{\cref}
\crefname{definition}{定義}{定義}
\crefname{proposition}{命題}{命題}
\crefname{theorem}{定理}{定理}
\crefname{lemma}{補題}{補題}
\crefname{corollary}{系}{系}
\crefname{example}{例}{例}
\crefname{practice}{演習問題}{演習問題}
\crefname{equation}{式}{式} 
\crefname{chapter}{第}{第}
\creflabelformat{chapter}{#2#1章#3}
\crefname{section}{第}{第}
\creflabelformat{section}{#2#1節#3}
\crefname{subsection}{第}{第}
\creflabelformat{subsection}{#2#1小節#3}
%----------

%---------------------
%章跨ぎの参照が不具合を起こすための代わり
% \mylabl でラベル付け
\newcommand{\mylabel}[1]{
\label{#1}
\hypertarget{#1}{}
}
% \myref で環境名付きリンクをつける
\newcommand{\myref}[1]{
\hyperlink{#1}{\cref*{#1}}
}
%-----------------

\usepackage{autonum} %参照した数式にだけ番号を振る
% \usepackage{docmute} %ファイル分割
% \begin{document}

% %---------- 1章1節
% \input 1-01.tex
% %---------- 1章2節
% \input 1-02.tex
% % ---------- 1章3節
% \input 1-03.tex
% % ---------- 1章4節
% \input 1-04.tex
% % ---------- 1章5節
% \input 1-05.tex
% % ---------- 1章6節
% \input 1-06.tex
% %---------- 1章7節
% \input 1-07.tex
% % ---------- 1章8節
% \input 1-08.tex
% % ---------- 1章9節
% \input 1-09.tex
% % ---------- 1章10節
% \input 1-10.tex
% % ---------- 1章11節
% \input 1-11.tex
% % ---------- 1章12節
% \input 1-12.tex
% % ---------- 参考文献
% \input reference.tex
% \end{document}
% ----------



\usepackage{bxtexlogo}
\usepackage{amsthm}
\usepackage{amsmath}
\usepackage{bbm} %小文字の黒板文字
\usepackage{physics}
\usepackage{amsfonts}
\usepackage{graphicx}
\usepackage{mathtools}
\usepackage{enumitem}
\usepackage[margin=20truemm]{geometry}
\usepackage{textcomp}
\usepackage{bm}
\usepackage{mathrsfs}
\usepackage{latexsym}
\usepackage{amssymb}
\usepackage{algorithmic}
\usepackage{algorithm}
\usepackage{tikz}
\usetikzlibrary{arrows.meta}
\usetikzlibrary{math,matrix,backgrounds}
\usetikzlibrary{angles}
\usetikzlibrary{calc}


%----------
%日本語フォント
% \usepackage[deluxe]{otf} platex用 lualatexでは動かない

%----------
%欧文フォント
\usepackage[T1]{fontenc}

%----------
%文字色
\usepackage{color}

%----------
\setlength{\parindent}{2\zw} %インデントの設定

%----------
% %参照した数式にだけ番号を振る cleverrefと併用するとうまくいかない
% \mathtoolsset{showonlyrefs=true}
%----------

%----------
%集合の中線
\newcommand{\relmiddle}[1]{\mathrel{}\middle#1\mathrel{}}
% \middle| の代わりに \relmiddle| を付ける
\newcommand{\sgn}{\mathop{\mathrm{sgn}}} %置換sgn
\newcommand{\Int}{\mathop{\mathrm{Int}}} %位相空間の内部Int
\newcommand{\Ext}{\mathop{\mathrm{Ext}}} %位相空間の外部Ext
\newcommand{\Cl}{\mathop{\mathrm{Cl}}} %位相空間の閉包Cl
\newcommand{\supp}{\mathop{\mathrm{supp}}} %関数の台supp
\newcommand{\restrict}[2]{\left. #1 \right \vert_{#2}}%関数の制限 \restrict{f}{A} = f|_A
\newcommand{\Ker}{\mathop{\mathrm{Ker}}}
\newcommand{\Coker}{\mathop{\mathrm{Coker}}}
\newcommand{\coker}{\mathop{\mathrm{coker}}}
\newcommand{\Coim}{\mathop{\mathrm{Coim}}}
\newcommand{\coim}{\mathop{\mathrm{coim}}}
\newcommand{\id}{\mathop{\mathrm{id}}}
\newcommand{\Gal}{\mathop{\mathrm{Gal}}}

\newtheorem{definition}{定義}[section]

\usepackage{aliascnt}

% \newaliastheorem{(環境とカウンターの名前)}{(元となるカウンターの名前)}{(表示される文字列)}
\newcommand*{\newaliastheorem}[3]{%
  \newaliascnt{#1}{#2}%
  \newtheorem{#1}[#1]{#3}%
  \aliascntresetthe{#1}%
  \expandafter\newcommand\csname #1autorefname\endcsname{#3}%
}
\newaliastheorem{proposition}{definition}{命題} 
\newaliastheorem{theorem}{definition}{定理}
\newaliastheorem{lemma}{definition}{補題}
\newaliastheorem{corollary}{definition}{系}
\newaliastheorem{example}{definition}{例}
\newaliastheorem{practice}{definition}{演習問題}

\newtheorem*{longproof}{証明}
\newtheorem*{answer}{解答}
\newtheorem*{supplement}{補足}
\newtheorem*{remark}{注意}
%----------

%----------
%古い記法を注意するパッケージ
\RequirePackage[l2tabu, orthodox]{nag}
%----------


% 定理環境(tcolorbox)
\usepackage{tcolorbox} %箱
\tcbuselibrary{breakable,skins,theorems}
\tcolorboxenvironment{definition}{
	blanker,breakable,
	left=3mm,right=3mm,
	top=2mm,bottom=2mm,
	before skip=15pt,after skip=20pt,
	borderline ={0.5pt}{0pt}{black}
}
\newtcolorbox{emptydefinition}{
	blanker,breakable,
	left=3mm,right=3mm,
	top=2mm,bottom=2mm,
	before skip=15pt,after skip=20pt,
	borderline ={0.5pt}{0pt}{black}
}
%----------
\tcolorboxenvironment{proposition}{
	blanker,breakable,
	left=3mm,right=3mm,
	top=3mm,bottom=3mm,
	before skip=15pt,after skip=15pt,
	borderline={0.5pt}{0pt}{black}
}
\newtcolorbox{emptyproposition}{
	blanker,breakable,
	left=3mm,right=3mm,
	top=3mm,bottom=3mm,
	before skip=15pt,after skip=15pt,
	borderline={0.5pt}{0pt}{black}
}
%----------
\tcolorboxenvironment{theorem}{
	blanker,breakable,
	left=3mm,right=3mm,
	top=3mm,bottom=3mm,
    sharp corners,boxrule=0.6pt,
	before skip=15pt,after skip=15pt,
	borderline={0.5pt}{0pt}{black},
    borderline={0.5pt}{1.5pt}{black}
}
\newtcolorbox{emptytheorem}{
	blanker,breakable,
	left=3mm,right=3mm,
	top=3mm,bottom=3mm,
    sharp corners,boxrule=0.6pt,
	before skip=15pt,after skip=15pt,
	borderline={0.5pt}{0pt}{black},
    borderline={0.5pt}{1.5pt}{black}
}
%----------
\tcolorboxenvironment{lemma}{
	blanker,breakable,
	left=3mm,right=3mm,
	top=3mm,bottom=3mm,
	before skip=15pt,after skip=15pt,
	borderline={0.5pt}{0pt}{black}
}
%----------
\tcolorboxenvironment{corollary}{
	blanker,breakable,
	left=3mm,right=3mm,
	top=3mm,bottom=3mm,
	before skip=15pt,after skip=15pt,
	borderline={1.0pt}{0pt}{black,dotted}
}
\newtcolorbox{emptycorollary}{
	blanker,breakable,
	left=3mm,right=3mm,
	top=3mm,bottom=3mm,
	before skip=15pt,after skip=15pt,
	borderline={1.0pt}{0pt}{black,dotted}
}
%----------
\tcolorboxenvironment{example}{
	blanker,breakable,
	left=3mm,right=3mm,
	top=3mm,bottom=3mm,
	before skip=15pt,after skip=15pt,
	borderline={0.5pt}{0pt}{black}
}
%----------
\tcolorboxenvironment{practice}{
	blanker,breakable,
	left=3mm,right=3mm,
	top=3mm,bottom=3mm,
	before skip=15pt,after skip=15pt,
	borderline={0.5pt}{0pt}{black}
}
%----------
\tcolorboxenvironment{proof}{
	blanker,breakable,
	left=3mm,right=3mm,
	top=2mm,bottom=2mm,
	before skip=15pt,after skip=20pt,
	% borderline west={1.5pt}{0pt}{black,dotted}
	borderline vertical={1pt}{0pt}{black,dotted}
	% borderline vertical={0.8pt}{0pt}{black,dotted,arrows={Square[scale=0.5]-Square[scale=0.5]}}
	}
%----------
\tcolorboxenvironment{supplement}{
	blanker,breakable,
	left=3mm,right=3mm,
	top=2mm,bottom=2mm,
	before skip=15pt,after skip=20pt,
	% borderline west={1.5pt}{0pt}{black,dotted}
	% borderline vertical={0.5pt}{0pt}{black,arrows = {Circle[scale=0.7]-Circle[scale=0.7]}}
	borderline vertical={0.5pt}{0pt}{black}
	% borderline vertical={0.5pt}{0pt}{black},
	% borderline north={0.5pt}{0pt}{white,arrows={Circle[black,scale=0.7]-Circle[black,scale=0.7]}}
	}
%----------
\tcolorboxenvironment{remark}{
	blanker,breakable,
	left=3mm,right=3mm,
	top=1mm,bottom=1mm,
	before skip=15pt,after skip=20pt,
	% borderline west={1.5pt}{0pt}{black,dotted}
	% borderline vertical={0.5pt}{0pt}{black,arrows = {Circle[scale=0.7]-Circle[scale=0.7]}}
	borderline vertical={0.5pt}{0pt}{black}
	% borderline vertical={0.5pt}{0pt}{black},
	% borderline north={0.5pt}{0pt}{white,arrows={Circle[black,scale=0.7]-Circle[black,scale=0.7]}}
	}
    
%---------------------
 

%----------
%ハイパーリンク
% 「%」は以降の内容を「改行コードも含めて」無視するコマンド
\usepackage[%
%  dvipdfmx,% 欧文ではコメントアウトする
luatex,%
pdfencoding=auto,%
 setpagesize=false,%
 bookmarks=true,%
 bookmarksdepth=tocdepth,%
 bookmarksnumbered=true,%
 colorlinks=false,%
 pdftitle={},%
 pdfsubject={},%
 pdfauthor={},%
 pdfkeywords={}%
]{hyperref}
%------------


%参照 参照するときに自動で環境名を含んで参照する
\usepackage[nameinlink]{cleveref}
\let\normalref\ref
\renewcommand{\ref}{\cref}
\crefname{definition}{定義}{定義}
\crefname{proposition}{命題}{命題}
\crefname{theorem}{定理}{定理}
\crefname{lemma}{補題}{補題}
\crefname{corollary}{系}{系}
\crefname{example}{例}{例}
\crefname{practice}{演習問題}{演習問題}
\crefname{equation}{式}{式} 
\crefname{chapter}{第}{第}
\creflabelformat{chapter}{#2#1章#3}
\crefname{section}{第}{第}
\creflabelformat{section}{#2#1節#3}
\crefname{subsection}{第}{第}
\creflabelformat{subsection}{#2#1小節#3}
%----------

%---------------------
%章跨ぎの参照が不具合を起こすための代わり
% \mylabl でラベル付け
\newcommand{\mylabel}[1]{
\label{#1}
\hypertarget{#1}{}
}
% \myref で環境名付きリンクをつける
\newcommand{\myref}[1]{
\hyperlink{#1}{\cref*{#1}}
}
%-----------------

\usepackage{autonum} %参照した数式にだけ番号を振る
\usepackage{docmute} %ファイル分割
\begin{document}

%\chapter{章のタイトル}
\section{H16数学A}
\fbox{1}
$\lambda,\mu$の固有空間をそれぞれ$V_\lambda,V_\mu$とする.対角化可能であるから$V=V_\lambda \oplus V_\mu$である.$W_\lambda=W \cap V_\lambda,W_\mu=W \cap V_\mu$とする.$W_\lambda\cap W_\mu=W\cap V_\lambda \cap V_\mu=\{0\}$である.また$w\in W$に対して$w=w_\lambda+w_\mu$となる$w_\lambda\in V_\lambda, w_\mu \in V_\mu$が一意的に存在する.$W\ni f(w) -\mu w=(\lambda-\mu)w_\lambda $より$w_\lambda\in W_\lambda$である.同様に$w_\mu\in W_\mu$である.よって$W=W_\lambda \oplus W_\mu$である.$W$を$f$の固有空間の直和に分解できたから$f|_W$は対角化可能である.

\fbox{2}
(1)$X$のコンパクト集合$C$をとる.$x \in X\setminus C$を一つ固定する.各$y \in C$に対して$x \in U_y,y\in V_y,U_y \cap V_y=\emptyset$となる開集合$U_y,V_y$が存在する.$\{ V_y \mid y \in C\}$は$C$の開被覆であるから有限部分集合$C'\subset C$が存在して$C\subset \bigcup_{y\in C'}V_y$となる.$U=\bigcap_{y\in C'}U_y$とする.$U$は$x$の開近傍であり$U \subset X \setminus C$であるから$x$は$C$の外点.任意の$x$でなりたつから$C$は閉集合である.

(2)$A \cap B$の開被覆$S= \{ U_\lambda \mid \lambda \in \Lambda\}$を任意にとる.
$S \cup \{ X\setminus A\}$は$B$の開被覆である.したがって有限部分集合$\Lambda' \subset \Lambda$が存在して$B \subset \bigcup_{\lambda \in \Lambda'}U_\lambda \cup (X\setminus A)$となる.$A \cap B \subset \bigcup_{\lambda \in \Lambda'}U_\lambda$である.したがって$A \cap B$はコンパクト集合である.

\fbox{3}
(1)$G(x)=\int_0^xf(x,y)dy$とする.
\begin{align}
 \frac{G(x+h)-G(x)}{h} &=  \int_0^{x+h} \frac{f(x+h,y)}{h}dy - \int_0^x \frac{f(x,y)}{h}dy =  \int_x^{x+h} \frac{f(x,y)}{h}dy + \int_0^{x+h} \frac{f(x+h,y)-f(x,y)}{h}dy \\
&= \int_0^h  \frac{f(x,y+x)}{h} dy  +  \int_0^x \frac{f(x+h,y)-f(x,y)}{h}dy+ \int_x^{x+h} \frac{f(x+h,y)-f(x,y)}{h}dy \\
\end{align}
である.第一項は$\lim\limits_{h\to 0}\int_0^h  \frac{f(x,y+x)}{h} dy =\frac{\partial }{\partial h}\int_0^h  f(x,y+x)dy =f(x,x)$である.

第二項は$\frac{f(x+h,y)-f(x,y)}{h}=\frac{\partial f}{\partial x}(x+\theta h,y)$となる$\theta \in (0,1)$が存在して$ \int_0^x \frac{f(x+h,y)-f(x,y)}{h}dy\le  \int_0^x \frac{\partial f}{\partial x}(x+\theta h,y)dy\le \infty$であるから優収束定理より$\lim\limits_{h\to 0} \int_0^x \frac{f(x+h,y)-f(x,y)}{h}dy=\int_0^x \frac{\partial f}{\partial x}(x,y)dy$である.

第三項はある$0$の近傍で$\left|  \frac{f(x+h,y)-f(x,y)}{h}\right|\le M$であるから$\lim\limits_{h\to 0} \int_x^{x+h} \frac{f(x+h,y)-f(x,y)}{h}dy \le\lim\limits_{h\to 0}\int_x^{x+h} Mdy=0$である.よって$G'(x)=f(x,x)+\int_0^x \frac{\partial f}{\partial x}(x,y)dy$である.

したがって$F'(x)=f(x,x)-f(x,-x)+\int_{-x}^x \frac{\partial f}{\partial x}(x,y)dy$である.さらに$F''(x)=\frac{\partial f}{\partial x}(x,x)+\frac{\partial f}{\partial y}(x,x)-\frac{\partial f}{\partial x}(x,-x)+\frac{\partial f}{\partial y}(x,-x)+\frac{\partial f}{\partial x}(x,x)-\frac{\partial f}{\partial x}(x,-x)+\int_{-x}^x \frac{\partial^2 f}{\partial x^2}(x,y)dy=2\frac{\partial f}{\partial x}(x,x)-2\frac{\partial f}{\partial x}(x,-x)+\frac{\partial f}{\partial y}(x,x)+\frac{\partial f}{\partial y}(x,-x)+\int_{-x}^x \frac{\partial^2 f}{\partial x^2}(x,y)dy$である.

\fbox{4}
$f(z)=e^{iz}/(z-i\varepsilon)$とすれば,$f$は$z\neq i\varepsilon$で正則である.積分経路$C$を原点中心の半径$R>2\varepsilon$の上半平面の半円とする.$C_1$を実軸上の$-R$から$R$までの部分,$C_2$を半円とする.$f$の$C$での積分は留数定理から$\int_C f(z)dz=2\pi i \mathrm{Res}(f,i\varepsilon)=2\pi i e^{-\varepsilon}$である.
$C_2$での積分は$z=Re^i\theta \quad(0\le \theta\le \pi)$とすると,
$|\int_{C_2}f(z)dz|\le\int_{0}^{\pi}e^{-R\sin \theta}/Rd\theta\rightarrow 0\quad (R\to \infty)$である.したがって$\int_{-\infty}^{\infty}f(x)dx=2\pi i e^{-\varepsilon}$より$\frac{1}{2\pi i}\int_{-\infty}^{\infty}f(x)dx=e^{-\varepsilon}$である.





\end{document}